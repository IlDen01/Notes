\documentclass{article}

\input{C:/Users/ilden/Documents/School/TeX памятка/header.tex}

\begin{document}
	\tableofcontents
	\setcounter{tocdepth}{3}
	\newpage
	\section{Первая мировая война.}
	\subsection{До войны.}
	$40$ лет до Первой мировой войны в Европе войн не было (только на Балканах). Одна из причин начала первой мировой войны~--- колониальный вопрос. Также Германскую Империю волновало начало перевооружения в Российской Империи. Основные участники Антанты, противостоящей Германской Империи, Османской Империи и их союзникам: Франция, Россия, Великобритания. Позже подключились Соединенные Штаты Америки. Балканы считались пороховой бочкой Европы. Во время дипломатического визита в Сараево Гаврило Принцип убил Австро-Венгерского правителя Франца Фердинанда. Так как Гаврило принадлежал к террористической группировке, которую напрямую поддерживала Сербские власти, то к ним появились претензии. Австро-Венгрия выставила 10 ультиматумов, среди которых был следующий пункт: "Допустить Австро-Венгерских полицейских к расследованию на территории Сербии". Все пункты были приняты, кроме этого. И по цепочки началась война: у Сербии в союзниках была Россия, у России~--- Франция, и тд. Аналогично у Австро-Венгрии была Германия и ее союзники.
	\subsection{Война.}
	$1$ августа $1914$ года считается началом войны. Сербия объявила войну Австро-Венгрии, Германия~--- России и Франции. Россия первая начала наступления на Германскую Империю, входя в Восточную Пруссию. Сначала армия успешно продвигалась в глубь, но позже движение замедлилось и остановилось. В это же время Германия наступала на Францию, и если бы Россия не наступила, то возможно Франция была бы сметена. Бельгия не хотела пропускать немцев через свои земли, так как тогда на нее обрушился бы удар со стороны Франции, поэтому она стала сопротивляться и невольно начала участвовать в войне на стороне Антанты. \\
	$1915$ был провальным для Российской Империи. Он назывался годом отступления. В какой-то момент было потеряно $\frac{2}{3}$ Польского государства. Также была потеряна часть Прибалтики и часть Украины. У Османской Империи на Кавказском фронте тоже шли дела плоха. Турки устроили геноцид Армян. Падение дисциплины. \\
	В марте $1918$ с Германией был подписан Брест-Литовский мирный договор. Условия были унизительны: признание независимости Прибалтики и огромная контрибуция.
\end{document}
