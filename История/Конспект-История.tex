\documentclass[12pt]{article}

\input{C:/Users/ilden/Documents/School/TeX памятка/header.tex}

\begin{document}
	\tableofcontents
	\setcounter{tocdepth}{3}
	\newpage
	\section{Первая мировая война.}
	\subsection{До войны.}
	$40$ лет до Первой мировой войны в Европе войн не было (только на Балканах). Одна из причин начала первой мировой войны~--- колониальный вопрос. Также Германскую Империю волновало начало перевооружения в Российской Империи. Основные участники Антанты, противостоящей Германской Империи, Италии, Османской Империи и их союзникам: Франция, Россия, Великобритания. Позже подключились Соединенные Штаты Америки. Балканы считались пороховой бочкой Европы. Во время дипломатического визита в Сараево Гаврило Принцип убил Австро-Венгерского правителя Франца Фердинанда. Так как Гаврило принадлежал к террористической группировке, которую напрямую поддерживала Сербские власти, то к ним появились претензии. Австро-Венгрия выставила 10 ультиматумов, среди которых был следующий пункт: <<Допустить Австро-Венгерских полицейских к расследованию на территории Сербии>>. Все пункты были приняты, кроме этого. И по цепочки началась война: у Сербии в союзниках была Россия, у России~--- Франция, и тд. Аналогично у Австро-Венгрии была Германия и ее союзники.
	\subsection{Война.}
	$1$ августа $1914$ года считается началом войны. Сербия объявила войну Австро-Венгрии, Германия~--- России и Франции. Россия первая начала наступления на Германскую Империю, входя в Восточную Пруссию. Сначала армия успешно продвигалась в глубь, но позже движение замедлилось и остановилось. В это же время Германия наступала на Францию, и если бы Россия не наступила, то возможно Франция была бы сметена. Бельгия не хотела пропускать немцев через свои земли, так как тогда на нее обрушился бы удар со стороны Франции, поэтому она стала сопротивляться и невольно начала участвовать в войне на стороне Антанты. \\
	$1915$ был провальным для Российской Империи. Он назывался годом отступления. В какой-то момент было потеряно $\frac{2}{3}$ Польского государства. Также была потеряна часть Прибалтики и часть Украины. У Османской Империи на Кавказском фронте тоже шли дела плоха. Турки устроили геноцид Армян. Падение дисциплины. \\
	В марте $1918$ с Германией был подписан Брест-Литовский мирный договор. Условия были унизительны: признание независимости Прибалтики и огромная контрибуция.
	\section{Свержения монархии в России.}
	\subsection{Причины.}
	\begin{itemize}
		\item Продовольственный вопрос.
		\item Недо-конституционная монархия.
		\item Усталость от войны.
		\item Рабочий вопрос.
		\item Земельный вопрос.
		\item Национальный и религиозный вопрос.
	\end{itemize}
	\subsection{Революция.}
	\subsubsection{Волнения в Петрограде.}
	\paragraph{Хлебные очереди.} Собирались огромные очереди из людей, стоящих за хлебом, и они перерастали в митинги.
	\paragraph{Забастовки на заводах.} К $23$ февраля $1917$ года бастовали $130$ тыс. человек.
	\paragraph{Огромное шествие протестующих.} $23$ февраля $1917$ года был Международным днем солидарности трудящихся женщин. На Невском вышли колонны работников требующих хлеба. Спустя некоторое время появились и политические лозунги. На следующий день стачка стала всеобщей и охватывала $200$ тыс. человек. $24$ февраля в столице произошли первые столкновения с полицией. Царь не за долго до всех событий уехал из Царского Села в Могилев (там был генштаб, в котором готовилось наступление на Германию). $25$ февраля вступают военные и открывают огонь, убив и ранив около $150$ человек. Некоторые военные все равно отказались стрелять по рабочим. Права военных были сильно принижены в Петрограде. Победа восстания.
	\subsubsection{Совет рабочих депутатов.}
	$27$ марта меньшевистская фракция РСДРП и рабочая группа Центрального ВПК организовали выборы в \textit{Совет рабочих депутатов}. Подавляющее большинство Петросовета составили эсеры и меньшевики. Представительство большевиков было незначительным. Среди $25$ членов Исполкома Петросовета было $4$ большевика. Это объяснялось тем, что их пораженчество и крайний радикализм были еще не популярны среди рабочих столицы и солдат гарнизона. Представителем Петросовета был избран лидер меньшевиков в государственной думе Н.С. Чхеидзе. Его заместителями стали лидер трудовиков эсеров А.Ф. Керенский и меньшевиков М.И. Скобелев. С первого дня существования совет стал реальной властью в Петрограде.
	\subsubsection{Роспуск Государственной Думы.}
	$27$ февраля Н.Д. Голицын в соответствии с повелением царя объявил о роспуске Думы. Депутаты Думы подчинились роспуску, но тут же создали временный комитет Государственной Думы во главе с М.В. Родзянко.
	\subsubsection{Отречение Николая II.}
	Сам Николай II $28$ февраля выехал в Царское Село. В пути поступило сообщение о том, что станции Тосно и Любань захвачены бунтовщиками. Рузский с трудом добился согласия Николая II пойти на уступки. Родзянко сказал, что уступка царя уже опоздала и необходимо отречение. Все говорят Николаю II, что отрекаться от престола надо.
	\subsection{После отречения.}
	\subsubsection{Временное правительство.}
	Новое правительство было буржуазно-либеральным.
	\begin{itemize}
		\item Премьер-министр Г.Е. Львов и министр финансов М.И. Терещенко~--- беспартийные.
		\item Военно-морской министр А.И. Гучков и государственный контролер И.В. Годнев~--- октябристы.
		\item Министр торговли и промышленности А.И. Коновалов~--- прогрессист.
		\item Министр иностранных дел П.Н. Милюков, министры земледелия А.И. Шингарев, путей сообщения Н.В. Некрасов, просвещения Финляндии Ф.И. Родичев~--- кадеты.
	\end{itemize}
	\subsubsection{Политика временного правительства.}
	Временное правительство должно было действовать до созыва Учредительного собрания. Оно упразднило Корпус жандармов и полицию, заменив их народной милицией. Была отменена смертная казнь, объявлена амнистия. Также была отменена цензура (кроме военной). Граждане получили право свободно проводить собрания и шествия, создавать любые союзы и общества. Права суда присяжных расширились, отменен имущественный ценз для присяжных.
	\subsubsection{Национальный вопрос.}
	Выполняли требования многих национальных меньшинств.
	\subsubsection{Революция и церковь.}
	Принятие закона <<О свободе и совести>>. Попробовали отделить церковь от власти.
	\subsubsection{Продовольственный вопрос.}
	Оказался нерешенным. Хотя решать его пыталось еще царское правительство. Правительство удваивало закупочные цены и надеялось, что зерно будут сдавать добровольно; насильно забирать его они не могли, так как это противоречит либеральным взглядам и у них не было надежной опоры (армии) для этого действия.
	\subsubsection{Аграрный вопрос.}
	Землю хотели отнимать у помещиков и делить между крестьянами, так как министром земледелия стал эсер. Правительство решилось только передать сделки с землей на усмотрение местных земельных комитетов. Но начались аграрные беспорядки, так как крестьяне требовали полного запрета продажи земли. Они начали нападать на имения. Чаще всего захватывали господскую землю, скот, инвентарь, но были и случаи разгрома усадеб. Правительство пыталось пресечь захват земли силой, но армия чаще всего переходила на сторону крестьян.
	\subsubsection{Борьба за армию.}
	$1$ марта $1917$ года Петросовет издал \textbf{Приказ №1}. Следствием приказа стало падение дисциплины в армии.
	\subsubsection{Вопрос о войне и мире.}
	Буржуазные круги настаивали на продолжении войны до полной победы, а совет хотел заключить мир.
	\section{Революционное движение.}
	\subsection{Большевики.}
	К началу $1917$ года большевистскими организациями в России на практике руководило Русское бюро ЦК во главе с А.Г. Шляпниковым. Уже $4$ марта $1917$ года бюро приняло резолюцию: <<Теперешнее Временное правительство, по существу, контрреволюционное, так как состоит из представителей крупной буржуазии и дворянства, а потому с ним не может быть никаких соглашений>>. Бюро выступало за превращение империалистической войны в гражданскую и отвергало революционное оборончество.
	\subsection{Апрельские тезисы.}
	$7$ апреля <<Правда>> опубликовала тезисы доклада~--- <<Апрельские тезисы>>. Политические противники Ленина резко отреагировали на <<Апрельские тезисы>> и обвинили их автора в разжигании гражданской войны.
	\subsection{Первый съезд Советов.}
	$3$ июня $1917$ в Петрограде~--- Первый съезд Советов рабочих и солдатских депутатов. $285$ эсеров, $248$ меньшевиков, $105$ большевиков. 16 июня избран ЦИК. ЦИК: $4$ эсера, $4$ меньшевика, $1$ большевик (Каменев). Председатель ЦИКа --- Чхеидзе. Решили сотрудничать со временным правительством. Большевики против этого решения. 10 июня, накануне съезда большевики решили провести антивоенную демонстрацию.
	\subsection{Наступление русской армии.}
	Инициатива Временного правительства (военного министра Керенского) с целью поднять боевой дух армии, укрепить свою власть и выполнить союзнические обязательства перед Антантой. $19$ июня известия о наступлении достигли Петрограда и резко изменили настроения. На смену лозунгам <<Долой войну>> пришли призывы <<Война до победного конца!>> и <<Доверие Временному правительству!>>.
	Серьезных успехов в наступлении добилась только $8$ армия, так как Немцы пришли на помощь к Австрийцам, и армии потерпели поражение, было бегство.
	\subsection{Июльский кризис.}
	$2$ июля одновременно с появлением известий о провале наступления в Петрограде разразился политический кризис. В апреле $1917$ года Всеукраинский съезд в Киеве избрал \textbf{\textit{Центральную Раду}}. Рада выдвинула лозунг автономии Украины. $28$ июня $1917$ года делегация Временного правительства в составе А.Ф. Керенского, М.И. Терещенко и И.Г. Церетели пообещала Центральной Раде предоставить Украине автономность.
	$3$ июля в Петрограде начались демонстрации. Их начали солдаты большевизированного $1$-го пулеметного полка, насчитывавшего $11$ тысяч человек.
	Лидеры Военной организации ПК РСДРП(б) заявили, что у них <<достаточно пулеметов для свержения временного правительства>>. Был даже сформирован Военно-революционный комитет (ВРК)~--- руководящий орган восстания.
	Ленин $3$ июля отдыхал за городом. $4$ июля он вернулся в Петроград и выступил перед демонстрантами, собравшимися у резиденции большевиков~--- бывшего особняка балерины М. Кшесинской, призвав к выдержке, стойкости и бдительности, подчеркивая, к разочарованию рабочих, солдат и матросов, мирный характер демонстрации.
	$5$ июля $1917$ года ЦК РСДРП(б) объявил о прекращении демонстраций, поскольку <<выступление $3$-$4$ июля достигло своей цели>>. Временное правительство объявило Петроград на военном положении.
	\subsection{Корниловский мятеж. Август $1917$ года.}
	После Корниловского кризиса формируется II коалиционное правительство. Представителем правительства становится Керенский. Правительство в основном состоит из эсеров и меньшевиков, но полностью отказаться от буржуазных партий нельзя, так как тогда временное правительство останется без денег. Кадеты готовы обеспечить правительству кредиты, но они хотят порядка, в первую очередь в армии, и в этот момент появляется идея назначить Корнилова главнокомандующим. Корнилов~--- сторонник жестокой дисциплины в армии, и поэтому требует разрешение восстановить смертную казнь в армии. Также Корнилов ограничивает полномочия солдатских комитетов; планирует отправить войска в Петроград для восстановления и поддержания порядка. У Керенского и Петросовета возникают опасения, что Корнилов планирует уничтожить власть Советов. Это дает возможность Керенскому обвинить Корнилова в попытке захвата власти, а себя наделить чрезвычайными полномочиями. $25$ августа Корнилов направляет в Петроград корпус генерала Крымова. $27$ августа Корнилова отстранили от должности главнокомандующего, а Керенский обращается за помощью к социалистам, в том числе выпускает из тюрем большевиков, которые тут же возглавляют оборонное движение. Корпус Крымова остановили на подходе к Петрограду, сам Крымов застрелился, и, если считать это мятежом, мятеж провалился. В результате в Петрограде полностью поменялась расстановка политических сил: кадеты дискредитированны связью с Корниловым, и теперь их не хотят видеть в правительстве, с другой стороны большевики оказываются спасителями революции, что приводит к резкому росту и популярности, и численности партии. $1$ сентября Керенский объявляет Россию республикой. Распавшееся II коалиционное правительство к середине сентября сменяется III коалиционным правительством. Меняется состав Петросовета~--- теперь в нем преобладают большевики, и председателем становится Л. Троцкий.
	\subsection{Развитие революции осенью $1917$ года.}
	Демократическое совещание открылось в Петрограде $14$ сентября $1917$ года. $1582$ делегата от Советов, городских дум, земств, армейских комитетов, профсоюзов, кооперации. $55$ народных социалистов, $461$ правый с-р, $116$ меньшевиков, $71$ левый с-р, $134$ большевика, $56$ интернационалистов.
	\subsubsection{Демократическое совещание.}
	19 сентября Демократическое совещание одобрило принцип коалиции с буржуазией. После этого были приняты поправки, запрещавшие включать в правительство кадетов.  20 сентября решено создать из членов совещания (по 15 процентов от каждой фракции) постоянно действующий Всероссийский демократический совет (предпарламент). Предпарламент должен был окончательно решить вопрос о власти. Накануне открытия Демократического совещания Ленин окончательно отказался от тактики, намеченной в статье <<О компромиссах>>. 12-14 сентября он направил в большевистский ЦК письма <<Большевики должны взять власть>> и <<Марксизм и восстание>>. Ленин отмечал усиление рабочих, крестьянских и солдатских волнений, а также рост влияния большевиков: в сентябре они добились преобладания в Петросовете и Моссовете, а Троцкий 9 сентября был избран председателем Петросовета. Кроме того, появились известия о росте революционных настроений в Германии, что позволяло надеяться на международную поддержку. Большевистский ЦК не согласился с Лениным и отказался распространить его письма среди членов партии. 21 сентября состоялось заседание ЦК. Предложения Ленина о немедленной подготовке восстания не рассматривались. Все надежды связывались с предстоящим съездом Советов. Каменев считал, что съезд Советов должен сформировать однородное социалистическое правительство. Троцкий рассчитывал, что съезд Советов передаст власть именно большевикам, и призывал к уходу с Демократического совещания и бойкоту Предпарламента (9 голосов против 8: бойкот Предпарламента).
	\subsubsection{Подготовка восстания.}
	$10$ октября состоялось заседание ЦК большевиков. В нем принял участие Ленин. вернувшийся в начале октября в Петроград. Он настаивал на организации восстания с опорой на съезд Советов Северной области, намеченный на $11$ октября. Возражали против немедленного восстания только Зиновьев и Каменев. Заседание поддержало Ленина $10$-ю голосами против двух. Расширенное заседание ЦК $16$ октября подтвердило решение о немедленной подготовке восстания.
	\subsubsection{Съезд Советов Северной области.}
	$11$-$16$ октября состоялся \textbf{\textit{Съезд Советов Северной области}} (севера и северо-запада России). При его созыве были нарушены нормы представительства: от тех Советов, где преобладали большевики, было приглашено непропорционально много делегатов.
	\subsubsection{Выступление Каменева и Зиновьева.}
	$11$ октября Зиновьев и Каменев обратились с письмом к членам большевистской партии. Ленин крайне остро отреагировал на это и потребовал исключения их из партии.
	\subsubsection{Подготовка восстания.}
	На совещании $16$ октября был создан партийный военно-революционный центр. $21$ октября ВРК направил своих комиссаров в части столичного гарнизона. $22$ октября Троцкий потребовал от гарнизона не выполнять никаких приказов, неподписанных ВРК.
	\subsubsection{Начало восстания.}
	И все же Керенский решил закрыть большевистские газеты <<Рабочий путь>> и <<Солдат>> и начать уголовное дело против ВРК. Керенский после заседания правительства отправился в Предпарламент, где произнес речь, обвиняя большевиков в измене. Предпарламент отказал Керенскому в дополнительных полномочиях на осуществление репрессий.
	\subsubsection{Восстание в Петрограде.}
	Днем $24$ октября отряды солдат и красной гвардии блокировали юнкерские училища, перекрыли железные дороги, овладели Финляндским вокзалом. Поздно вечером Ленин, не выдержав нервного напряжения, пришел с конспиративной квартиры в Смольный. К этому времени ВРК, ощутив беспомощность правительства, окончательно перешел к наступлению. Ночью $25$ октября восставшие заняли Николаевский и Балтийские вокзалы, Центральную электростанцию. Рано утром~--- Госбанк, телефонную станцию, Варшавский вокзал.
	\subsubsection{Ультиматум правительству.}
	В $18$:$00$ ВРК передал Временному правительству ультиматум о сдаче, оставленный без ответа. После того, как в $19$ часов восставшие заключили здание Главного штаба, правительство обратилось к населению России с последним воззванием. В $21$:$45$ Аврора делает холостой залп.
	\section{Советская власть.}
	\subsection{Становление советской власти.}
	Как только большевики пришли к власти их пытались свергнуть, так как было много недовольных тем, что они пришли к власти.
	\subsection{Подписание Брестского мира.}
	Немецкое наступление позволяет Центральному Большевистскому Комитету переосмыслить все, что они сделали, и принять предложение Ленина пойти на Немецкие условия. $21$ февраля $1918$ года был принят декрет <<Социалистическое отечество в опасности!>>, $22$ февраля он был опубликован.
	\subsection{Первые месяцы советской власти.}
	\begin{note}
		Для любой экономики государства необходима банковская система для кредитования.
	\end{note}
	\begin{note}
		В Российской империи банки зачастую вступали в союз с промышленными компаниями, банковская система была развита.
	\end{note}
	\subsubsection{Национализация.}
	\begin{definition}[Красногвардейская атака на капитал]
		\textbf{21 января 1918 года} Советское правительство аннулирует долговые обязательства царской власти. Происходит национализация промышленности. \textbf{До июня} было национализировано $521$ предприятие. Установленна монополия внешней торговли. Правительство переходит к национализации целых отраслей промышленности.
	\end{definition}
	\begin{note}
		Национализация не вполне соответствовала идеям Маркса, поскольку собственность государства не равна собственности общества.
	\end{note}
	Национализация привела к падению трудовой дисциплин, прогулам, расхищениям имущества, и в конечном итога~--- падению эффективности производства.
	\begin{definition}[Трудовая повинность]
		Изначально буржуев принуждали к грязной работе, например уборке улиц. Позже трудовая повинность распространилась и на рабочих.
	\end{definition}
	\subsubsection{Продовольственный вопрос.}
	Уже в первые недели власти большевики бросают отряды на поиски зерна. \textbf{\textit{В январе 1918 года}} продовольственный паек составляли \textbf{100--115 грамм} в день. В других губерниях, которые не специализировались на производстве зерна паек был еще меньше. Люди стали требовать разрешить свободную продажу хлеба. Деньги обесценивались и крестьяне меняли хлеб на любые полезные вещи, возвращался бартер.
	\subsubsection{Аграрный вопрос.}
	Крестьянам предлагали отнять помещичью землю, хотя на самом деле у помещиков было лишь \textbf{20\%} земли. После принятия \textbf{\textit{Декрета о земле}} крестьяне начали захватывать земли помещиков и крупных землевладельцев.
	\section{Гражданская война.}
\end{document}
