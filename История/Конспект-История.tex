\documentclass[12pt]{article}

\input{C:/Users/ilden/Documents/School/TeX памятка/header.tex}

\begin{document}
	\tableofcontents
	\setcounter{tocdepth}{3}
	\newpage
	\section{Первая мировая война.}
	\subsection{До войны.}
	$40$ лет до Первой мировой войны в Европе войн не было (только на Балканах). Одна из причин начала первой мировой войны~--- колониальный вопрос. Также Германскую Империю волновало начало перевооружения в Российской Империи. Основные участники Антанты, противостоящей Германской Империи, Италии, Османской Империи и их союзникам: Франция, Россия, Великобритания. Позже подключились Соединенные Штаты Америки. Балканы считались пороховой бочкой Европы. Во время дипломатического визита в Сараево Гаврило Принцип убил Австро-Венгерского правителя Франца Фердинанда. Так как Гаврило принадлежал к террористической группировке, которую напрямую поддерживала Сербские власти, то к ним появились претензии. Австро-Венгрия выставила 10 ультиматумов, среди которых был следующий пункт: "Допустить Австро-Венгерских полицейских к расследованию на территории Сербии". Все пункты были приняты, кроме этого. И по цепочки началась война: у Сербии в союзниках была Россия, у России~--- Франция, и тд. Аналогично у Австро-Венгрии была Германия и ее союзники.
	\subsection{Война.}
	$1$ августа $1914$ года считается началом войны. Сербия объявила войну Австро-Венгрии, Германия~--- России и Франции. Россия первая начала наступления на Германскую Империю, входя в Восточную Пруссию. Сначала армия успешно продвигалась в глубь, но позже движение замедлилось и остановилось. В это же время Германия наступала на Францию, и если бы Россия не наступила, то возможно Франция была бы сметена. Бельгия не хотела пропускать немцев через свои земли, так как тогда на нее обрушился бы удар со стороны Франции, поэтому она стала сопротивляться и невольно начала участвовать в войне на стороне Антанты. \\
	$1915$ был провальным для Российской Империи. Он назывался годом отступления. В какой-то момент было потеряно $\frac{2}{3}$ Польского государства. Также была потеряна часть Прибалтики и часть Украины. У Османской Империи на Кавказском фронте тоже шли дела плоха. Турки устроили геноцид Армян. Падение дисциплины. \\
	В марте $1918$ с Германией был подписан Брест-Литовский мирный договор. Условия были унизительны: признание независимости Прибалтики и огромная контрибуция.
	\section{Свержения монархии в России.}
	\subsection{Причины.}
	\begin{itemize}
		\item Продовольственный вопрос.
		\item Недо-конституционная монархия.
		\item Усталость от войны.
		\item Рабочий вопрос.
		\item Земельный вопрос.
		\item Национальный и религиозный вопрос.
	\end{itemize}
	\subsection{Революция.}
	\subsubsection{Волнения в Петрограде.}
	\paragraph{Хлебные очереди.} Собирались огромные очереди из людей, стоящих за хлебом, и они перерастали в митинги.
	\paragraph{Забастовки на заводах.} К $23$ февраля $1917$ года бастовали $130$ тыс. человек.
	\paragraph{Огромное шествие протестующих.} $23$ февраля $1917$ года был Международным днем солидарности трудящихся женщин. На Невском вышли колонны работников требующих хлеба. Спустя некоторое время появились и политические лозунги. На следующий день стачка стала всеобщей и охватывала $200$ тыс. человек. $24$ февраля в столице произошли первые столкновения с полицией. Царь не за долго до всех событий уехал из Царского Села в Могилев (там был генштаб, в котором готовилось наступление на Германию). $25$ февраля вступают военные и открывают огонь, убив и ранив около $150$ человек. Некоторые военные все равно отказались стрелять по рабочим. Права военных были сильно принижены в Петрограде. Победа восстания.
	\subsubsection{Совет рабочих депутатов.}
	$27$ марта меньшевистская фракция РСДРП и рабочая группа Центрального ВПК организовали выборы в \textit{Совет рабочих депутатов}. Подавляющее большинство Петросовета составили эсеры и меньшевики. Представительство большевиков было незначительным. Среди $25$ членов Исполкома Петросовета было $4$ большевика. Это объяснялось тем, что их пораженчество и крайний радикализм были еще не популярны среди рабочих столицы и солдат гарнизона. Представителем Петросовета был избран лидер меньшевиков в государственной думе Н.С. Чхеидзе. Его заместителями стали лидер трудовиков эсеров А.Ф. Керенский и меньшевиков М.И. Скобелев. С первого дня существования совет стал реальной властью в Петрограде.
	\subsubsection{Роспуск Государственной Думы.}
	$27$ февраля Н.Д. Голицын в соответствии с повелением царя объявил о роспуске Думы. Депутаты Думы подчинились роспуску, но тут же создали временный комитет Государственной Думы во главе с М.В. Родзянко.
	\subsubsection{Отречение Николая II.}
	Сам Николай II $28$ февраля выехал в Царское Село. В пути поступило сообщение о том, что станции Тосно и Любань захвачены бунтовщиками. Рузский с трудом добился согласия Николая II пойти на уступки. Родзянко сказал, что уступка царя уже опоздала и необходимо отречение. Все говорят Николаю II, что отрекаться от престола надо.
	\subsection{После отречения.}
	\subsubsection{Временное правительство.}
	Новое правительство было буржуазно-либеральным.
	\begin{itemize}
		\item Премьер-министр Г.Е. Львов и министр финансов М.И. Терещенко~--- беспартийные.
		\item Военно-морской министр А.И. Гучков и государственный контролер И.В. Годнев~--- октябристы.
		\item Министр торговли и промышленности А.И. Коновалов~--- прогрессист.
		\item Министр иностранных дел П.Н. Милюков, министры земледелия А.И. Шингарев, путей сообщения Н.В. Некрасов, просвещения Финляндии Ф.И. Родичев~--- кадеты.
	\end{itemize}
	\subsubsection{Политика временного правительства.}
	Временное правительство должно было действовать до созыва Учредительного собрания. Оно упразднило Корпус жандармов и полицию, заменив их народной милицией. Была отменена смертная казнь, объявлена амнистия. Также была отменена цензура (кроме военной). Граждане получили право свободно проводить собрания и шествия, создавать любые союзы и общества. Права суда присяжных расширились, отменен имущественный ценз для присяжных.
	\subsubsection{Национальный вопрос.}
	Выполняли требования многих национальных меньшинств.
	\subsubsection{Революция и церковь.}
	Принятие закона "О свободе и совести". Попробовали отделить церковь от власти.
	\subsubsection{Продовольственный вопрос.}
	Оказался нерешенным. Хотя решать его пыталось еще царское правительство. Правительство удваивало закупочные цены и надеялось, что зерно будут сдавать добровольно; насильно забирать его они не могли, так как это противоречит либеральным взглядам и у них не было надежной опоры (армии) для этого действия.
	\subsubsection{Аграрный вопрос.}
	Землю хотели отнимать у помещиков и делить между крестьянами, так как министром земледелия стал эсер. Правительство решилось только передать сделки с землей на усмотрение местных земельных комитетов. Но начались аграрные беспорядки, так как крестьяне требовали полного запрета продажи земли. Они начали нападать на имения. Чаще всего захватывали господскую землю, скот, инвентарь, но были и случаи разгрома усадеб. Правительство пыталось пресечь захват земли силой, но армия чаще всего переходила на сторону крестьян.
	\subsubsection{Борьба за армию.}
	$1$ марта $1917$ года Петросовет издал \textbf{Приказ №1}. Следствием приказа стало падение дисциплины в армии.
	\subsubsection{Вопрос о войне и мире.}
	Буржуазные круги настаивали на продолжении войны до полной победы, а совет хотел заключить мир.
	\subsection{Большевики.}
	К началу $1917$ года большевистскими организациями в России на практике руководило Русское бюро ЦК во главе с А.Г. Шляпниковым. Уже $4$ марта $1917$ года бюро приняло резолюцию: "Теперешнее Временное правительство, по существу, контрреволюционное, так как состоит из представителей крупной буржуазии и дворянства, а потому с ним не может быть никаких соглашений". Бюро выступало за превращение империалистической войны в гражданскую и отвергало революционное оборончество.
	\subsubsection{Апрельские тезисы.}
	$7$ апреля "Правда" опубликовала тезисы доклада~--- "Апрельские тезисы". Политические противники Ленина резко отреагировали на "Апрельские тезисы" и обвинили их автора в разжигании гражданской войны.
\end{document}
