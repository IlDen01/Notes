\documentclass{article}

\input{C:/Users/ilden/Documents/School/TeX памятка/header.tex}

\begin{document}
	\tableofcontents
	\setcounter{tocdepth}{3}
	\newpage
	\section{Органическая химия.}
	\subsection{Теория строения органических веществ.}
	\begin{definition}
		Основные положения:
		\begin{enumerate}
			\item Атомы в молекулах соединены друг с другом в соответствии с их валентностью. \\
			$C (IV); H (I); O (II); P (V); N (III); Hal (I)$
			\item Атомы молекул органических веществ соединяются между собой в определенной последовательности, что обусловливает химическое строение молекулы.
			\item Свойства органических соединений зависят не только от числа и природы, входящих в их состав, атомов, но и от химического строения.
			\item В молекулах существует взаимное влияние как связанных, так и непосредственно не связанных друг с другом атомов.
		\end{enumerate}
	\end{definition}
\end{document}
