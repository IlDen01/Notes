\documentclass[12pt]{article}

\input{C:/Users/ilden/Documents/School/TeX памятка/header.tex}

\begin{document}
	\tableofcontents
	\setcounter{tocdepth}{3}
	\newpage
	\section{Органическая химия.}
	\subsection{Теория строения органических веществ.}
	\begin{definition}
		Основные положения:
		\begin{enumerate}
			\item Атомы в молекулах соединены друг с другом в соответствии с их валентностью. \\
			$C (IV); H (I); O (II); P (V); N (III); Hal (I)$
			\item Атомы молекул органических веществ соединяются между собой в определенной последовательности, что обусловливает химическое строение молекулы.
			\item Свойства органических соединений зависят не только от числа и природы, входящих в их состав, атомов, но и от химического строения.
			\item В молекулах существует взаимное влияние как связанных, так и непосредственно не связанных друг с другом атомов.
			\item По химическому строению вещества можно предсказать его свойства, а по свойствам~--- строение.
		\end{enumerate}
	\end{definition}
	\subsection{Электронное состояние атома углерода в органических соединениях.}
	\subsection{Изомерия.}
	\begin{definition}
		Вещества, имеющие одинаковый качественный и количественный состав, но разное строение и, следовательно, свойства называются изомерами.
	\end{definition}
	\subsubsection{Изомерия по положению заместителя.}
	\begin{note}
		Если функциональных групп несколько, то определяется их старшинство, и младшая группа становится приставкой.
	\end{note}
	\subsubsection{Таблица функциональных групп.}
	\begin{xltabular}{\textwidth}{|X|X|X|X|X|}
		\hline
		Формула функциональной группы. & Название функциональной группы. & Префикс. & Суффикс. & Класс соединения. \\
		\hline
		$-COOH$ & Карбоксильная & --- & --овая кислота & Карбоновые кислоты \\
		\hline
		$-SO_3H$ & Сульфогруппа & сульфо-- & --сульфокислота & Сульфокислоты \\
		\hline
		$-COH$/$-CO-$ & Карбонильная & формил-- / оксо-- & --аль / --он & Альдегиды / Кетомы \\
		\hline
		$OH$ & Гидрокси-группа & гидроксо-- & --ол & Спирты \\
		\hline
		$SH$ & Тиогруппа & тио-- & --тиол & Тиоспирты / Меркаптаны \\
		\hline
		$NH_2$ & Аминогруппа & амино-- & --амин & Амины \\
		\hline
	\end{xltabular}
	\subsection{Алканы.}
	Общая формула: $C_nH_{2n + 2}, n \geqslant 1$.
\end{document}
