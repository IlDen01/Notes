\documentclass[12pt]{article}

\input{C:/Users/ilden/Documents/School/TeX памятка/header.tex}
\newcommand{\orb}{\text{orb}}
\newcommand{\ord}{\text{ord}}
\newcommand{\cycl}{\text{cycl}}

\begin{document}
	\tableofcontents
	\setcounter{tocdepth}{3}
	\newpage
	\section{Теория групп.}
	\begin{definition}
		\label{group-def}
		Группа~--- множество с одной операцией $*$: $G \times G \rightarrow G$ со следующими свойствами:
		\begin{enumerate}
			\item $a * (b * c) = (a * b) * c$
			\item $\exists e$: $a * e = e * a = a$
			\item $\forall a$ $\exists a^{-1}$: $a * a^{-1} = a^{-1} * a = e$
		\end{enumerate}
	\end{definition}
	\begin{example}
		\begin{itemize}
			\item $(\mathbb{Z}; +)$
			\item $(\mathbb{Q}; +)$
			\item $(\mathbb{R}; +)$
			\item $(\mathbb{C}; +)$
			\item $(V; +)$
			\item $(\mathbb{R}_+; \cdot)$
			\item $(\mathbb{R} \setminus \{ 0 \}; \cdot)$
			\item $(\mathbb{Z}_n; +)$
		\end{itemize}
	\end{example}
	\begin{definition}
		Пусть $G$~--- группа, $H \subset G$. Говорим, что $H$ является подгруппой (пишем $H < G$), если $H$ является группой относительно операции в $G$. Чтобы проверить, что $H$ является подгруппой, необходимо убедиться, что произведение двух элементов из $H$ принадлежит $H$, и элементы, обратные к $H$, тоже лежат в $H$.
	\end{definition}
	\begin{theorem}[Кэли]
		Любая группа $G$ является подгруппой в группе подстановок, а именно $S_G$.
	\end{theorem}
	\begin{definition}
		Абелева группа~--- \hyperref[group-def]{группа} с коммутативностью.
	\end{definition}
	\begin{definition}
		Говорят, что группа $G$ порождается элементами $\{ x_i \}$, если любой элемент из $G$ можно представить как произведение нескольких $x_i$ и обратных к ним. Группа называется циклической, если она порождена одним элементом.
	\end{definition}
	\begin{theorem}
		Конечная циклическая группа изоморфна $(\mathbb{Z}_n, +)$.
	\end{theorem}
	\begin{definition}
		Пусть $G$~--- группа, $H < G$. Введем два отношения эквивалентности на $G$: $x \sim_1 y$ если $xy^{-1} \in H$, $x \sim_2 y$ если $x^{-1}y \in H$.
	\end{definition}
	\begin{statement}
		$\sim_1$ и $\sim_2$ совпадают в абелевой группе.
	\end{statement}
	\begin{note}
		Классы эквивалентности по отношению $\sim_1$ называются левыми смежными классами; класс элемента $x$ обозначается $xH$. Классы эквивалентности по $\sim_2$~--- правые смежные классы, обозначаются $Hx$.
	\end{note}
	\begin{definition}
		Пусть теперь группа $G$ конечна, $H < G$. Количество классов эквивалентности $\sim_1$ называется индексом $G$ по $H$ и обозначается $[G : H]$.
	\end{definition}
	\begin{theorem}[Лагранжа]
		$|G| = |H| \cdot [G : H]$.
	\end{theorem}
	\begin{definition}
		Пусть $x \in G$. Порядком элемента $x$ называется наименьшее натуральное число $n$, такое что $x^n = e$, где $e$~--- нейтральный элемент. Обозначение: $ord(x)$. Если такого $n$ не существует, то пишем $ord(x) = + \infty$.
	\end{definition}
	\begin{definition}
		Пусть $X, Y \subset G$~--- подмножества группы. Их произведением будем называть множество $\{ xy | x \in X, y \in Y \}$.
	\end{definition}
	\begin{definition}
		Полная линейна группа $GL(n, F)$~--- это множество всех квадратных матриц размера $n \times n$ с элементами из поля $F$, которые являются обратимыми ($\det \not= 0$), вместе с операцией матричного умножения.
	\end{definition}
	\begin{definition}
		Специальная линейная группа $SL(n, F)$~--- это подгруппа полной линейной группы, состоящая из всех матриц с определителем, равным $1$. То есть это множество всех матриц $A$ размера $n \times n$ над полем $F$, таких что $\det(A) = 1$.
	\end{definition}
	\begin{definition}
		Специальная ортогональная группа $SO(n, F)$~--- группа из ортогональных матриц. Матрица $A$ называется ортогональной, если $A^T \times A = A \times A^T = E$.
	\end{definition}
	\begin{statement}
		Пусть $X$~--- произвольное множество. Тогда множество всех биекций $f: X \rightarrow X$ образуют группу относительно композиции. Эту группу обозначают $S_X$. Если $X = \{ 1, 2, \dots, n \}$, то $S_X$ обозначают $S_n$~--- группа подстановок.
	\end{statement}
	\subsection{Таблица Кэли.}
	\begin{definition}
		Пусть $G$~--- конечная группа порядка $n$. Её таблицей Кэли (таблицей умножения) будем называть таблицу $(n + 1) \times (n + 1)$ (левый столбец и левая строка считаются нулевыми и служат лишь для нумерации). В нулевом столбце и в нулевой строке стоят все элементы группы в одном и том же порядке. На пересечении строки и столбца этой таблицы будем ставить произведение соответствующих элементов в нулевом столбце и в нулевой строке (слева пишется элемент, задающий строку, справа — столбец).
	\end{definition}
	\begin{definition}
		Напомним, что перестановкой мы будем называть биекцию $f: \{ 1, 2, \dots, n \} \rightarrow \{ 1, 2, \dots, n \}$. Умножение перестановок — композиция биекций. Перестановки записываются в две строчки: в первой~--- числа от $1$ до $n$ (как правило, в порядке возрастания, но не обязательно), а во второй строчке под числом $k$ стоит число $f(k)$.
	\end{definition}
	\subsection{Группы.}
	\begin{theorem}
		$G$~--- группа; $H < G$. Равносильно:
		\begin{enumerate}
			\item $\forall x$ $xH = Hx$
			\item $\forall x$ $xH \subset Hx$
			\item $\forall x$ $xH \supset Hx$
			\item $\forall x$ $xHx^{-1} = H$
			\item $\forall x$ $xHx^{-1} \subset H$
			\item $\forall x$ $xHx^{-1} \supset H$
			\item $\forall x \in G$ $\forall h \in H$ $x^{-1}hx \in H$
		\end{enumerate}
	\end{theorem}
	\begin{definition}
		Такая $H$ называется нормальной подгруппой. Обозначается $H \vartriangleleft G$.
	\end{definition}
	\begin{definition}
		$xhx^{-1}$ называется сопряженным элементом $h$.
	\end{definition}
	\begin{statement}
		$h \sim x^{-1}hx$~--- отношение эквивалентности.
	\end{statement}
	\begin{definition}
		$H \vartriangleleft G$. Класс смежности по $H$ можно перемножать.
	\end{definition}
	\begin{definition}
		Множество классов смежности образуют группу. Это называется фактор-группой $G$ по $H$. $G/H$.
	\end{definition}
	\begin{definition}
		Простая группа~--- группа без нетривиальных нормальных подгрупп.
	\end{definition}
	\subsection{Перестановки.}
	\begin{definition}
		Перестановкой конечного множества $M$ называется биекция $\pi : M \rightarrow M $. Множество всех перестановок множества $M$ обозначается символом $S(M)$. Произведением перестановок $\pi$ и $\sigma$ называется перестановка $\sigma \cdot \pi$, соответствующая биекции $x \mapsto \sigma (\pi (x))$.
	\end{definition}
	\begin{definition}
		Пусть $\pi \in S(M)$. Орбитой элемента $a \in M$ называется множество $\orb_{\pi}(a) = \{ a, \pi(a), \pi^2(a), \dots \}$. Порядком элемента $a$ называется мощность его орбиты: $\ord(a) = |\orb(a)|$.
	\end{definition}
	\begin{definition}
		В предыдущей серии показано, что любая перестановка является произведением циклов. Количество циклов в перестановке обозначается $\cycl(\pi)$. Если перестановка разбивается на циклы, длины которых~--- $l_1, \dots, l_m$, то $(l_1, \dots,l_m)$ называется цикленным типом перестановки $\pi$. Перестановка, у которой цикленный тип $(2, 1, 1, \dots, 1)$, называется транспозицией.
	\end{definition}
	\begin{definition}
		Перестановка $\pi$ называется четной или нечетной в зависимости от четности числа $n + \cycl(\pi)$. Знаком перестановки называется число $\text{sign}(\pi) = (-1)^{n + \cycl(\pi)}$.
	\end{definition}
	\begin{definition}
		Хотим $\forall$ перестановки называть ее четной или нечетной: ЧЧ $=$ Ч, НН $=$ Ч, ЧН $=$ Н, НЧ $=$ Н и $\exists$ Ч и НЧ перестановки.
	\end{definition}
	\begin{definition}
		Инверсия: $\alpha = \begin{pmatrix}
			1 & 2 & \dots & i & \dots & j & \dots & n \\
			a_1 & a_2 & \dots & a_i & \dots & a_j & \dots & n \\
		\end{pmatrix}$. Пара $(i, j)$ называется инверсией, если $i < j$ и $a_i > a_j$.
	\end{definition}
	\begin{definition}
		Перестановка с четным количеством инверсий~--- четная, с нечетным~--- нечетная.
	\end{definition}
	\begin{note}
		$\forall$ перестановка~--- произведение транспозиций. Четное количество транспозиций $\Leftrightarrow$ четная перестановка.
	\end{note}
	\begin{definition}
		Четная перестановка $= (n + \text{количество циклов}) \mod{2}$.
	\end{definition}
	\begin{theorem}
		Перестановки сопряжены $\Leftrightarrow$ совпадает их циклический тип.
	\end{theorem}
	\subsection{Центр группы.}
	\begin{definition}
		$G$~--- группа. Центр $G$: $Z(G) := \{ g \in G | gh = hg, \forall g \in G \}$.
	\end{definition}
	\begin{statement}
		$Z(S_n) = \{ e \}$.
	\end{statement}
	\begin{theorem}
		$Z(G) \vartriangleleft G$.
	\end{theorem}
	\begin{statement}
		$Z(GL(\mathbb{R})) = \begin{pmatrix}
			\alpha & 0 & \dots & 0 \\
			0 & \alpha & \dots & 0 \\
			\vdots & \vdots & \ddots & \vdots \\
			0 & 0 & \dots & \alpha
		\end{pmatrix}$
	\end{statement}
	\begin{definition}[Коммутатор]
		$x, y \in G$. $[x, y] := xyx^{-1}y^{-1}$.
	\end{definition}
	\begin{statement}
		$[x, y]^{-1} = [y, x]$.
	\end{statement}
	\begin{note}
		Произведение коммутаторов не обязательно является коммутатором.
	\end{note}
	\begin{definition}[Коммутант]
		$[G, G] := \left< [x, y] \right>$.
	\end{definition}
	\begin{theorem}
		\begin{enumerate}
			\item $[G, G] \vartriangleleft G$.
			\item $G / [G; G]$~--- абелева.
			\item $G / H$~--- абелева $\Rightarrow H \supset [G, G]$.
		\end{enumerate}
	\end{theorem}
	\begin{statement}
		$\alpha^{-1} [x, y] \alpha = [\alpha^{-1} x \alpha, \alpha^{-1} y \alpha]$.
	\end{statement}
	\begin{statement}
		$\alpha^{-1} [x_1, y_1] [x_2, y_2] \dots [x_k, y_k] \alpha \in [G, G]$.
	\end{statement}
	\begin{theorem}[Галуа.]
		$A_n$ простая, если $n \geqslant 5$.
	\end{theorem}
	
	
	
	
	
	%\section{Теория узлов.}
	%\begin{definition}
	%	Узел~--- замкнутая, несамопересекающаяся кривая (ломаная с конечным числом звеньев/гладкая кривая) в $\mathbb{R}^3$.
	%\end{definition}
	%\begin{definition}
	%	Кривая Пеано~--- $f: [0, 1] \rightarrow [0, 1] \times [0, 1]$; непрерывная и сюръективная.
	%\end{definition}
	%\begin{definition}
	%	Эквивалентные узлы~--- узлы, которые переходят из одного в другой с помощью следующих преобразований (преобразования Рейдемейчстера): избавиться от петли, растащить две дуги, перетащить нижнюю дугу, через точку пересечения двух других, на верх.
	%\end{definition}
	%\begin{note}
	%	Открытые проблемы:
	%	\begin{itemize}
	%		\item $\# (n)$~--- количество точек с $n$ двойных узлов в минимальной конфигурации. Экспериментально~--- функция не растет. Но не доказано.
	%		\item $c(K)$~--- количество двойных точек в минимальной конфигурации. Гипотеза: $c(K_1 \# K_2) = c(K_1) + c(K_2)$.
	%	\end{itemize}
	%\end{note}
\end{document}