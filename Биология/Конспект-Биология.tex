\documentclass[12pt]{article}

\input{/home/ilden/Documents/School/TeX-Cheat-Sheet/header.tex}

\begin{document}
	\tableofcontents
	\setcounter{tocdepth}{3}
	\newpage
	\section{Биология.}
	\begin{xltabular}{\textwidth}{|m{3cm}|X|m{3cm}|}
		\hline
		Направление & ОХ & Ученный\\
		\hline
		Классическое. & Изучает многообразие живой природы. Наблюдает и анализирует все в живой природе. & Гиппократ, Аристотель, Теофраст.\\
		\hline
		Эволюционное. & Изучает эволюцию живых организмов. Объяснение органического разнообразия природы. & Дарвин, Шлейден, Опарин, Ламарк.\\
		\hline
		Физико-химическое. & Изучение с использованием новых физико-химических методов и знаний. & Мечников, Пастер, Кох, Гарвей.\\
		\hline
	\end{xltabular}
	\begin{xltabular}{\textwidth}{|m{3.8cm}|X|m{3cm}|}
		\hline
		Метод & ОХ & Ученый\\
		\hline
		Описание. & Наблюдение и фиксирование фактического материала. Самый древний. Основной метод примерно до $18$ века. & Гиппократ, Аристотель, Теофраст.\\
		\hline
		Сравнение. & Сходства и различия организмов. Данные для систематизации. & Аристотель, Ламарк, Бэр.\\
		\hline
		Исторический. & Осмысление факторов по предыдущем результатам. & Дарвин, Ламарк.\\
		\hline
		Экспериментальный. & Изучение при помощи опытов. Дополнительные вспомогательные инструменты. & Гарвей, Мендель, Матье Бал, Кох.\\
		\hline
	\end{xltabular}
	\subsection{Свойства живого.}
	\begin{itemize}
		\item Обмен веществ (дыхание, пищеварение).
		\item Раздражимости (реакция на окружающую среду).
		\item Рост (количественное) и развитие (качественное).
		\item Размножение.
		\item Единство химического состава (основные~--- $C, O, H, N$).
		\item Структурная организация.
		\item Открытость.
		\item Наследственность и изменчивость.
		\item Саморегуляция.
	\end{itemize}
	\subsection{Уровни организации живой материи.}
	Молекулярный уровень~--- вирусы. Клеточный~--- бактерии. Организменный~--- одно- и многоклеточные. Популяционно-видовой. Экосистемный. Биосферный.
	\section{Клетка.}
	\begin{itemize}
		\item Наименьшая структурная единица.
		\item Наименьшая функциональная единица.
	\end{itemize}
	\subsection{Клеточная теория.}
	\begin{person}
		Роберт Гук. Первый микроскоп. Ввел понятие "клетка".
	\end{person}
	\begin{person}
		Антони ван Левенгук, XVI век. Первый микроскоп с увеличением в $300$ раз.
	\end{person}
	\begin{person}
		Шлейден и Шванн, XIX век. Положения клеточной теории. Ошибка в том, что не было объяснено откуда появляются клетки (считали, что появились из неклеточного вещества).
	\end{person}
	\begin{person}
		Мечников, конец XIX века. Фагоцитоз (процесс, когда клетки захватывают и переваривают твердые частицы).
	\end{person}
	\subsection{Молекулярный уровень.}
	Химические элементы:
	\begin{itemize}
		\item Макро; до $\dfrac{1}{100}$; основные~--- $C, O, H, N$.
		\item Микро; от $\dfrac{1}{1000}$ до $\dfrac{1}{1000000}$.
		\item Ульра-микро.
	\end{itemize}
	\subsection{Вещества клетки.}
	\begin{itemize}
		\item Органические (большая часть органики~--- белки).
		\item Неорганические (преобладают из-за воды).
	\end{itemize}
	\subsubsection{Вода.}
	\begin{xltabular}{\textwidth}{|m{3.5cm}|X|X|}
		\hline
		Свойство & ОХ & Пример\\
		\hline
		Растворитель. & Легко растворяет ионные соединения (соли, кислоты, основания); некоторые не ионные, но полярные соединения. Вещества, хорошо растворимые в воде~--- гидрофильные, плохо~--- гидрофобные. Благодаря полярности и водородных связях. & Кислород, углекислый газ.\\
		\hline
		Теплоемкость. & Способность поглощать тепловую энергию при минимальном повышении собственной температуры. & Защищает ткани от быстрого и сильного повышения температуры. Охлаждение с помощью выделения воды.\\
		\hline
		Теплопроводность. & Обеспечение равномерного распределения температуры. & Высокая удельная теплоемкость и высокая теплопроводность делают воду идеальной жидкостью для поддержания теплового равновесия клетки и организма.\\
		\hline
		Сжимаемость. & Практически не сжимается. Создает тургорное давление, определяя объем и упругость клеток и тканей. & Гидростатический скелет поддерживает форму у круглых червей, медуз и других.\\
		\hline
		Поверхностное натяжение. & Возникает благодаря образованию водородных связей между молекулами воды и молекулами других веществ. & Капилярный кровоток, восходящий и нисходящий токи растворов в растениях.\\
		\hline
	\end{xltabular}
	\section{Минеральные вещества.}
	\begin{xltabular}{\textwidth}{|X|X|X|}
		\hline
		Свойство & Химический элемент & ОХ\\
		\hline
		Кристаллические включения. & Слаборастворимые соли кальция и фосфора. & Образование опорных структур клетки, например вещества костных ткани у моллюсков.\\
		\hline
		Проводимость. & Катионы и Анионы минеральных веществ. & Разность потенциалов из-за различной концентрации.\\
		\hline
		Кислотность. & Ионы $H^+$. & Нейтральные, кислотные, основные. Определяют кислотную среду.\\
		\hline
		Буферные системы. & $HPO_4^{2-}$, $H_2PO_4^-$, $H_2CO_3$, $HCO_4^-$. & Поддерживает постоянство $pH$ в клетках.\\
		\hline
		Синтез. & Соединения азота, фосфора, кальция и другие неорганические вещества. & Синтез белков, аминокислот, нуклеиновых кислот.\\
		\hline
	\end{xltabular}
	\section{Органические вещества.}
	\subsection{Углеводы.}
	Углеводы ($C_n(H_2O)_m$):
	\begin{itemize}
		\item Моносахариды
		\item Олигосахариды
		\item Полисахариды
	\end{itemize}
	Сахариды так как большинство хорошо растворимы в воде; сладкие.\\
	С увеличением количества мономеров растворимость полисахаридов уменьшается и исчезает сладкий вкус.\\
	Углеводы являются первичным продуктом фотосинтеза.\\
	Углеводы есть во всех клетках.
	\begin{xltabular}{\textwidth}{|X|X|X|}
		\hline
		Группа & Пример & Особенность\\
		\hline
		Моносахариды. & Рибоза, глюкоза, фруктоза, дезоксирибоза, галактоза. & Имеют сладкий вкус, бесцветные, кристаллические, растворимые, во всех клетках, являются мономерами.\\
		\hline
		Олигосахариды. & Сахароза, мальтоза, лактоза & Образованы двумя или более моносахаридами. Также растворимы в воде и имеют сладковатый вкус. Связаны ковалентно друг с дургом.\\
		\hline
		Полисахариды. & Хитин, крахмал, гликоген, целлюлоза. & Полимеры. Состоят из неопределенного большого числа остатков молекул моносахаридов.\\
		\hline
	\end{xltabular}
	\begin{xltabular}{\textwidth}{|X|X|X|}
		\hline
		Функция & Пример углевода & Характеристика\\
		\hline
		Энергетическая. & Моносахариды (глюкоза). & При ферментативном расщеплении и окислении молекул углеводов выделяется энергия, которая обеспечивает жизнедеятельность организма. При полном расщеплении $1$г углеводов высвобождает $17.6$кДж энергии.\\
		\hline
		Запасающая. & Полисахариды (крахмал и гликоген). & При избытке они накапливаются в клетке в качетсве запасающих веществ и при необходимости используется организмом как источник энергии.\\
		\hline
		Структурная/строительная. & Целлюлоза, хитин. & Строительный материал. В среднем $20$--$40\%$ материала клеточных стенок составляет целлюлоза.\\
		\hline
		Защитная. & Камеди $\rightarrow$ производный моносахаридов. & Препятствуют проникновению в раны болезнетворных микроорганизмов. Твердые клеточные стенки одноклеточных и хитиновые покровы членистоногих.\\
		\hline
	\end{xltabular}
	\subsection{Липиды или жиры.}
	Молекул жира состоит из глицерина и трех остатков жирной кислоты. Иногда вместо остатка жирной кислоты могут быть белки, углеводы или остатки фосфорной кислоты.\\
	Более $600$ жиров. $180$~--- животных, $420$~--- растительных.\\
	Жиры бывают:
	\begin{itemize}
		\item Протоплазменный.
		\item Резервный.
	\end{itemize}
	\begin{xltabular}{\textwidth}{|X|X|X|}
		\hline
		Функция & Пример & Характеристика\\
		\hline
		Энергетическая & Триглицериды (жиры и масла) & Основная функция. При окислении 1 г жира выделяется около 38,9 кДж (9,3 ккал) энергии, что более чем в два раза превышает энергетическую ценность углеводов или белков. Жиры служат основным запасом энергии в организме.\\
		\hline
		Структурная (строительная) & Фосфолипиды, холестерин & Образование клеточных мембран. Фосфолипиды формируют липидный бислой всех клеточных мембран, обеспечивая их текучесть и избирательную проницаемость. Холестестрол стабилизирует мембрану, придавая ей жесткость.\\
		\hline
		Запасающая & Триглицериды (в жировой ткани) & Создание резервов энергии. Жиры запасаются в подкожной клетчатке, сальнике и вокруг внутренних органов. Жировые запасы также обеспечивают механическую защиту (амортизация) и термоизоляцию.\\
		\hline
		Регуляторная (гормональная) & Стероидные гормоны (половые гормоны, кортикостероиды), эйкозаноиды (простагландины) & Липиды выступают в роли гормонов и сигнальных молекул. Стероиды регулируют обмен веществ, репродуктивную функцию, стрессовые реакции. Эйкозаноиды регулируют воспаление, боль, температуру тела, артериальное давление.\\
		\hline
		Защитная и теплоизоляционная & Триглицериды (подкожный жир) & Защита от механических повреждений и потерь тепла. Жировая прослойка смягчает удары и защищает внутренние органы. Благодаря низкой теплопроводности жир помогает сохранять тепло организма (особенно важно у морских млекопитающих).\\
		\hline
		Источник метаболической воды & Триглицериды & При окислении жиров образуется вода. Из 100 г жира получается около 107 мл воды. Это особенно важно для животных пустыни (верблюды, тушканчики) и впадающих в спячку (сурки, медведи).\\
		\hline
		Каталитическая (ферментативная)	& Жирорастворимые витамины (A, D, E, K) & Витамины-липиды являются коферментами или предшественниками коферментов. Например, витамин А входит в состав зрительного пигмента родопсина; витамин К необходим для синтеза факторов свертывания крови.\\
		\hline
		Улучшение вкуса пищи и насыщения & Триглицериды & Жиры улучшают вкусовые качества пищи и продлевают чувство сытости, так как они медленно перевариваются и подавляют секрецию желудочного сока.\\
		\hline
	\end{xltabular}
	\subsection{Белки.}
	Белок~--- полимерная молекула. Его мономером является аминокислота ($20$ штук). Белки $=$ протеины $=$ полипептиды.
	\paragraph{Аминокислота.} Общая формула: $NH_2 - CH (R) - COOH$. По радикалу ($R$) определяем аминокислоту. $NH_2$~--- $N$-конец аминокислоты, $COOH$~--- $C$-конец аминокислоты.\\
	\subsubsection{Структура белка.}
	\begin{enumerate}
		\item Первичная структура белка в виде цепочки; индивидуальна для каждого белка. Очень большая, поэтому клетки не удобно.
		\item Вторичная структура белка в виде спирали. Удерживается водородными связями.
		\item Третичная (глобал). Спираль упаковывается в шарик. Образуется за счет связей внутри радикалов.
		\item Четвертичная. Несколько глобал, соединенных между собой. Характерна только для белков с очень важной функцией.
	\end{enumerate}
	\begin{definition}
		Денатурация~--- разрушение структуры белка. Ренатурация~--- восстановление структуры белка (возможна, если белок не утратил первичную структуру).
	\end{definition}
	\begin{xltabular}{\textwidth}{|X|X|X|}
		\hline
		Функция & Пример & Характеристика\\
		\hline
		Структурная (опорная) & Коллаген, кретин & Образуют волокна и сети, обеспечивающие прочность и эластичность тканей. Коллаген — основа соединительной ткани (сухожилия, хрящи), кератин — основной белок волос, ногтей, перьев.\\
		\hline
		Ферментативная (каталитическая) & Амилаза, пепсин, РНК-полимераза & Биологические катализаторы (ферменты), которые в тысячи раз ускоряют химические реакции в клетке. Амилаза расщепляет крахмал, пепсин — белки в желудке.\\
		\hline
		Транспортная & Гемоглобин, транспортные белки мембраны & Связывают и переносят различные вещества. Гемоглобин переносит кислород в крови. Белки-переносчики в мембранах транспортируют ионы и молекулы.\\
		\hline
		Защитная & Антитела (иммуноглобулины), фибриноген & Распознают и обезвреживают чужеродные объекты (вирусы, бактерии). Фибриноген участвует в свёртывании крови, предотвращая кровопотерю.\\
		\hline
		% Двигательная (сократительная) & Актин, миозин & Образуют сократимые структуры в мышечных волокнах, обеспечивая движение (мышцы, движение органелл внутри клетки).\\
		% \hline
		Регуляторная & Инсулин, гормон роста & Белки-гормоны регулируют обмен веществ и физиологические процессы. Инсулин, например, регулирует уровень глюкозы в крови.\\
		\hline
		% Рецепторная (сигнальная) & Белки-рецепторы (например, родопсин в сетчатке) & Расположены в мембранах клеток, принимают сигналы (свет, гормоны) из внешней среды и передают их внутрь клетки.\\
		% \hline
		% Запасающая (резервная) & Казеин (в молоке), яичный альбумин & Накопление питательных веществ для последующего использования организмом или зародышем.\\
		% \hline
		Энергетическая & Любой белок (в крайних случаях) & При недостатке углеводов и жиров белки могут расщепляться для получения энергии (при этом выделяется около $17,6$ $\frac{\text{кДж}}{\text{г}}$).\\
		\hline
	\end{xltabular}
	\subsection{Нуклеиновые кислоты.}
	Нуклеиновые кислоты~--- полимеры, их мономеры~--- нуклеотиды.
	\subsubsection{Основные нуклеиновые кислоты.}
	ДНК и РНК. Их состав: фосфатная группа, пентозный сахар и азотистое основание.
	\begin{xltabular}{\textwidth}{|X|X|X|}
		\hline
		Признак & ДНК & РНК\\
		\hline
		Название & Дезоксирибонуклеиновая кислота & Рибонуклеиновая кислота\\
		\hline
		Белок & Дезоксирибоза & Рибоза\\
		\hline
		Основание & Аденин ($2$ водородные связи), гуанин ($3$), цитозин ($3$), \textit{тимин} ($2$) & Аденин ($2$), гуанин ($3$), цитозин ($3$), \textit{урацил} ($2$)\\
		\hline
		Водородные связи & Постоянные & Временные\\
		\hline
		Внешний вид & Спираль. $5$'-конец (фосфатная группа) и $3$'-конец (пентозный сахар) & Также $3$'- и $5$'- концы\\
		\hline
		Местоположение & Ядро клетки, митохондрии, пластиды & Цитоплазма, рибосома, ядро, митохондрии, пластиды\\
		\hline
	\end{xltabular}	\begin{figure}[H]
		\includegraphics[height=0.25\textwidth]{extra-materials/тРНК}
		\caption{транспортная РНК}
	\end{figure}
	\begin{xltabular}{\textwidth}{|m{1.5cm}|m{1.5cm}|X|X|X|}
		\hline
		Название & Процент & Местоположение & ОХ & Функция\\
		\hline
		иРНК (мРНК) & $1-5\%$ & Ядро (в процессе синтеза), цитоплазма, рибосомы & Одноцепочечная молекула, образующаяся в процессе транскрипции на матрице ДНК. Имеет самую большую длину среди РНК. Нестабильна. & Перенос генетической информации от ДНК в ядре к рибосомам в цитоплазме, где служит матрицей для синтеза белка.\\
		\hline
		тРНК & $10-15\%$ & Цитоплазма, рибосомы & Небольшая молекула ($70-90$ нуклеотидов), имеющая сложную пространственную структуру ("клеверный лист"). Имеет участок для присоединения аминокислоты (акцепторный стебель) и антикодон. & Транспорт специфических аминокислот к растущей полипептидной цепи на рибосоме. Узнаёт свой кодон в иРНК благодаря антикодону.\\
		\hline
		рРНК & $80-85\%$ & Синтезируется в ядрышке, составляет основу рибосом & Самый распространённый тип РНК. Составляет вместе с белками субъединицы рибосом. Имеет сложную вторичную и третичную структуру. & Структурная (является каркасом рибосомы) и каталитическая (рибозимы): обеспечивает связывание рибосомы с иРНК, катализирует образование пептидных связей между аминокислотами.\\
		\hline
	\end{xltabular}
	\subsection{АТФ.}
	Аденозинтрифосфат.
	\subsubsection{Состав.}
	Аденин + рибоза + три остатка фосфорной кислоты (именно они определяют свойства АТФ; между ними макроэргическая связь). При отделении третьего и второго остатка фосфорной кислоты (разрушение макроэргической связи) выделяется до $40$ кДж энергии. При отделении первого остатка от углевода выделяется $14$ кДж.
	\subsubsection{Синтез АТФ.}
	Синтез проходит в митохондриях. Аденозинмонофосфат (АМФ) $\rightarrow$ аденозиндифосфат (АДФ) $\rightarrow$ аденозинтрифосфат (АТФ).
	\subsection{Витамины.}
	Открыты Луниным в $1880$ году. Термин "Витамины" введен в $1912$ году Функом.\\
	Суточная доля витаминов мала, они не заменяемые и не синтезируются.
	\subsubsection{Виды.}
	\begin{itemize}
		\item Водорастворимые. Основные: $C, B, PP, H$.
		\item Жирорастворимые. Основные: $A, D, E, K$.
	\end{itemize}
	\subsection{Сравнение АТФ, ДНК, РНК.}
	Сходства: общее строение, аналогичное местоположение.
	\section{Клеточный уровень.}
	Клетки есть у животных, растений, грибов, бактерий.\\
	Клетка~--- наименьшая структурная и функциональная единица.\\
	Науки~--- цитология, молекулярная биология, биохимия.\\
	\begin{xltabular}{\textwidth}{|X|X|X|}
		\hline
		Часть & ОХ & функция\\
		\hline
		Цитоплазма. & Основное вещество~--- гиалоплазма. Представляет собой густой бесцветный коллоидный раствор органических и неорганических веществ. Основа гиалоплазмы~--- вода ($70-90\%$ от массы), в ней много белков, обнаруживаются также липиды и различные неорганические соединения. Цитоплазма постоянно перемещается внутри клетки. & В ней протекают процессы обмена веществ в клетке, через нее происходит взаимодействие ядра и органоидов.\\
		\hline 
		Клеточная мембрана. & Толщина $8-12$ нМ. Универсальная биологическая мембрана, окружающая клетку. Имеет жидкомозаичную структуру: двойной слой липидов, в который погружены белки. Углеводы образуют гликокаликс~--- наружный слой. Обладает избирательной проницаемостью. & \begin{enumerate}
			\item Барьерная: Отделяет содержимое клетки от внешней среды, защищает от повреждений.
			\item Транспортная: Обеспечивает избирательный перенос веществ в клетку и из нее (диффузия, осмос, активный транспорт, эндо- и экзоцитоз).
			\item Рецепторная: Белки-рецепторы принимают сигналы из внешней среды (например, гормоны), обеспечивая коммуникацию с другими клетками.
			\item Структурная (опорная): Придает клетке форму, служит местом прикрепления цитоскелета.
		\end{enumerate}\\
		\hline
		Генетический аппарат. & Центр управления клетки; локализовано более $90\%$ ДНК. Обычно имеет шаровидную форму. Отделен от цитоплазмы оболочкой, состоящей из двух мембран. Содержит хроматин (комплекс ДНК и белков), который во время деления конденсируется в хромосомы. Внутри находится одно или несколько ядрышек. & \begin{enumerate}
			\item Хранение наследственной информации: В ДНК ядра закодирована вся генетическая информация о строении и функциях клетки и организма.
			\item Реализация наследственной информации: Контроль всех процессов жизнедеятельности клетки через регуляцию синтеза белков (транскрипция ДНК $\rightarrow$ иРНК).
			\item Воспроизведение и передача информации: Удвоение ДНК (репликация) перед делением клетки, что обеспечивает передачу генетического материала дочерним клеткам.
			\item Образование субъединиц рибосом: Происходит в ядрышке.
		\end{enumerate}\\
		\hline
	\end{xltabular}
	\subsection{Органоиды или органеллы.}
	\begin{itemize}
		\item Мембранные
		\begin{itemize}
			\item Одно-мембранные: вакуоль, аппарат Гольджи, ЭПС, лизосома.
			\item Двух-мембранные
		\end{itemize}
		\item Не мембранные
	\end{itemize}
	\section{Вирусы.}
	\begin{itemize}
		\item \textbf{Сущность:} Не клеточные формы жизни. Занимают положение между живой и неживой природой. Вне клетки хозяина существуют в виде кристаллоподобных частиц (\textbf{вирионов}) и не проявляют признаков жизни.
		\item \textbf{Строение:} Очень простое. Состоят из \textbf{генетического материала} (ДНК \textbf{или} РНК) и \textbf{белковой оболочки} (\textbf{капсида}). У некоторых есть дополнительная липопротеидная суперкапсидная оболочка.
		\item \textbf{Жизненный цикл:}
		\begin{enumerate}
			\item \textbf{Прикрепление} к специфической клетке-хозяину.
			\item \textbf{Проникновение} внутрь клетки и ``раздевание'' (высвобождение генома).
			\item \textbf{Встраивание} своего генома в генетический аппарат хозяина.
			\item \textbf{Репликация} -- использование ресурсов и систем клетки для создания своих компонентов (нуклеиновых кислот и белков).
			\item \textbf{Сборка} новых вирусных частиц.
			\item \textbf{Выход} из клетки (часто с её разрушением, т.е. \textbf{лизисом}).
		\end{enumerate}
		\item \textbf{Специфика:} \textbf{Абсолютные паразиты}, не способны к самостоятельному обмену веществ и размножению вне клетки-хозяина. Поражают животных, растения, бактерии, археи.
	\end{itemize}
	\subsection{Бактериофаги.}
	\begin{itemize}
		\item \textbf{Сущность:} Это \textbf{частный случай вирусов}, специализированные паразиты бактерий (и архей). Название буквально означает ``пожиратель бактерий''.
		\item \textbf{Строение:} Часто имеют сложную структуру. Классический ``фаг'' похож на космический посадочный модуль:
		\begin{itemize}
			\item \textbf{Головка} (капсид с генетическим материалом, обычно ДНК).
			\item \textbf{Хвостовой отросток} (чехол), через который геном впрыскивается в бактерию.
			\item \textbf{Базальная пластинка} и \textbf{нити} для прикрепления к клеточной стенке бактерии.
		\end{itemize}
		\item \textbf{Жизненный цикл:} Аналогичен общему вирусному, но имеет две стратегии:
		\begin{enumerate}
			\item \textbf{Литический цикл:} Классический, как описано выше, заканчивается быстрым разрушением (лизисом) бактериальной клетки и выходом новых фагов.
			\item \textbf{Лизогенный цикл:} ДНК фага встраивается в хромосому бактерии (становится \textbf{профагом}) и пассивно реплицируется вместе с ней долгое время, не вызывая гибели клетки. При определенных условиях профаг может активироваться и перейти в литический цикл.
		\end{enumerate}
		\item \textbf{Специфика:} Высокая видоспецифичность (обычно поражают только определенный вид или штамм бактерий). Играют огромную роль в регуляции бактериальных сообществ в природе.
	\end{itemize}
	\section{Метаболизм.}
	Состоит из двух противоположных процессов:
	\begin{itemize}
		\item Энергетический обмен (диссимиляция, катаболизм). Энергия выделяется, вещество разрушается.
		\item Пластический обмен (ассимиляция, анаболизм). Энергия поглощается, вещество синтезируется.
	\end{itemize}
	Метаболизм поддерживает гомеостаз (постоянство внутренней среды). Обменные процессы протекают с помощью ферментов (веществ, которые ускоряют химические реакции в живых организмах). Примеры ферментов: амилаза (катализирует распад крахмала в ротовой полости), уреаза (катализирует расщепление мочевины до аммиака и угольной кислоты).
	\subsection{Энергетический обмен в клетке.}
	Три этапа:
	\begin{itemize}
		\item Подготовительный
		\item Бескислородный
		\item Кислородный
	\end{itemize}
	\begin{xltabular}{\textwidth}{|m{1cm}|m{3.5cm}|m{1.8cm}|m{2.2cm}|X|X|m{1cm}|X|}
		\hline
		Этап & Название этапа & Организм & Место & Исходные вещества & Конечные вещества & АТФ & ОХ\\
		\hline
		I & Подготовительный & Аэробы и анаэробы. & Лизосомы, органы пищеварения. & Крупные пищевые полимеры. Полисахариды. Белки. Жиры. & Мелкие фрагменты. Ди- и моносахариды. Аминокислоты. Глицерин и жирные кислоты. & -- & Мало тепла.\\
		\hline
		II & Бескислородный (гликолиз) & Аэробы и анаэробы. & Цитоплазма клеток. & Конечные вещества первого этапа. & 2 ПВК $+$ вода. & $2$ & У некоторых грибов спиртовым брожением. Не много тепла. $40\%$ АТФ, остальное рассеивается.\\
		\hline
		III & Кислородный & Аэробы & На мембранах митохондрий, кристах. & Конечные вещества второго этапа. & Углекислый газ и вода. Образовывается $6$ молекул углекислого газа, $42$ молекулы воды. & $36$ & КПД выше. Пользуются не все, тк опасно. Цикл Крепса.\\
		\hline
	\end{xltabular}
	\subsection{Общая формула энергетического обмена.}
	$C_6H_{12}O_6 + 6O_2 + 38\text{АДФ} + 38H_3PO_4 \rightarrow 6CO_2 + 44H_2O + 38\text{АТФ}$
	\subsubsection{Задачи.}
	\begin{enumerate}
		\item В процессе гликолиза образовалось $112$ молекул ПВК. Какое количество глюкозы подверглось расщеплению? Сколько молекул АТФ образовалось при полном окислении глюкозы у эукариотов.
		\item В процессе кислородного этапа катаболизма образовалось $972$ АТФ. Какое количество молекул глюкозы подверглось расщеплению? Сколько молекул АТФ образовалось в процессе гликолиза и полного окисления.
		\begin{solve}
			Глюкоза $= \dfrac{972}{36} = 27$\\
			АТФ $=$ глюкоза$\cdot 2 +$глюкоза$\cdot 36 = 27 \cdot 2 + 27 \cdot 36 = 1026$
		\end{solve}
		\item В процессе гликолиза образовалось 84 молекул ПВК. Какое количество молекул глюкозы подверглось расщеплению? Сколько молекул АТФ образовалось при полном окислении?
	\end{enumerate}
	\subsection{Пластический обмен на примере фотосинтеза.}
	\begin{xltabular}{\textwidth}{|m{2cm}|m{2cm}|m{2cm}|m{2.2cm}|m{2cm}|X|}
		\hline
		Фаза & Место & АТФ & Исходные вещества & Конечные вещества & ОХ\\
		\hline
		Световая & Внутри мембранных хлоропластов (на гранах хлоропластов) & Образуется $1$ & АДФ, вода, свет & АТФ, ионы водорода, кислород $\uparrow$ &
		\begin{enumerate}[1)]
			\item Фотолиз. $2H_2O \rightarrow 4H^+ + 4e^- + O_2 \uparrow$
			\item Выделяется кислород.
			\item Обязателен свет $\rightarrow$ 1 квант.
			\item Молекула хлорофилла переходит в возбужденное состояние ($1 e^-$ молекулы получает избыток энергии). Энергия тратится на синтез АТФ.
			\item Процесс очень эффективен (в 30 раз больше, чем в митохондриях).
		\end{enumerate}\\
		\hline
		Темновая (так как без света) & Пластиды $\rightarrow$ хлоропласты & Не образуется & Углекислый газ, водород & Глюкоза и вещество, способное захватывать $CO_2$, вода &
		\begin{enumerate}[1)]
			\item Свет не нужен.
			\item $CO_2$ захватывается из внешней среды специальным веществом.
			\item Обеспечиваются энергией, запасенной в световой фазе.
		\end{enumerate}
		\\
		\hline
	\end{xltabular}
	\section{Синтез белка.}
	\begin{enumerate}
		\item Место. Белок синтезируется в рибосомах (не мембранные органоиды, состоящие из двух субъединиц).
		\item Необходимые вещества.
		\begin{enumerate}[I.]
			\item АТФ, так как энергоемкий процесс.
			\item Аминокислоты.
			\item ДНК и РНК.
			\item Ферменты.
			\item тРНК, иРНК, рРНК.
		\end{enumerate}
		\item Результат~--- белок. Мономером белка является аминокислота. В синтезе белка участвует $20$ аминокислот.
		\item Информация зашифрована генетическим кодом.
		Свойства:
		\begin{enumerate}[I.]
			\item Универсальность для всех живых организмов.
		\end{enumerate}
		\item ДНК. Мономером ДНК~--- является нуклеотид.
		Нуклеотид состоит из:
		\begin{enumerate}[I.]
			\item Азотистое основание (аденин, гуанин, цитозин, тимин).
			\item Углевод.
			\item Фосфорный остаток.
		\end{enumerate}
		\textbf{Триплет}~--- последовательность из $3$ нуклеотидов.
		\item РНК. Мономером РНК~--- является нуклеотид.
		Нуклеотид состоит из:
		\begin{enumerate}[I.]
			\item Азотистое основание (аденин, гуанин, цитозин, урацил).
			\item Углевод.
			\item Фосфорный остаток.
		\end{enumerate}
		\textbf{Кодон} (иРНК)~--- последовательность из $3$ нуклеотидов. Комплементарный с триплетом. \\
		\textbf{Антикодон} (тРНК)~--- триплет на тРНК, который подхватывает кислоту нужную для синтеза.
	\end{enumerate}
	\begin{xltabular}{\textwidth}{|p{0.15\textwidth}|p{0.1\textwidth}|p{0.15\textwidth}|p{0.15\textwidth}|X|}
		\hline
		Этап & Место & Исходные вещества & Конечные вещества & ОХ \\
		\hline
		Транскрипция (считывание) & Ядро у эукариотов, в цитоплазме у прокариотов & ДНК $\rightarrow$ триплет (белки, энергия АТФ, нуклеотиды) & иРНК $\rightarrow$ кодон &
		\begin{enumerate}[1)]
			\item Информация переходит от ДНК к РНК.
			\item Г -- Ц, А -- У, Т -- А, Ц -- Г.
			\item Переписывание II цепочки ДНК в иРНК, комплементарную I.
			\item У прокариотов нет.
		\end{enumerate} \\
		\hline
		Трансляция (передача) & На рибосомах (в цитоплазмах) & Нуклеотиды & Аминокислоты &
		\begin{enumerate}[1)]
			\item Происходит расшифровка генетической информации.
			\item В цитоплазме должны быть все аминокислоты (одни из белков из пищи, другие синтезируются).
			\item Рибосома передвигается по иРНК (задержка 0.2 с)
			тРНК ищет комплементарный кусочек.
			\item Заканчивается, когда появляется стоп-триплет.
			\item Когда рибосома сдвигается, на ее место сразу приходит другая. Полисома~--- все рибосомы, синтезирующие один и тот же белок от одной и той же иРНК.
		\end{enumerate} \\
		\hline
	\end{xltabular}
	\subsection{Отличие процесса синтеза белка у эукариотов и прокариотов.}
	\begin{enumerate}
		\item Разные места. У эукариотов начинается в ядре, у прокариотов в цитоплазме.
		\item Разные механизмы регуляции. У эукариотов гораздо сложнее.
		\item Кодирование белков. У эукариотов гены могут быть закодированы в генах различных хромосом, когда у прокариотов ДНК в клетке представлена одной-единственной молекулой.
	\end{enumerate}
	\section{Размножение клетки.}
	Жизнь любого организма, кроме вирусов, начинается с клетки.
	\subsection{Клеточный цикл.}
	Клеточный цикл~--- жизнь клетки с момента ее образования, до момента ее гибели или деления.
	\subsection{Митоз.}
	\begin{itemize}
		\item Интерфаза.
		\begin{enumerate}
			\item Пред-синтетический. Клета растет и накапливает энергию. Самая длинная фаза. От $2$--$3$ часов, до нескольких суток. $2n2c$.
			\item Синтетический. Репликация ДНК. Удвоение всего необходимого. $6$--$10$ часов. $2n4c$.
			\item Пост-синтетический. Образование материала веретена деления. $2$--$5$ часов. $2n4c$.
		\end{enumerate}
		\item Деление (митоз).
		\begin{enumerate}
			\item Про-фаза. $2n4c$. Начало образование веретена деления.
			\item Мета-фаза. $2n4c$. Получилось веретено деления.
			\item Ана-фаза. $4n4c$.
			\item Тело-фаза. $\dfrac{\text{Ранняя~--- } 4n4c}{\text{Поздняя~--- } 2n2c \times 2}$.
		\end{enumerate}
	\end{itemize}
	\subsubsection{Апоптоз.}
	``Запрограммированная'' клеточная смерть.
	\subsubsection{Значение митоза.}
	\begin{itemize}
		\item Рост.
		\item Регенерация.
		\item Размножение.
	\end{itemize}
	Митоз это непрямое деление.
	\subsection{Амитоз.}
	\begin{itemize}
		\item Прямое деление клетки.
		\item Редкое.
		\item Начинается с ядра и без видимых изменений.
		\item Не равномерное распределение ДНК, хромосомы не образуются.
		\item Иногда не происходит цтокинез (деление цитоплазмы). Тогда образуется двуядерная клетка.
		\item Велика вероятность, что дочерние клетки будут неполноценными.
	\end{itemize}
	\subsubsection{Значение амиотза.}
	\begin{itemize}
		\item Нужен для отмирающих тканей и опухолей, чтобы контролировать численность организмов на земле.
	\end{itemize}
	\subsection{Мейоз.}
	\begin{itemize}
		\item Интерфаза.
		\begin{enumerate}
			\item Пред-синтетический. Клета растет и накапливает энергию. Самая длинная фаза. От $2$--$3$ часов, до нескольких суток. $2n2c$.
			\item Синтетический. Репликация ДНК. Удвоение всего необходимого. $6$--$10$ часов. $2n4c$.
			\item Пост-синтетический. Образование материала веретена деления. $2$--$5$ часов. $2n4c$.
		\end{enumerate}
		\item Мейоз.
		\begin{enumerate}
			\item Про-фаза I. Конъюгация (сближение) $\rightarrow$ кроссинговер (обмен). $2n4c$.
			\item Мета-фаза I. Образование пластинки. $2n4c$.
			\item Ана-фаза I. $2n4c$.
			\item Тело-фаза I. К полюсам расходятся хромосомы. $\dfrac{\text{Ранняя~--- } 2n4c}{\text{Поздняя~--- } n2c \times 2}$.
			\item Про-фаза II. $n2c$.
			\item Мета-фаза II. $n2c$.
			\item Ана-фаза II. $2n2c$.
			\item Тело-фаза II. $\dfrac{\text{Ранняя~--- } 2n2c}{\text{Поздняя~--- } nc \times 2}$.
		\end{enumerate}
	\end{itemize}
	\subsubsection{Значения мейоза.}
	\begin{itemize}
		\item Образование гамет, необходимых для полового размножения.
		\item Генетическое разнообразие.
	\end{itemize}
	\section{Размножение.}
	\begin{itemize}
		\item Половое.
		\begin{itemize}
			\item Специализированные половое клетки.
			\item Есть генетическое разнообразие.
			\item Адаптация.
			\item Виды:
			\begin{itemize}
				\item Оплодотворение. Образуется зигота, затем зародыш.
				\item Неотения.
				\item Конъюгация. Спирагира, инфузория. Обмен малыми ядрами.
				\item Копуляция. Целые клетки-организмы превращаются в неотличимые друг от друга гаметы и сливаются, образуя зиготу.
			\end{itemize}
		\end{itemize}
		\item Бесполое.
		\begin{itemize}
			\item Отсутствие спец клеток, кроме спор.
			\item Нет генетического разнообразие.
			\item Количество (расселение).
			\item Виды:
			\begin{itemize}
				\item Вегетативный. С помощью вегетативных органов.
				\item Почкование. Дрожжи, кактусы, некоторые виды папоротников, кишечно-полостные, губки, оболочники.
				\item Фрагментация (архитомия и паратомия). Основана на процессе регенерации. Как правило растения у которых есть корневище, червяки, морские звезды.
				\item Деление.
				\item Споровое.
			\end{itemize}
		\end{itemize}
	\end{itemize}
	\paragraph{ВСР.} У высших споровых растений гаметы формируются в архигониях и антеридиях.
	\paragraph{Семенные растения.} Женский гаметофит представлен зародышевым мешком, мужской~--- пыльцевым зерном.
	\paragraph{Животные.} Половое размножение началось с медуз.
	\begin{definition}
		Гаметогенез~--- процесс образование гамет. Процесс образования женских гамет~--- оогенез, мужских~--- сперматогенез.
	\end{definition}
	\begin{xltabular}{\textwidth}{|X|X|X|X|}
		\hline
		Фаза/этап & Набор хромосом & Оогенез & Сперматогенез\\
		\hline
		Размножение & $2n$ & Деление половых клеток митозом. Делятся только в период внутриутробного развития плода и до наступления полового созревания сохраняются в покое. & Деление половых клеток митозом. С момента наступления половой зрелости, до глубокой старости.\\
		\hline
		Рост & $2n$ & Увеличение в размерах яйцеклеток. Происходит репликация ДНК, запасание веществ, необходимых для последующих делений. & Увеличение в размере сперматозоидов. Происходит репликация ДНК, запасание веществ, необходимых для последующих делений.\\
		\hline
		Созревание & $2n \rightarrow n$ & Во время этой фазы будущие гаметы делятся мейозом, в результате которого из каждой диплоидной клетки получается $4$ гаплоидных. Образуется одна яйцеклетка. & Во время этой фазы будущие гаметы делятся мейозом, в результате которого из каждой диплоидной клетки получается $4$ гаплоидных. Образуется 4 сперматозоида. \\
		\hline
		Формирование & & & Только в сперматогенезе. У сперматозоидов образуются специфические приспособления.\\
		\hline
	\end{xltabular}
	\subsection{Виды оплодотворения по месту.}
	\begin{itemize}
		\item Внешнее. Минусы: нужны: определенная среда, большое количество гамет; данный процесс сложен.
		\item Внутреннее. Минусы: специализированные органы. Плюсы: высокая выживаемость, меньшее количество гамет.
	\end{itemize}
	\subsection{Виды оплодотворения по количеству.}
	\begin{itemize}
		\item Простое. $\venus (n) + \mars (n) =$ зигота $(2n)$ $\xrightarrow{\div}$ зародыш $(2n)$.
		\item Двойное.
		\begin{enumerate}
			\item $\mars (n) + \venus (n) =$ зигота $(2n)$ $\xrightarrow{\div}$ зародыш $(2n)$.
			\item $\mars (n) +$ центральная клетка $(2n) =$ триплоид $(3n)$ $\xrightarrow{\div}$ эндосперм $(3n)$.
		\end{enumerate}
		Семенная кожура~--- покровы семязачатка.
	\end{itemize}
	\subsection{Лирическое отступление.}
	Горизонтальные переносы генов:
	\begin{quote}
		\begin{definition}
			Конъюгация~--- процесс сближения бактерий и обмена плазмидами по пили.
		\end{definition}
		\begin{definition}
			Трансдукция~--- перенос генов с участием генов.
		\end{definition}
		\begin{definition}
			Трансформация~--- подбор ДНК из вне.
		\end{definition}
	\end{quote}
	\begin{definition}
		Пили~--- волосок, по которому передается информация между бактериями.
	\end{definition}
	\begin{definition}
		Нуклеоид~--- основой геном бактерии. Огромная молекула, которая состоит из миллионов пар азотистых оснований.
	\end{definition}
	\begin{definition}
		Плазмиды~--- штука, которая копируется по пили из одной бактерии в другую.
	\end{definition}
	\paragraph{Жизнь высших растений.} Поколения чередуются: спорофит и гаметофит.
	\paragraph{Хвощи, плауны, папоротники.} Маленький фотосинтезирующий спорофит. У плаунов гаметофиты подземные. Свободно живущий гаметофит. Мужской гаметофит~--- заросток.
	\paragraph{Семенные растения.} Мужской гаметофит~--- пыльцевое зерно. Женский гаметофит~--- первичный эндосперм.
	\paragraph{Цветковые растения.} Зародышевый мешок.
	
	\subsection{Этапы онтогенеза.}
	\begin{enumerate}
		\item Начинается с эпизода размножения.
		\item При половом размножении первой стадией будет зигота~--- результат слияния половых клеток (гамет), которое увеличивает плоидность в 2 раза.
	\end{enumerate}
	\subsubsection{Типы онтогенеза.}
	\paragraph{С точки зрения ресурсов для развития эмбрионов.}
	\begin{itemize}
		\item Яйцерождение~--- откладка яйца во внешнюю среду, развитие за счет желтка.
		\item Яйцеживорождение~--- зародыш развивается в половых путях матери, но под скорлупироваными оболочками за счет желтка.
		\item Живорождение~--- зародыш развивается в половых путях матери, получая от нее питательные вещества через плаценту (не в виде желтка).
	\end{itemize}
	\paragraph{С точки зрения стадий постнатального развития.}
	\begin{itemize}
		\item Прямое развитие.
		\item Личиночное развитие.
	\end{itemize}
	\subsubsection{Фазы развития.}
	\begin{enumerate}
		\item Эмбриональный этап.
		\item Личиночный.
		\item Постнатальное развитие.
		\begin{itemize}
			\item Ювенильный.
			\item Пубертатный.
			\item Сенильный.
		\end{itemize}
	\end{enumerate}
	\subsubsection{Стадии онтогенеза.}
	\begin{enumerate}
		\item Личиночный. Стадия личинки. Насекомые, рыбы. Непрямое развитие, через стадию личинки.
		\item Яйцекладный. Яйцо, покрытое специальной оболочкой. Рептилии, птицы, ехидны, утконосы и др первозвери.
		\item Внутриутробный.\\
		Эмбриональный период:
		\begin{enumerate}
			\item Дробление ($1-32$)
			\item Бластула
			\item Гаструла
			\item Нейрула
		\end{enumerate}
		Слои:
		\begin{enumerate}
			\item Эктодерма (наружная, бластула). Кожа.
			\item Энтодерма (внутренняя, гаструла). Печень, поджелудочная железа.
			\item Мезодерма (промежуточная, нейрула). Мышцы, скелет, половая система, кровеносная система.
		\end{enumerate}
	\end{enumerate}
	\section{Генетика.}
	
	\subsection{Термины.}
	
	\begin{definition}[Наследственность]
		Способность живых организмов передавать свои признаки, свойства и особенности развития следующему поколению.
	\end{definition}
	
	\begin{definition}[Изменчивость]
		Свойство организмов приобретать новые признаки и различия в процессе индивидуального развития, отличающие их от родительских форм.
	\end{definition}
	
	\begin{definition}[Аутосомы]
		Парные хромосомы соматических клеток, одинаковые у мужских и женских организмов одного вида (все хромосомы, за исключением половых).
	\end{definition}
	
	\begin{definition}[Ген]
		Структурная и функциональная единица наследственности; участок молекулы ДНК, содержащий информацию о первичной структуре одного белка или молекулы РНК.
	\end{definition}
	
	\begin{definition}[Локус]
		Строго определенное местоположение конкретного гена в хромосоме.
	\end{definition}
	
	\begin{definition}[Генотип]
		Совокупность всех генов (аллелей) организма, полученных им от родителей.
	\end{definition}
	
	\begin{definition}[Фенотип]
		Совокупность всех внешних и внутренних признаков организма, сформировавшихся в результате взаимодействия генотипа с факторами внешней среды.
	\end{definition}
	
	\begin{definition}[Аллели]
		Различные формы одного и того же гена, расположенные в одинаковых локусах гомологичных хромосом и отвечающие за альтернативные варианты проявления одного признака (например, желтая и зеленая окраска семян).
	\end{definition}
	
	\begin{definition}[Зигота (гетеро и гомо)]
		Клетка, образующаяся при слиянии гамет. В зависимости от аллельного состава различают:
		
		\begin{itemize}
			\item \textbf{Гомозигота}~--- организм (зигота), имеющий одинаковые аллели данного гена (например, $AA$ или $aa$) и не дающий расщепления в потомстве.
			
			\item \textbf{Гетерозигота}~--- организм (зигота), имеющий разные аллели данного гена (например, $Aa$) и дающий расщепление признаков в потомстве.
		\end{itemize}
	\end{definition}
	
	\begin{definition}[Гомологичные хромосомы]
		Парные хромосомы, одинаковые по форме, размерам и набору генов, но различные по происхождению (одна получена от матери, другая~--- от отца).
	\end{definition}
	
	\begin{definition}[Гибрид]
		Организм, полученный в результате скрещивания генетически различающихся родительских форм.
	\end{definition}
	
	\begin{definition}[Метод Менделя]
		Гибридологический метод~--- система скрещиваний, позволяющая проследить закономерности наследования отдельных признаков в ряду поколений при половом размножении.
	\end{definition}
	
	\begin{definition}[Чистые линии]
		Группа организмов, которые являются генетически однородными (гомозиготными) по подавляющему большинству генов и при половом размножении (путем самоопыления или близкородственного скрещивания) дают потомство, в точности повторяющее признаки родителей.
	\end{definition}
	
	\begin{definition}[Анализирующее скрещивание]
		Проводится с рецессивной гомозиготой для выявления генотипа скрещиваемого организма. Может быть два результата:
		
		\begin{enumerate}
			\item Расщепление по фенотипу не происходит (значит скрещиваемые организм был гомозиготный).
			
			\item Расщепление по фенотипу происходит (значит организм был гетерозиготный).
		\end{enumerate}
	\end{definition}
	
	\begin{person}[Грегор Мендель, 19 век]
		Основоположник генетики. Использовал гибридологический метод, для обработки результатов статистический метод. Проводил исследования на горохе (бобовые); изучил около 22 различных сортов. В результате вывел три основных закона:
		
		\begin{enumerate}
			\item Независимое наследование.
			
			\item Единообразие гибридов первого поколения.
			
			\item Закон расщепления: при скрещивании гибридов первого поколения (гетерозиготных) в результате получается расщепление по фенотипу 3 к 1 (25 рецессивных гомозигот), по генотипу 1 к 2 к 1.
		\end{enumerate}
	\end{person}
	
	\subsection{Изменчивость.}
	
	\paragraph{Виды изменчивости.}
	
	\begin{itemize}
		\item Наследственная (генотипическая).
		\item Не наследственная (фенотипическая, модификационная).
	\end{itemize}
	
	\begin{definition}
		Изменчивость~--- приобретение организмом новых свойств.
	\end{definition}
	
	\subsubsection{Сравнение наследственной и не наследственной изменчивости.}
	\begin{xltabular}{\textwidth}{|X|X|X|}
		\hline
		Признак & Наследственная & Не наследственная\\
		\hline
		Что изменяется (причина) & Изменяется генотип (структура генов, хромосом или их сочетания). & Изменяется только фенотип под влиянием условий среды; генотип остается неизменным.\\
		\hline
		Передача потомству & Передается по наследству (если происходит в половых клетках). & Не передается по наследству.\\
		\hline
		Характер проявления & Индивидуальный (появляется у единичных особей), случайный и ненаправленный. & Групповой (массовый) — проявляется у всех особей вида, помещенных в одинаковые условия.\\
		\hline
		Значение для эволюции & Является материалом для естественного отбора, ведет к образованию новых видов. & Способствует выживанию организма (адаптации) в конкретных условиях, но не создает новые виды.\\
		\hline
		Обратимость	& Практически необратима (изменения стабильны и сохраняются в поколениях). & Часто обратима (если условия среды меняются, признак может вернуться к исходному состоянию).\\
		\hline
		Пределы изменчивости (Ограниченность) & Практически неограничена. Мутации могут привести к появлению совершенно новых признаков, которых раньше не было. & Ограничена нормой реакции. Степень изменения признака не может выйти за границы, определенные генотипом.\\
		\hline
		Формы изменчивости & Подразделяется на: \begin{enumerate}
			\item Комбинативную (перекомбинация генов при скрещивании).
			\item Мутационную (внезапные сбои в ДНК).
		\end{enumerate} & Существует в виде модификаций (длительных или кратковременных изменений признака в ответ на среду).\\
		\hline
	\end{xltabular}
\end{document}
