\documentclass[12pt]{article}

\input{C:/Users/ilden/Documents/School/TeX памятка/header.tex}

\begin{document}
	\tableofcontents
	\setcounter{tocdepth}{3}
	\newpage
	\section{Биология.}
	\begin{xltabular}{\textwidth}{|p{3cm}|X|p{3cm}|}
		\hline
		Направление & ОХ & Ученный \\
		\hline
		Классическое. & Изучает многообразие живой природы. Наблюдает и анализирует все в живой природе. & Гиппократ, Аристотель, Теофраст. \\
		\hline
		Эволюционное. & Изучает эволюцию живых организмов. Объяснение органического разнообразия природы. & Дарвин, Шлейден, Опарин, Ламарк. \\
		\hline
		Физико-химическое. & Изучение с использованием новых физико-химических методов и знаний. & Мечников, Пастер, Кох, Гарвей. \\
		\hline
	\end{xltabular}
	\begin{xltabular}{\textwidth}{|p{3.8cm}|X|p{3cm}|}
		\hline
		Метод & ОХ & Ученый \\
		\hline
		Описание. & Наблюдение и фиксирование фактического материала. Самый древний. Основной метод примерно до $18$ века. & Гиппократ, Аристотель, Теофраст. \\
		\hline
		Сравнение. & Сходства и различия организмов. Данные для систематизации. & Аристотель, Ламарк, Бэр. \\
		\hline
		Исторический. & Осмысление факторов по предыдущем результатам. & Дарвин, Ламарк. \\
		\hline
		Экспериментальный. & Изучение при помощи опытов. Дополнительные вспомогательные инструменты. & Гарвей, Мендель, Матье Бал, Кох. \\
		\hline
	\end{xltabular}
	\subsection{Свойства живого.}
	\begin{itemize}
		\item Обмен веществ (дыхание, пищеварение).
		\item Раздражимости (реакция на окружающую среду).
		\item Рост (количественное) и развитие (качественное).
		\item Размножение.
		\item Единство химического состава (основные~--- $C, O, H, N$).
		\item Структурная организация.
		\item Открытость.
		\item Наследственность и изменчивость.
		\item Саморегуляция.
	\end{itemize}
	\subsection{Уровни организации живой материи.}
	Молекулярный уровень~--- вирусы. Клеточный~--- бактерии. Организменный~--- одно- и многоклеточные. Популяционно-видовой. Экосистемный. Биосферный.
	\section{Клетка.}
	\begin{itemize}
		\item Наименьшая структурная единица.
		\item Наименьшая функциональная единица.
	\end{itemize}
	\subsection{Клеточная теория.}
	\begin{person}
		Роберт Гук. Первый микроскоп. Ввел понятие "клетка".
	\end{person}
	\begin{person}
		Антони ван Левенгук, XVI век. Первый микроскоп с увеличением в $300$ раз.
	\end{person}
	\begin{person}
		Шлейден и Шванн, XIX век. Положения клеточной теории. Ошибка в том, что не было объяснено откуда появляются клетки (считали, что появились из неклеточного вещества).
	\end{person}
	\begin{person}
		Мечников, конец XIX века. Фагоцитоз (процесс, когда клетки захватывают и переваривают твердые частицы).
	\end{person}
	\subsection{Молекулярный уровень.}
	Химические элементы:
	\begin{itemize}
		\item Макро; до $\dfrac{1}{100}$; основные~--- $C, O, H, N$.
		\item Микро; от $\dfrac{1}{1000}$ до $\dfrac{1}{1000000}$.
		\item Ульра-микро.
	\end{itemize}
	\subsection{Вещества клетки.}
	\begin{itemize}
		\item Органические (большая часть органики~--- белки).
		\item Неорганические (преобладают из-за воды).
	\end{itemize}
	\subsubsection{Вода.}
	\begin{xltabular}{\textwidth}{|p{3.5cm}|X|X|}
		\hline
		Свойство & ОХ & Пример \\
		\hline
		Растворитель. & Легко растворяет ионные соединения (соли, кислоты, основания); некоторые не ионные, но полярные соединения. Вещества, хорошо растворимые в воде~--- гидрофильные, плохо~--- гидрофобные. Благодаря полярности и водородных связях. & Кислород, углекислый газ. \\
		\hline
		Теплоемкость. & Способность поглощать тепловую энергию при минимальном повышении собственной температуры. & Защищает ткани от быстрого и сильного повышения температуры. Охлаждение с помощью выделения воды. \\
		\hline
		Теплопроводность. & Обеспечение равномерного распределения температуры. & Высокая удельная теплоемкость и высокая теплопроводность делают воду идеальной жидкостью для поддержания теплового равновесия клетки и организма. \\
		\hline
		Сжимаемость. & Практически не сжимается. Создает тургорное давление, определяя объем и упругость клеток и тканей. & Гидростатический скелет поддерживает форму у круглых червей, медуз и других. \\
		\hline
		Поверхностное натяжение. & Возникает благодаря образованию водородных связей между молекулами воды и молекулами других веществ. & Капилярный кровоток, восходящий и нисходящий токи растворов в растениях. \\
		\hline
	\end{xltabular}
	\section{Минеральные вещества.}
	\begin{xltabular}{\textwidth}{|X|X|X|}
		\hline
		Свойство & Химический элемент & ОХ \\
		\hline
		Кристаллические включения. & Слаборастворимые соли кальция и фосфора. & Образование опорных структур клетки, например вещества костных ткани у моллюсков. \\
		\hline
		Проводимость. & Катионы и Анионы минеральных веществ. & Разность потенциалов из-за различной концентрации. \\
		\hline
		Кислотность. & Ионы $H^+$. & Нейтральные, кислотные, основные. Определяют кислотную среду. \\
		\hline
		Буферные системы. & $HPO_4^{2-}$, $H_2PO_4^-$, $H_2CO_3$, $HCO_4^-$. & Поддерживает постоянство $pH$ в клетках. \\
		\hline
		Синтез. & Соединения азота, фосфора, кальция и другие неорганические вещества. & Синтез белков, аминокислот, нуклеиновых кислот. \\
		\hline
	\end{xltabular}
	\section{Органические вещества.}
	Углеводы ($C_n(H_2O)_m$):
	\begin{itemize}
		\item Моносахариды
		\item Олигосахариды
		\item Полисахариды
	\end{itemize}
	Сахариды так как большинство хорошо растворимы в воде; сладкие. \\
	С увеличением количества мономеров растворимость полисахаридов уменьшается и исчезает сладкий вкус. \\
	Углеводы являются первичным продуктом фотосинтеза. \\
	Углеводы есть во всех клетках.
	\begin{xltabular}{\textwidth}{|X|X|X|}
		\hline
		Группа & Пример & Особенность \\
		\hline
		Моносахариды. & Рибоза, глюкоза, фруктоза, дезоксирибоза, галактоза. & Имеют сладкий вкус, бесцветные, кристаллические, растворимые, во всех клетках, являются мономерами. \\
		\hline
		Олигосахариды. & Сахароза, мальтоза, лактоза & Образованы двумя или более моносахаридами. Также растворимы в воде и имеют сладковатый вкус. Связаны ковалентно друг с дургом. \\
		\hline
		Полисахариды. & Хитин, крахмал, гликоген, целлюлоза. & Полимеры. Состоят из неопределенного большого числа остатков молекул моносахаридов. \\
		\hline
	\end{xltabular}
	\begin{xltabular}{\textwidth}{|X|X|X|}
		\hline
		Функция & Пример углевода & Характеристика \\
		\hline
		Энергетическая. & Моносахариды (глюкоза). & При ферментативном расщеплении и окислении молекул углеводов выделяется энергия, которая обеспечивает жизнедеятельность организма. При полном расщеплении $1$г углеводов высвобождает $17.6$кДж энергии. \\
		\hline
		Запасающая. & Полисахариды (крахмал и гликоген). & При избытке они накапливаются в клетке в качетсве запасающих веществ и при необходимости используется организмом как источник энергии. \\
		\hline
		Структурная/строительная. & Целлюлоза, хитин. & Строительный материал. В среднем $20$--$40\%$ материала клеточных стенок составляет целлюлоза. \\
		\hline
		Защитная. & Камеди $\rightarrow$ производный моносахаридов. & Препятствуют проникновению в раны болезнетворных микроорганизмов. Твердые клеточные стенки одноклеточных и хитиновые покровы членистоногих. \\
		\hline
	\end{xltabular}
	\section{Липиды или жиры.}
	Молекул жира состоит из глицерина и трех остатков жирной кислоты. Иногда вместо остатка жирной кислоты могут быть белки, углеводы или остатки фосфорной кислоты. \\
	Более $600$ жиров. $180$~--- животных, $420$~--- растительных. \\
	Жиры бывают:
	\begin{itemize}
		\item Протоплазменный.
		\item Резервный.
	\end{itemize}
	\begin{xltabular}{\textwidth}{|X|X|X|}
		\hline
		Функция & Пример & Характеристика \\
		\hline
		Энергетическая & Триглицериды (жиры и масла) & Основная функция. При окислении 1 г жира выделяется около 38,9 кДж (9,3 ккал) энергии, что более чем в два раза превышает энергетическую ценность углеводов или белков. Жиры служат основным запасом энергии в организме. \\
		\hline
		Структурная (строительная) & Фосфолипиды, холестерин & Образование клеточных мембран. Фосфолипиды формируют липидный бислой всех клеточных мембран, обеспечивая их текучесть и избирательную проницаемость. Холестестрол стабилизирует мембрану, придавая ей жесткость. \\
		\hline
		Запасающая & Триглицериды (в жировой ткани) & Создание резервов энергии. Жиры запасаются в подкожной клетчатке, сальнике и вокруг внутренних органов. Жировые запасы также обеспечивают механическую защиту (амортизация) и термоизоляцию. \\
		\hline
		Регуляторная (гормональная) & Стероидные гормоны (половые гормоны, кортикостероиды), эйкозаноиды (простагландины) & Липиды выступают в роли гормонов и сигнальных молекул. Стероиды регулируют обмен веществ, репродуктивную функцию, стрессовые реакции. Эйкозаноиды регулируют воспаление, боль, температуру тела, артериальное давление. \\
		\hline
		Защитная и теплоизоляционная & Триглицериды (подкожный жир) & Защита от механических повреждений и потерь тепла. Жировая прослойка смягчает удары и защищает внутренние органы. Благодаря низкой теплопроводности жир помогает сохранять тепло организма (особенно важно у морских млекопитающих). \\
		\hline
		Источник метаболической воды & Триглицериды & При окислении жиров образуется вода. Из 100 г жира получается около 107 мл воды. Это особенно важно для животных пустыни (верблюды, тушканчики) и впадающих в спячку (сурки, медведи). \\
		\hline
		Каталитическая (ферментативная)	& Жирорастворимые витамины (A, D, E, K) & Витамины-липиды являются коферментами или предшественниками коферментов. Например, витамин А входит в состав зрительного пигмента родопсина; витамин К необходим для синтеза факторов свертывания крови. \\
		\hline
		Улучшение вкуса пищи и насыщения & Триглицериды & Жиры улучшают вкусовые качества пищи и продлевают чувство сытости, так как они медленно перевариваются и подавляют секрецию желудочного сока. \\
		\hline
	\end{xltabular}
\end{document}
