\documentclass{article}

\input{C:/Users/ilden/Documents/School/TeX памятка/header.tex}

\begin{document}
	\tableofcontents
	\setcounter{tocdepth}{3}
	\newpage
	\section{Биология.}
	\begin{xltabular}{\textwidth}{|p{3cm}|X|p{3cm}|}
		\hline
		Направление & ОХ & Ученный \\
		\hline
		Классическое. & Изучает многообразие живой природы. Наблюдает и анализирует все в живой природе. & Гиппократ, Аристотель, Теофраст. \\
		\hline
		Эволюционное. & Изучает эволюцию живых организмов. Объяснение органического разнообразия природы. & Дарвин, Шлейден, Опарин, Ламарк. \\
		\hline
		Физико-химическое. & Изучение с использованием новых физико-химических методов и знаний. & Мечников, Пастер, Кох, Гарвей. \\
		\hline
	\end{xltabular}
	\begin{xltabular}{\textwidth}{|p{3.5cm}|X|p{3cm}|}
		\hline
		Метод & ОХ & Ученый \\
		\hline
		Описание. & Наблюдение и фиксирование фактического материала. Самый древний. Основной метод примерно до $18$ века. & Гиппократ, Аристотель, Теофраст. \\
		\hline
		Сравнение. & Сходства и различия организмов. Данные для систематизации. & Аристотель, Ламарк, Бэр. \\
		\hline
		Исторический. & Осмысление факторов по предыдущем результатам. & Дарвин, Ламарк. \\
		\hline
		Экспериментальный. & Изучение при помощи опытов. Дополнительные вспомогательные инструменты. & Гарвей, Мендель, Матье Бал, Кох. \\
		\hline
	\end{xltabular}
	\subsection{Свойства живого.}
	\begin{itemize}
		\item Обмен веществ (дыхание, пищеварение).
		\item Раздражимости (реакция на окружающую среду).
		\item Рост (количественное) и развитие (качественное).
		\item Размножение.
		\item Единство химического состава (основные~--- $C, O, H, N$).
		\item Структурная организация.
		\item Открытость.
		\item Наследственность и изменчивость.
		\item Саморегуляция.
	\end{itemize}
	\subsection{Уровни организации живой материи.}
	Молекулярный уровень~--- вирусы. Клеточный~--- бактерии. Организменный~--- одно- и многоклеточные. Популяционно-видовой. Экосистемный. Биосферный.
\end{document}
