\documentclass[12pt]{article}

\input{C:/Users/ilden/Documents/School/TeX памятка/header.tex}

\begin{document}
	\tableofcontents
	\setcounter{tocdepth}{4}
	\newpage
	\section{Политология.}
	\begin{definition}
		Политология~--- (с греческого) искусство управления полисом (городом). Человек не участвующий в жизни полиса~--- идиот.
	\end{definition}
	\noindent
	Политика строится вокруг государства.
	\subsection{Государство.}
	\begin{definition}
		Государство~--- политический институт с научной точки зрения; политическая организация обладающая властью на определенной территории. С бытовой~--- страна, народ страны.
	\end{definition}
	\begin{person}[Никколо Макиавелли]
		Флорентийский (Итальянский) мыслитель $16$ века, впервые заговоривший о государстве.
	\end{person}
	\subsubsection{Признаки государства.}
	\paragraph{Основные признаки государства.}
	\begin{itemize}
		\item Суверинетет:
		\begin{itemize}
			\item Внутренний~--- полнота власти на своей территории; у государства есть право на легальное применение силы.
			\item Внешний~--- независимость от других государств.
		\end{itemize}
		У государства должен быть и внутренний и внешний суверенитет, иначе это не государство.
		\item Территория. У государства обязательно должна быть своя территория, иначе тяжело управлять государством, сбором налогов и тому подобное.
		\item Наличие государственной власти. Власть состоит из двух частей:
		\begin{itemize}
			\item Аппарат управления.
			\item Аппарат принуждения.
		\end{itemize}
		\item Регулярный сбор налогов.
		\item Законодательная деятельность.
	\end{itemize}
	\paragraph{Дополнительные признаки.}
	\begin{itemize}
		\item Наличие собственной государственной символики (герб, гимн, флаг, особые атрибуты).
		\item Государственный язык.
		\item Гражданство.
		\item Обладание определенными материальными ресурсами.
	\end{itemize}
	\subsubsection{Функции государства.}
	\paragraph{Внутренние функции государства.}
	\begin{itemize}
		\item Правовая~--- обеспечение правопорядком, законности, борьба с преступностью.
		\item Экономическая~--- организация хозяйственной жизни страны.
		\item Социальная~--- выполнение обязательств перед обществом (система здравоохранения, система образования, поддержка материнства и детства, помощь малоимущим, выплата пенсий и так далее).
		\item Образовательная (иногда выделяется из социальной).
		\item Экологическая функция.
	\end{itemize}
	\paragraph{Внешние функции государства.}
	\begin{itemize}
		\item Оборонная~--- защита границ и противодействие внешним угрозам.
		\item Внешне-политическая (дипломатическая).
		\item Международное сотрудничество.
	\end{itemize}
	\subsection{Политический режим.}
	\begin{definition}
		Политический режим~--- то, как совершается управление государством.
	\end{definition}
	\subsubsection{Основные типы политических режимов.}
	\begin{enumerate}
		\item Демократический~--- власть в основном договаривается с обществом.
		\item Тоталитарный~--- метод подчинения, навязывания, насилия.
		\item Авторитарный~--- комбинация демократического и тоталитарного режимов.
	\end{enumerate}
	\subsubsection{Тоталитарный режим.}
	\begin{definition}
		От латинского totalize~--- полный; вся полнота власти над обществом сосредоточена в руках государства, отсутствие предела контроля.
	\end{definition}
	\paragraph{Признаки тоталитаризма.}
	\begin{itemize}
		\item Наличие единственной господствующей идеологии, остальные являются враждебными и преследуются.
		\item Однопартийность.
		\item Экономика подчинена государству, свободного рынка нет.
		\item Опора на насилие, массовый террор.
	\end{itemize}
	\subsubsection{Демократический режим.}
	\begin{definition}
		Противоположность тоталитаризма. Государство существует для народа. Основной источник власти~--- народ.
	\end{definition}
	\paragraph{Признаки демократии.}
	\begin{itemize}
		\item Многопартийность.
		\item Свободный рынок.
	\end{itemize}
	\paragraph{Типы демократии.}
	\begin{itemize}
		\item Прямая~--- граждане принимают непосредственное участие в управлении государством. Референдум~--- всенародное голосование по особо важным вопросом; пример современной прямой демократии. Особенность в том, что референдум не имеет обязательной юридической силы, в отличии от плебисцита.
		\item Представительная~--- выбор представителей в органы государственной власти.
	\end{itemize}
	\subsubsection{Авторитарный режим.}
	\begin{definition}
		Авторитарный режим~--- тип не демократического политического режима, основанного на несменяемой централизованной власти одного лица или группы лиц при сохранении в стране определенного уровня экономических, гражданских и идейных свобод; осуществляется полный контроль, зачастую, только власти. Самый распространенный тип за всю историю.
	\end{definition}
	\subsection{Форма государства.}
	\begin{definition}
		Форма государства~--- особенность внутренней организации государства, которая охватывает форму правления, государственное устройство и политический режим.
	\end{definition}
	\subsubsection{Формы государства.}
	\begin{enumerate}
		\item Монархия.
		\item Республика.
	\end{enumerate}
	\subsubsection{Монархия.}
	\begin{definition}[Монархия]
		Власть принадлежит одному лицу.
	\end{definition}
	\paragraph{Признаки монархии.}
	\begin{itemize}
		\item Власть передается по наследству.
		\item Власть бессрочна.
		\item Власть неподвластна населению.
	\end{itemize}
	\begin{note}
		Самая древняя форма правления.
	\end{note}
	\paragraph{Древние формы монархий.}
	\begin{enumerate}
		\item Деспотия~--- власть правителя схожа с властью хозяина над рабами. Примеры: государства древнего Востока, древний Египет, древний Китай.
		\item Теократия~--- монарх представляется как живой бог (или наместник бога).
		\item Абсолютная монархия~--- вся власть (законодательная, исполнительная, судебная, военная и так далее) сконцентрирована в руках одного человека. Примеры: государства Ближнего Востока, Франция при Людовике XIV.
		\item Просвещенная монархия~--- предполагает определенные обязанности государственной власти. Монарх~--- не просто единоличный правитель, а образованный, просвещенный государь, который обязан заботится о просвещении своего народа и благе государства. Пример: Россия при Екатерине Великой. Важный шаг к демократии.
	\end{enumerate}
	\paragraph{Современные формы монархий.}
	\begin{enumerate}
		\item Абсолютная монархия~--- вся полнота власти в руках одного человека.
		\item Конституционная монархия~--- полномочия монарха ограничены конституцией.
		\begin{enumerate}
			\item Дуалистическая монархия~--- у монарха обширная власть. У парламента мало власти.
			\item Парламентская монархия~--- у монарха минимальная власть. По большей части функции монарха~--- символические. Законы принимает парламент. В некоторых случаях у монарха есть право вето, но этим право они не пользуются.
		\end{enumerate}
	\end{enumerate}
	\subsubsection{Республика.}
	\begin{definition}[Республика]
		Законодательная власть принадлежит выборному представительному органу (парламенту). Глава государства (президент) либо избирается всем населением, либо специальной избирательной комиссией.
	\end{definition}
	\paragraph{Признаки республики.}
	\begin{enumerate}
		\item Выборность власти.
		\item Срочность власти.
		\item Зависимость от воли избирателей.
	\end{enumerate}
	\paragraph{Типы республики.}
	\begin{enumerate}
		\item Президентская. Президент~--- глава исполнительной власти и государства. Избирается, как правило, прямым голосованием всех граждан. У президента большая власть и большой объем полномочий, но несмотря на все это он~--- не монарх, и его возможности ограничены парламентом.
		\item Парламентская~--- правительство формируется парламентом, и перед парламентом же несет ответственность. Парламент может выразить недовольство правительству или отдельным министрам, что может спровоцировать правительственный кризис. Глава государства~--- тоже президент, но полномочия весьма скромные, в отличии от полномочий в президентской республике; в основном представительные функции (мало чем отличаются от функций монарха в парламентской монархии).
		\item Смешанная~--- высшие властные полномочия распределенны между президентом и парламентом.
	\end{enumerate}
	\subsection{Формы государственного устройства.}
	\begin{definition}
		Государственное устройство~--- способ взаимодействия частей государства.
	\end{definition}
	\begin{enumerate}
		\item Унитарное государство~--- части такого государства подчинены центру. Единая конституция, единое право. Примеры: Великобритания, Польша, Финляндия, Китай.
		\item Федерация~--- объединение, союз. Части федерации обладают признаками суверенитет (имеют право на собственную конституцию, законодательство, судебную систему и тому подобное). Два уровня власти: федеральная (высшая) и власть субъектов. Примеры: США, Германия, Индия, Канада, Российская Федерация.
		\item Конфедерация~--- не государство, а объединение разных, абсолютно независимых политических субъектов (возможно государств). Примеры: Речь Посполитая.
		\item Содружество государств~--- новоя форма, которая частично заменили федерацию. Примеры: ЕС, СНГ, Британское содружество наций.
	\end{enumerate}
	\subsection{Политические партии и организации.}
	\begin{definition}
		Партия~--- группа лиц, которых объединяют общие интересы.
	\end{definition}
	\begin{definition}
		Политическая партия~--- политическая организация, которая выражает интересы определенных групп и слоев населения, которая борется за власть и осуществляет государственную власть в своих интересах.
	\end{definition}
	\begin{note}
		Первые полит кружки появились в начале 19 века~--- кружки декабристов. Но они не были партиями, так как не боролись за власть.
	\end{note}
	\subsubsection{Функции политических партий.}
	\begin{enumerate}
		\item Организационная (завлечение людей в партию).
		\item Представительная (выражение и представление определенных слоев).
		\item Идеологическая (пропаганда своих идей).
	\end{enumerate}
	\subsubsection{Признаки политических партий.}
	\begin{enumerate}
		\item Стремление к завоеванию власти.
		\item Наличие идеологии.
		\item Устойчивость.
		\item Наличие организации (устав, программа и членство). Программа~--- что партия будет делать, когда придет к власти. Устав~--- документ. в котором прописаны нормы внутрипартийной жизни. Членство~--- партия структурно состоит из руководства (люди, профессионально занятые политикой), передовых людей и социальной базы.
	\end{enumerate}
	\subsubsection{Виды политических партий.}
	\begin{enumerate}
		\item По идеологическому признаку.
		\begin{itemize}
			\item Либерализм. Свобода и права человека. Появились в $17 - 18$ веке, раньше консерваторов. \\
			Консерватизм. Сохранение старого и традиционных ценностей; людей надо научить пользоваться свободой, и лучше всего их этому научит государство. \\
			Обе идеологии выражали интересы собственников.
			\item Социал-демократизм. За социальную справедливость. чем беднее человек, тем меньше он платит налогов; социальная политика. Но при всем этом сохранение частной собственности. \\
			Коммунизм. Частная собственность~--- основа для эксплуатации человека человеком, поэтому ее надо ликвидировать. \\
			Выражали интересы трудящихся, в первую очередь рабочего класса.
			\item Националистические партии.
			\begin{itemize}
				\item Умеренный национализм. Регулирование потоков миграции.
				\item Радикальный национализм. Отмена миграции, этнические чистки. Такие националистические партии под запретом.
			\end{itemize}
		\end{itemize}
		\item По шкале политического спектра.
		\begin{itemize}
			\item Правые. Те, кто отстаивает силы государства, охраняющие частную собственность.
			\item Центристские. За компромис, соединение государственных и общественных интересов.
			\item Левые. Те, кто больше ориентирован на общество и общественные интересы.
		\end{itemize}
		\item По отношению к другим элементам политической системы.
		\begin{itemize}
			\item Демократические. Терпимость по отношению к другим партия, сотрудничество и ведения диалогов с ними.
			\item Не демократические. Не намерены вести диалоги, стремление подчинить себе другие партии, институты, государство.
		\end{itemize}
		\item По положению в отношении правления.
		\begin{itemize}
			\item Правящие. Те, кто победили на выборах и правят.
			\item Оппозиция. Те, кто не у власти. Как правило критикуют правящую партию, ждут их ошибки и пользуются этим, чтобы в будущем самим прийти к власти.
			\begin{itemize}
				\item Системная оппозиция. Представлена в парламенте.
				\item Несистемная оппозиция. Легальная в парламенте не представлена, но и не запрещена. Нелегальная~--- законом запрещена.
			\end{itemize}
		\end{itemize}
		\item По численности.
		\begin{itemize}
			\item Массовые. Заинтересована в максимальном количестве своих членов. Главный источник дохода таких партий~--- членские взносы.
			\item Кадровые. Имеют другое финансирование. Существуют за счет финансовой помощи спонсоров.
		\end{itemize}
	\end{enumerate}
	\subsubsection{Политические движения.}
	\begin{definition}
		Политические движения~--- отличаются от партий тем, что не участвуют в борьбе за власть. Также отсутствуют членство и организационное единство. В политической жизни участвуют, пытаясь воздействовать на власть и заявить о себе.
	\end{definition}
	\subsection{Партийные системы и их виды.}
	\begin{definition}
		Партийная система~--- система взаимодействия между партиями.
	\end{definition}
	\subsubsection{Виды партийных систем.}
	\begin{itemize}
		\item Однопартийная. Характерна для стран с не демократическими политическими режимами. При этом может сохраняться видимость многопартийности. Таким образом однопартийные политические системы делятся на:
		\begin{itemize}
			\item Реально однопартийные. Пример: СССР (партия КПСС).
			\item Формально многопартийные. Не смотря на существование нескольких партий власть контролируется одной партией. Пример: Китай, Северная Корея.
		\end{itemize}
		\item Двухпартийная. Характерна для стабильных, устоявшихся политических режимов. Пример: США, Великобритания, Австралия. Каждая из двух партия способна прийти к власти и сформировать правительство. При этом партий может быть больше. Причины двухпартийности:
		\begin{itemize}
			\item Бипартизм всегда имеет исторические корни.
			\item Психологическое значение.
		\end{itemize}
		Сильная сторона двухпартийной системы~--- обеспечение стабильности и порядка (одна партия правит, другая ее критикует, а потом они меняются местами). Но слабая сторона~--- неспособность такой системы выразить все многообразие мнений и взглядов народа.
		\item Многопартийная. Характеризует ситуаций когда партий не просто много/мало, а когда не одна из существующих партий не располагает достаточной поддержкой избирателей, чтобы самостоятельно прийти к власти. Для того, чтобы организовать правительство партиям необходимо объединяться и вступать в коалиции. Преимущества такой системы в том, что она лучше отображает интересы разных социальных групп. Но ее главный недостаток в том, что она не обеспечивает такой уровень стабильности как двухпартийная система.
		\item Двух с половиной партийная. Ни одна из соперничающих партий оказывается не в состоянии победить на выборах, набрать необходимое для организации правительства голосов. В такой ситуации несколько процентов голосов, которые может им дать какая-нибудь малозначительная партия, принесут им победу; и тогда эти партии объединяются и маленькая партия получает большое политическое значение.
		\item С доминирующей партией. Суть в том, что одна и та же партия в течении длительного времени побеждает всех своих конкурентов и находится у власти. Это может длиться годами, десятилетиями. Например либерально-демократическая партия Японии находилась у власти без перерыва $38$ лет, причем в стране не однопартийная система. Похожая ситуация была в Индии. Причин может быть несколько: люди просто искренне поддерживают партию, неосознанность населения ("лишь бы не стало хуже, поэтому менять ничего не будем").
	\end{itemize}
	\subsection{Разбор политики на примерах.}
	\subsubsection{Великобритания.}
	\begin{note}
		В Великобритании парламентская монархия. Король/королева несет только символическую роль. Основная власть принадлежит премьер-министру. Зачастую он~--- представитель партии, победившей на выборах в нижней палате. Нижняя палата является главной в английском парламенте, так как она общая и избирается на $5$ лет народным голосованием. Также есть верхняя палата Лордов, которая не избирается. До недавнего времени она формировалась по наследственному признаку. Сейчас же наследственным путем формируется $60\%$ палаты (наследственные пэры), а остальные получают место в палате за выдающиеся достижения (пожизненные пэры). Во главе верховной палаты~--- лорд-канцлер, который назначается монархом на 5 лет.
	\end{note}
	\begin{note}
		По форме государственного устройства Великобритания~--- унитарное государство, которое состоит из Англии, Уэльса, Шотландия и Северная Ирландия. Хотя все ее части претендуют на самостоятельность.
	\end{note}
	\begin{note}
		В Великобритании двухпартийная система. Партии~--- лейбористская и консервативная.
	\end{note}
	\subsubsection{США.}
	\begin{note}
		Президент представляет верховную власть. В его руках исполнительная власть, он руководит деятельностью правительства, внутренней и внешней политикой. Избирается на $4$ года, но не по средствам прямых выборов, а коллегией выборщиков. То есть население голосует не за кандидатов на пост президента, а за выборщиков. А сами выборщики уже голосуют за республиканцев или демократов. Так исторически сложилось, так как при всеобщих выборах: голоса малых штатов будут не сильно учитываться; когда принималось это законодательство в США не было сильных политических штатов, СМИ, и народ, в основной своей массе, был мало образован и мало грамотен, и тогда симпатией избирателей могли бы завладеть популисты, демагоги, болтуны и тому подобное, и поэтому было поручено оставить выборы президента уважаемым людям~--- выборщикам. Характерной чертой политической системы США является ее двухпартийность~--- демократы и республиканцы. Американский парламент называется \textbf{Конгресс}. Он состоит из двух палат: верхней палаты (сенат) и нижней (палаты представителей). В сенат входят по два представителя от каждого штата. Нижняя палат представляет американское общество в целом.
	\end{note}
	\begin{note}
		США~--- федеративной государство. Представляет собой $50$ штатов и один федеральный округ. У каждого штата свои законы, которые могут сильно различаться.
	\end{note}
	\subsubsection{Германия.}
	\begin{note}
		Парламентская республика. Глава государства~--- федеральный президент. Он избирается специально созываемым для этой цели органом~--- федеральным собранием, сроком на $5$ лет и реальной властью не обладает. Играет такую же символическую роль, как и Английская королева. Он формально назначает и увольняет министров, но предлагает кандидатуры министров канцлер~--- глава правительства. Президент теоретически может не подписать федеральный закон (но за всю историю Германии такое было только $8$ раз). Также он может объявить помилование (помиловать преступников, осуществить акт государственного милосердия). Президент, в основном, фигура символическая, надпартийная, он олицетворяет все немецкое общество, представляет все государство. Пока он находится в должности президента, он прекращает партийную деятельность. Реальной властью обладает канцлер~--- глава правительства. Он избирается на $4$-х летний срок бундестагом (парламент). И у него огромная власть. Он определяет внешнюю и внутреннюю политику. А Бундестаг издает федеральные законы и контролирует правительство. На ряду с Бундестагом есть еще Бундесрат~--- орган представления земель Германии ($16$ федеральных частей).
	\end{note}
	\begin{note}
		Многопартийная система. Крупнейшие партии: христианский-демократический союз и христианский-социальный союз. ХСС действует в основном на территории Баварии, а ХДС действует в остальных землях. ХСС и ХСД~--- блок умеренных консерваторов. Важны христианские ценности. Одна из старейших и влиятельных партий~--- Социал-демократическая партия германии. Известна антифашисткой борьбой.
	\end{note}
	\subsection{Избирательные системы.}
	\begin{definition}
		Избирательная система~--- то, как организованы выборы в той или иной стране. Включает в себя теоретическую часть (избирательное право) и практическую часть (избирательный процесс). Избирательное право делится на: объективное избирательное право (законы, на основе которых осуществляются выборы) и субъективное право (право избирать и право избираться). Субъективное право имеет возрастные ограничения: избирать (в России) можно только с $18$ лет, а избираться можно в зависимости от уровня выборов в разном возрасте (например в депутаты разных уровней можно избираться с $21$ года; на должность главы субъекта можно претендовать с $30$ лет; а на должность президента с $35$ лет).
	\end{definition}
	\subsubsection{Принципы демократической избирательной системы.}
	\begin{enumerate}
		\item Принцип равенства.
		\item Всеобщность. Отстранение от выборов после преступления. Ограничение, накладываемые на принцип всеобщности~--- ценз (основные цензы~--- возрастной, гражданский; также есть половой (чистая дискриминация), имущественный (люди с имуществом заинтересованы в стабильности), военный (в некоторых государствах не допускаются военные), грамотности).
		\item Принцип тайного голосования. Голосовать надо тайно, чтобы на избирателей нельзя было повлиять.
		\item Принцип состязательности. Выбор должен быть по крайней мере из двух кандидатов.
		\item Принцип гласности. Выборы должны быть открытыми и публичными. Можно наблюдать и за процессом выборов и за процессом подсчетов голосов.
		\item Принцип прямого голосования.
	\end{enumerate}
	\subsection{Системы выборов.}
	\subsubsection{Мажоритарная система.}
	\begin{definition}[Мажоритарная система (фр. Majoritaro~--- большинство)]
		При такой системе можно избрать только одного человека, набравшее установленное большинство. Кандидаты могут представлять партии, а могут быть самовыдвиженцами. Большинство бывает \textbf{относительными} (больше, чем остальные) и \textbf{абсолютным} (более $50 \%$).
	\end{definition}
	\paragraph{Плюсы мажоритарной системы.}
	\begin{enumerate}
		\item Хорошая связь между избирателями и депутатами.
	\end{enumerate}
	\paragraph{Минусы мажоритарной системы.}
	\begin{enumerate}
		\item Низкий уровень легитимности.\\Рассмотрим ситуацию, при которой за победителя проголосовали  $51\%$, тогда оставшиеся $49\%$ голосовали за других кандидата.
		\item Не ограничивает популизм. Кандидат может многое пообещать и ничего не исполнить.
		\item Не способствует развитию партийной системы.
	\end{enumerate}
	\subsubsection{Пропорциональная система.}
	\begin{definition}[Пропорциональная система]
		Вид выборов, при котором партии получают места в представительном органе пропорционально числу голосов, отданых за них.
	\end{definition}
	\paragraph{Плюсы пропорциональной системы.}
	\begin{enumerate}
		\item Более справедливая система. Партии выбираются в соответствие с их популярностью. Каждый избиратель найдет себе партию, будет минимум пропавших голосов.
		\item Способствует развитию партийной системы.
		\item Защита от популизма. Депутатов могут призвать к ответу внутри партии.
	\end{enumerate}
	\paragraph{Минусы пропорциональной системы.}
	\begin{enumerate}
		\item В парламент могут попасть мелкие партии, выражающие интересы очень маленькой группы. Для этого существует \textbf{\textit{избирательный порог}}. Избирательный порог в России~--- $5\%$.
		\item Выбираются не люди, а партии. Партия формирует списки депутатов, которые обычно заранее не оглашаются.
	\end{enumerate}
	\subsubsection{Смешанная система.}
	\begin{definition}[Смешанная система]
		Способ проведения выборов, при котором часть депутатов (или представителей) избирается по одной системе, а другая часть — по другой. Чаще всего это комбинация мажоритарной и пропорциональной систем.
	\end{definition}
	\paragraph{Плюсы смешанной системы.}
	\begin{enumerate}
		\item Сбалансированность: Сочетает в себе преимущества обеих систем.
		\item Связь с избирателями: Депутаты, избранные по округам, представляют интересы конкретных территорий и должны работать с запросами своих избирателей.
		\item Представительство партий: Пропорциональная часть позволяет отразить в парламенте весь спектр политических предпочтений общества, давая шанс малым партиям.
		\item Стабильность и разнообразие: Система может способствовать как формированию стабильного большинства (за счет мажоритарной части), так и плюрализму мнений (за счет пропорциональной).
	\end{enumerate}
	\paragraph{Минусы смешанной системы.}
	\begin{enumerate}
		\item Сложность для избирателя: Гражданину бывает сложно разобраться в двойном механизме голосования.
		\item Риск конфликта интересов: В парламенте могут возникнуть две ``касты'' депутатов: ``окружники'' и ``списочники'', которые могут по-разному понимать свою ответственность.
		\item В несвязанной системе: Результаты могут быть непропорциональными. Крупная партия, выиграв много округов, может получить и так большое количество мест по спискам, что ее общее представительство будет превышать ее реальную поддержку в стране.
		\item Манипуляции: Партии могут использовать ``технических'' кандидатов в округах или сложные схемы формирования списков для увеличения своего представительства.
	\end{enumerate}
	\subsection{Политическая власть.}
	\begin{definition}[Власть]
		Способность осуществлять свою волю в отношении кого-то или чего-то. Власть это всегда отношение двух сторон: субъекта (того, кто осуществляет власть) и объекта (того, на кого власть направлена).
	\end{definition}
	\subsubsection{Признаки политической власти.}
	\begin{enumerate}
		\item Верховенство власти: ее решение обязательно для всех.
		\item Легальность (законность): политическая власть должна быть законна.
		\item Централизованность (моноцентричность): существование общего государственного центра принятия решений.
		\item Многообразие ресурсов.
		\item Публичность: она действует от имени всего общества.
	\end{enumerate}
	\subsubsection{Природа власти.}
	\begin{enumerate}
		\item Теологическая теория: власть дана от бога, всякий правитель, осуществляющий ее, осуществляет волю бога.
		\item Биологическая теория: стремление к власти~--- проявление инстинкта выживания, который характерен для всего живого.
		\item Элитарная теория: происхождение политической власти из общественного неравенства.
		\item Классовая теория (Марксистская концепция): связывает власть с отношениями собственности, те политическая власть с точки зрения этой теории~--- порождение экономической власти.
		\item Системная теория: понимание общества как сложной саморегулируемой системы, которая стремиться к своей стабильности, и власть обеспечивает стабильность.
	\end{enumerate}
	\subsubsection{Власть по способам осуществления власти.}
	\begin{enumerate}
		\item Тоталитарная: насилие.
		\item Авторитарная: насилие и убеждение.
		\item Демократическая: убеждение.
	\end{enumerate}
	\subsubsection{Власть по субъекту.}
	\begin{enumerate}
		\item Автократия: неограниченное и бесконтрольное правление одного человека.
		\item Аристократия: власть ``лучших''.
		\item Меритократии: власть ``достойных''.
		\item Олигополия: власть ``немногих''.
		\item Демократия: власть народа.
	\end{enumerate}
	\subsubsection{Структура политической власти.}
	\begin{itemize}
		\item Субъект: то, что совершает власть.
		\item Объект: то, на что власть направлена.
		\item Ресурсы: то, посредствам чего власть достигает своих целей. Ресурсы бывают:
		\begin{itemize}
			\item Экономические: материальные ценности и блага.
			\item Социальные: возможность воздействовать на человека повышая или понижая его социальный статус.
			\item Информационные: информация и средства ее распространения.
			\item Силовые: средства принуждения и специально обученные и подготовленные люди для применения этих средств.
		\end{itemize}
		\item Методы: поощрение, наказание, убеждение и принуждение. Самый хороший метод~--- убеждение.
	\end{itemize}
	\subsubsection{Легальность и легитимность власти.}
	\begin{definition}[Легальность]
		Действие по четко прописанным законам.
	\end{definition}
	\begin{definition}[Легитимность]
		Доверие народа к власти.
	\end{definition}
	\paragraph{Три типа легитимности власти.}
	\begin{enumerate}
		\item Легальная легитимность: в этом случае легальность и легитимность совпадают; главным мотивом подчинения власти является интерес. В политической системе, построенной на легальной легитимности, люди подчиняются не какой-то конкретной личности, а подчиняются законам, при чем подчиняются все: и общество, и сама власть.
		\item Традиционная легитимность: основана не на интересе, а на вере; люди верят, что существующие традиции и порядки являются правильными и справедливыми. Роль закона очень не велика.
		\item Харизматическая легитимность: особое политическое обаяние, которое заставляет людей следовать за данным человеком.
		\item Персональная легитимность: основана на вере людей в личные качества лидеров.
		\item Идеологическая: основана на религии.
	\end{enumerate}
	\section{Идеология.}
	\begin{definition}
		Идеология~--- система взглядов, представлений людей, выражающих интересы того или иного общества, или той или иной социальной группы.
	\end{definition}
	\begin{person}[Антуан Дестют де Траси]
		Создатель слова ``идеология'', лидер движения ``идеологов''.
	\end{person}
	\subsection{Идеи Маркса.}
	\begin{itemize}
		\item Буржуазная идеология (болтовня).
		\item Идеология пролетариата (научная и хорошая).
	\end{itemize}
	\begin{note}
		После него идеологию стали наполнять политическим смыслом.
	\end{note}
	\subsection{Функции идеологий.}
	\begin{enumerate}
		\item Ориентационная.
		\item Мобилизационная: привидение в людей действие.
		\item Интеграционная: идеология способна объединить людей.
		\item Социально-классовая: идеология порождает и защищает идеи определенных социальных групп.
	\end{enumerate}
	\subsection{Идеологии по признакам.}
	\subsubsection{По шкале политического спектра.}
	\begin{enumerate}
		\item Оценка роли государства~--- правая.
		\item Роль общества должна повышаться, роль государства должна уменьшаться~--- левая.
	\end{enumerate}
	\subsubsection{По глубине преобразований в обществе.}
	\begin{enumerate}
		\item Радикальные: резкое изменение, разрушение.
		\item Умеренные: постепенное улучшение жизни общества путем осторожных реформ.
		\item Консерваторы: не допускаются никакие изменения.
	\end{enumerate}
	\subsection{Три основные идеологии современности.}
	\subsubsection{Либерализм.}
	\paragraph{Классический либерализм.}
	\begin{note}
		Основу классического либерализма заложили такие мыслители, как Адам Смит, Вольтер, Монтескье, Франклин, Джеферсон. Право на жизнь, свободу, личную неприкосновенность, собственность. Принцип индивидуализма, реализуемый в социальный, экономической, политической и духовной сферах. В социальной сфере: не может быть преимущества у одних людей, над другими, все люди равны, <<Свобода, равенство, братство!>> В экономической сфере: каждый в праве владеть собственностью, преумножать, распоряжаться по своей воле; идея свободного рынка и ничем не ограниченной конкуренции. В политической сфере была сформулирована идея народа как источника власти, все могут учавствовать в совершении власти; также либерализм ограничивает возможность государственного вмешательства в частную жизнь людей. В духовной сфере либерализм проводит политику культурного легитимизма: все культуры являются разными, но равными.
	\end{note}
	\paragraph{Нео-либерализм.}
	\begin{definition}[Нео-либерализм.]
		Необходимость социальной поддержки государства, но сохранения главного посыла либерализма о необходимости соблюдать вечные свободы и права человека.
	\end{definition}
\end{document}
