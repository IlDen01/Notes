\documentclass{article}

\input{C:/Users/ilden/Documents/School/TeX памятка/header.tex}

\begin{document}
	\tableofcontents
	\setcounter{tocdepth}{4}
	\newpage
	\section{Политология.}
	\begin{definition}
		Политология~--- (с греческого) искусство управления полисом (городом). Человек не участвующий в жизни полиса~--- идиот.
	\end{definition}
	\noindent
	Политика строится вокруг государства.
	\subsection{Государство.}
	\begin{definition}
		Государство~--- политический институт с научной точки зрения; политическая организация обладающая властью на определенной территории. С бытовой~--- страна, народ страны.
	\end{definition}
	\begin{person}[Никколо Макиавелли]
		Флорентийский (Итальянский) мыслитель $16$ века, впервые заговоривший о государстве.
	\end{person}
	\subsubsection{Признаки государства.}
	\paragraph{Основные признаки государства.}
	\begin{itemize}
		\item Суверинетет:
		\begin{itemize}
			\item Внутренний~--- полнота власти на своей территории; у государства есть право на легальное применение силы.
			\item Внешний~--- независимость от других государств.
		\end{itemize}
		У государства должен быть и внутренний и внешний суверенитет, иначе это не государства.
		\item Территория. У государства обязательно должна быть своя территория, иначе тяжело управлять государством, сбором налогов и тому подобное.
		\item Наличие государственной власти. Власть состоит из двух частей:
		\begin{itemize}
			\item Аппарат управления.
			\item Аппарат принуждения.
		\end{itemize}
		\item Регулярный сбор налогов.
		\item Законодательная деятельность.
	\end{itemize}
	\paragraph{Дополнительные признаки.}
	\begin{itemize}
		\item Наличие собственной государственной символики (герб, гимн, флаг, особые атрибуты).
		\item Государственный язык.
		\item Гражданство.
		\item Обладание определенными материальными ресурсами.
	\end{itemize}
	\subsubsection{Функции государства.}
	\paragraph{Внутренние функции государства.}
	\begin{itemize}
		\item Правовая~--- обеспечение правопорядком, законности, борьба с преступностью.
		\item Экономическая~--- организация хозяйственной жизни страны.
		\item Социальная~--- выполнение обязательств перед обществом (система здравоохранения, система образования, поддержка материнства и детства, помощь малоимущим, выплата пенсий и так далее).
		\item Образовательная (иногда выделяется из социальной).
		\item Экологическая функция.
	\end{itemize}
	\paragraph{Внешние функции государства.}
	\begin{itemize}
		\item Оборонная~--- защита границ и противодействие внешним угрозам.
		\item Внешне-политическая (дипломатическа).
		\item Международное сотрудничество.
	\end{itemize}
	\subsection{Политический режим.}
	\begin{definition}
		Политический режим~--- то, как совершается управление государством.
	\end{definition}
	\subsubsection{Основные типы политических режимов.}
	\begin{enumerate}
		\item Демократический~--- власть в основном договаривается с обществом.
		\item Тоталитарный~--- метод подчинения, навязывания, насилия.
		\item Авторитарный~--- комбинация демократического и тоталитарного режимов.
	\end{enumerate}
	\subsubsection{Тоталитарный режим.}
	\begin{definition}
		От латинского totalize~--- полный; вся полнота власти над обществом сосредоточена в руках государства, отсутствие предела контроля.
	\end{definition}
	\paragraph{Признаки тоталитаризма.}
	\begin{itemize}
		\item Наличие единственной господствующей идеологии, остальные являются враждебными и преследуются.
		\item Однопартийность.
		\item Экономика подчинена государству, свободного рынка нет.
		\item Опора на насилие, массовый террор.
	\end{itemize}
	\subsubsection{Демократический режим.}
	\begin{definition}
		Противоположность тоталитаризма. Государство существует для народа. Основной источник власти~--- народ.
	\end{definition}
	\paragraph{Признаки демократии.}
	\begin{itemize}
		\item Многопартийность.
		\item Свободный рынок.
	\end{itemize}
	\paragraph{Типы демократии.}
	\begin{itemize}
		\item Прямая~--- граждане принимают непосредственное участие в управлении государством. Референдум~--- всенародное голосование по особо важным вопросом; пример современной прямой демократии. Особенность в том, что референдум не имеет обязательной юридической силы.
		\item Представительная~--- выбор представителей в органы государственной власти.
	\end{itemize}
	\subsubsection{Авторитарный режим.}
	\begin{definition}
		Авторитарный режим~--- тип не демократического политического режима, основанного на несменяемой централизованной власти одного лица или группы лиц при сохранении в стране определенного уровня экономических, гражданских и идейных свобод; осуществляется полный контроль, зачастую, только власти. Самый распространенный тип за всю историю.
	\end{definition}
\end{document}
