\documentclass[12pt]{article}

\input{C:/Users/ilden/Documents/School/TeX памятка/header.tex}

\begin{document}
	\tableofcontents
	\setcounter{tocdepth}{4}
	\newpage
	\section{Политология.}
	\begin{definition}
		Политология~--- (с греческого) искусство управления полисом (городом). Человек не участвующий в жизни полиса~--- идиот.
	\end{definition}
	\noindent
	Политика строится вокруг государства.
	\subsection{Государство.}
	\begin{definition}
		Государство~--- политический институт с научной точки зрения; политическая организация обладающая властью на определенной территории. С бытовой~--- страна, народ страны.
	\end{definition}
	\begin{person}[Никколо Макиавелли]
		Флорентийский (Итальянский) мыслитель $16$ века, впервые заговоривший о государстве.
	\end{person}
	\subsubsection{Признаки государства.}
	\paragraph{Основные признаки государства.}
	\begin{itemize}
		\item Суверинетет:
		\begin{itemize}
			\item Внутренний~--- полнота власти на своей территории; у государства есть право на легальное применение силы.
			\item Внешний~--- независимость от других государств.
		\end{itemize}
		У государства должен быть и внутренний и внешний суверенитет, иначе это не государство.
		\item Территория. У государства обязательно должна быть своя территория, иначе тяжело управлять государством, сбором налогов и тому подобное.
		\item Наличие государственной власти. Власть состоит из двух частей:
		\begin{itemize}
			\item Аппарат управления.
			\item Аппарат принуждения.
		\end{itemize}
		\item Регулярный сбор налогов.
		\item Законодательная деятельность.
	\end{itemize}
	\paragraph{Дополнительные признаки.}
	\begin{itemize}
		\item Наличие собственной государственной символики (герб, гимн, флаг, особые атрибуты).
		\item Государственный язык.
		\item Гражданство.
		\item Обладание определенными материальными ресурсами.
	\end{itemize}
	\subsubsection{Функции государства.}
	\paragraph{Внутренние функции государства.}
	\begin{itemize}
		\item Правовая~--- обеспечение правопорядком, законности, борьба с преступностью.
		\item Экономическая~--- организация хозяйственной жизни страны.
		\item Социальная~--- выполнение обязательств перед обществом (система здравоохранения, система образования, поддержка материнства и детства, помощь малоимущим, выплата пенсий и так далее).
		\item Образовательная (иногда выделяется из социальной).
		\item Экологическая функция.
	\end{itemize}
	\paragraph{Внешние функции государства.}
	\begin{itemize}
		\item Оборонная~--- защита границ и противодействие внешним угрозам.
		\item Внешне-политическая (дипломатическая).
		\item Международное сотрудничество.
	\end{itemize}
	\subsection{Политический режим.}
	\begin{definition}
		Политический режим~--- то, как совершается управление государством.
	\end{definition}
	\subsubsection{Основные типы политических режимов.}
	\begin{enumerate}
		\item Демократический~--- власть в основном договаривается с обществом.
		\item Тоталитарный~--- метод подчинения, навязывания, насилия.
		\item Авторитарный~--- комбинация демократического и тоталитарного режимов.
	\end{enumerate}
	\subsubsection{Тоталитарный режим.}
	\begin{definition}
		От латинского totalize~--- полный; вся полнота власти над обществом сосредоточена в руках государства, отсутствие предела контроля.
	\end{definition}
	\paragraph{Признаки тоталитаризма.}
	\begin{itemize}
		\item Наличие единственной господствующей идеологии, остальные являются враждебными и преследуются.
		\item Однопартийность.
		\item Экономика подчинена государству, свободного рынка нет.
		\item Опора на насилие, массовый террор.
	\end{itemize}
	\subsubsection{Демократический режим.}
	\begin{definition}
		Противоположность тоталитаризма. Государство существует для народа. Основной источник власти~--- народ.
	\end{definition}
	\paragraph{Признаки демократии.}
	\begin{itemize}
		\item Многопартийность.
		\item Свободный рынок.
	\end{itemize}
	\paragraph{Типы демократии.}
	\begin{itemize}
		\item Прямая~--- граждане принимают непосредственное участие в управлении государством. Референдум~--- всенародное голосование по особо важным вопросом; пример современной прямой демократии. Особенность в том, что референдум не имеет обязательной юридической силы, в отличии от плебисцита.
		\item Представительная~--- выбор представителей в органы государственной власти.
	\end{itemize}
	\subsubsection{Авторитарный режим.}
	\begin{definition}
		Авторитарный режим~--- тип не демократического политического режима, основанного на несменяемой централизованной власти одного лица или группы лиц при сохранении в стране определенного уровня экономических, гражданских и идейных свобод; осуществляется полный контроль, зачастую, только власти. Самый распространенный тип за всю историю.
	\end{definition}
	\subsection{Форма государства.}
	\begin{definition}
		Форма государства~--- особенность внутренней организации государства, которая охватывает форму правления, государственное устройство и политический режим.
	\end{definition}
	\subsubsection{Формы государства.}
	\begin{enumerate}
		\item Монархия.
		\item Республика.
	\end{enumerate}
	\subsubsection{Монархия.}
	\begin{definition}[Монархия]
		Власть принадлежит одному лицу.
	\end{definition}
	\paragraph{Признаки монархии.}
	\begin{itemize}
		\item Власть передается по наследству.
		\item Власть бессрочна.
		\item Власть неподвластна населению.
	\end{itemize}
	\begin{note}
		Самая древняя форма правления.
	\end{note}
	\paragraph{Древние формы монархий.}
	\begin{enumerate}
		\item Деспотия~--- власть правителя схожа с властью хозяина над рабами. Примеры: государства древнего Востока, древний Египет, древний Китай.
		\item Теократия~--- монарх представляется как живой бог (или наместник бога).
		\item Абсолютная монархия~--- вся власть (законодательная, исполнительная, судебная, военная и так далее) сконцентрирована в руках одного человека. Примеры: государства Ближнего Востока, Франция при Людовике XIV.
		\item Просвещенная монархия~--- предполагает определенные обязанности государственной власти. Монарх~--- не просто единоличный правитель, а образованный, просвещенный государь, который обязан заботится о просвещении своего народа и благе государства. Пример: Россия при Екатерине Великой. Важный шаг к демократии.
	\end{enumerate}
	\paragraph{Современные формы монархий.}
	\begin{enumerate}
		\item Абсолютная монархия~--- вся полнота власти в руках одного человека.
		\item Конституционная монархия~--- полномочия монарха ограничены конституцией.
		\begin{enumerate}
			\item Дуалистическая монархия~--- у монарха обширная власть. У парламента мало власти.
			\item Парламентская монархия~--- у монарха минимальная власть. По большей части функции монарха~--- символические. Законы принимает парламент. В некоторых случаях у монарха есть право вето, но этим право они не пользуются.
		\end{enumerate}
	\end{enumerate}
	\subsubsection{Республика.}
	\begin{definition}[Республика]
		Законодательная власть принадлежит выборному представительному органу (парламенту). Глава государства (президент) либо избирается всем населением, либо специальной избирательной комиссией.
	\end{definition}
	\paragraph{Признаки республики.}
	\begin{enumerate}
		\item Выборность власти.
		\item Срочность власти.
		\item Зависимость от воли избирателей.
	\end{enumerate}
	\paragraph{Типы республики.}
	\begin{enumerate}
		\item Президентская. Президент~--- глава исполнительной власти и государства. Избирается, как правило, прямым голосованием всех граждан. У президента большая власть и большой объем полномочий, но несмотря на все это он~--- не монарх, и его возможности ограничены парламентом.
		\item Парламентская~--- правительство формируется парламентом, и перед парламентом же несет ответственность. Парламент может выразить недовольство правительству или отдельным министрам, что может спровоцировать правительственный кризис. Глава государства~--- тоже президент, но полномочия весьма скромные, в отличии от полномочий в президентской республике; в основном представительные функции (мало чем отличаются от функций монарха в парламентской монархии).
		\item Смешанная~--- высшие властные полномочия распределенны между президентом и парламентом.
	\end{enumerate}
	\subsection{Формы государственного устройства.}
	\begin{definition}
		Государственное устройство~--- способ взаимодействия частей государства.
	\end{definition}
	\begin{enumerate}
		\item Унитарное государство~--- части такого государства подчинены центру. Единая конституция, единое право. Примеры: Великобритания, Польша, Финляндия, Китай.
		\item Федерация~--- объединение, союз. Части федерации обладают признаками суверенитет (имеют право на собственную конституцию, законодательство, судебную систему и тому подобное). Два уровня власти: федеральная (высшая) и власть субъектов. Примеры: США, Германия, Индия, Канада, Российская Федерация.
		\item Конфедерация~--- не государство, а объединение разных, абсолютно независимых политических субъектов (возможно государств). Примеры: Речь Посполитая.
		\item Содружество государств~--- новоя форма, которая частично заменили федерацию. Примеры: ЕС, СНГ, Британское содружество наций.
	\end{enumerate}
	\subsection{Политические партии и организации.}
	\begin{definition}
		Партия~--- группа лиц, которых объединяют общие интересы.
	\end{definition}
	\begin{definition}
		Политическая партия~--- политическая организация, которая выражает интересы определенных групп и слоев населения, которая борется за власть и осуществляет государственную власть в своих интересах.
	\end{definition}
	\begin{note}
		Первые полит кружки появились в начале 19 века~--- кружки декабристов. Но они не были партиями, так как не боролись за власть.
	\end{note}
	\subsubsection{Функции политических партий.}
	\begin{enumerate}
		\item Организационная (завлечение людей в партию).
		\item Представительная (выражение и представление определенных слоев).
		\item Идеологическая (пропаганда своих идей).
	\end{enumerate}
	\subsubsection{Признаки политических партий.}
	\begin{enumerate}
		\item Стремление к завоеванию власти.
		\item Наличие идеологии.
		\item Устойчивость.
		\item Наличие организации (устав, программа и членство). Программа~--- что партия будет делать, когда придет к власти. Устав~--- документ. в котором прописаны нормы внутрипартийной жизни. Членство~--- партия структурно состоит из руководства (люди, профессионально занятые политикой), передовых людей и социальной базы.
	\end{enumerate}
	\subsubsection{Виды политических партий.}
	\begin{enumerate}
		\item По идеологическому признаку.
		\begin{itemize}
			\item Либерализм. Свобода и права человека. Появились в $17 - 18$ веке, раньше консерваторов. \\
			Консерватизм. Сохранение старого и традиционных ценностей; людей надо научить пользоваться свободой, и лучше всего их этому научит государство. \\
			Обе идеологии выражали интересы собственников.
			\item Социал-демократизм. За социальную справедливость. чем беднее человек, тем меньше он платит налогов; социальная политика. Но при всем этом сохранение частной собственности. \\
			Коммунизм. Частная собственность~--- основа для эксплуатации человека человеком, поэтому ее надо ликвидировать. \\
			Выражали интересы трудящихся, в первую очередь рабочего класса.
			\item Националистические партии.
			\begin{itemize}
				\item Умеренный национализм. Регулирование потоков миграции.
				\item Радикальный национализм. Отмена миграции, этнические чистки. Такие националистические партии под запретом.
			\end{itemize}
		\end{itemize}
		\item По шкале политического спектра.
		\begin{itemize}
			\item Правые. Те, кто отстаивает силы государства, охраняющие частную собственность.
			\item Центристские. За компромис, соединение государственных и общественных интересов.
			\item Левые. Те, кто больше ориентирован на общество и общественные интересы.
		\end{itemize}
		\item По отношению к другим элементам политической системы.
		\begin{itemize}
			\item Демократические. Терпимость по отношению к другим партия, сотрудничество и ведения диалогов с ними.
			\item Не демократические. Не намерены вести диалоги, стремление подчинить себе другие партии, институты, государство.
		\end{itemize}
		\item По положению в отношении правления.
		\begin{itemize}
			\item Правящие. Те, кто победили на выборах и правят.
			\item Оппозиция. Те, кто не у власти. Как правило критикуют правящую партию, ждут их ошибки и пользуются этим, чтобы в будущем самим прийти к власти.
			\begin{itemize}
				\item Системная оппозиция. Представлена в парламенте.
				\item Несистемная оппозиция. Легальная в парламенте не представлена, но и не запрещена. Нелегальная~--- законом запрещена.
			\end{itemize}
		\end{itemize}
		\item По численности.
		\begin{itemize}
			\item Массовые. Заинтересована в максимальном количестве своих членов. Главный источник дохода таких партий~--- членские взносы.
			\item Кадровые. Имеют другое финансирование. Существуют за счет финансовой помощи спонсоров.
		\end{itemize}
	\end{enumerate}
	\subsubsection{Политические движения.}
	\begin{definition}
		Политические движения~--- отличаются от партий тем, что не участвуют в борьбе за власть. Также отсутствуют членство и организационное единство. В политической жизни участвуют, пытаясь воздействовать на власть и заявить о себе.
	\end{definition}
	\subsection{Политические системы и их виды.}
	\begin{definition}
		Политическая система~--- система взаимодействия между партиями.
	\end{definition}
	\subsubsection{Виды политических систем.}
	\begin{itemize}
		\item Однопартийная. Характерна для стран с не демократическими политическими режимами. При этом может сохраняться видимость многопартийности. Таким образом однопартийные политические системы делятся на:
		\begin{itemize}
			\item Реально однопартийные. Пример: СССР (партия КПСС).
			\item Формально многопартийные. Не смотря на существование нескольких партий власть контролируется одной партией. Пример: Китай, Северная Корея.
		\end{itemize}
		\item Двухпартийная. Характерна для стабильных, устоявшихся политических режимов. Пример: США, Великобритания, Австралия. Каждая из двух партия способна прийти к власти и сформировать правительство. При этом партий может быть больше. Причины двухпартийности:
		\begin{itemize}
			\item Бипартизм всегда имеет исторические корни.
			\item Психологическое значение.
		\end{itemize}
		Сильная сторона двухпартийной системы~--- обеспечение стабильности и порядка (одна партия правит, другая ее критикует, а потом они меняются местами). Но слабая сторона~--- неспособность такой системы выразить все многообразие мнений и взглядов народа.
		\item Многопартийная. Характеризует ситуаций когда партий не просто много/мало, а когда не одна из существующих партий не располагает достаточной поддержкой избирателей, чтобы самостоятельно прийти к власти. Для того, чтобы организовать правительство партиям необходимо объединяться и вступать в коалиции. Преимущества такой системы в том, что она лучше отображает интересы разных социальных групп. Но ее главный недостаток в том, что она не обеспечивает такой уровень стабильности как двухпартийная система.
		\item Двух с половиной партийная. Ни одна из соперничающих партий оказывается не в состоянии победить на выборах, набрать необходимое для организации правительства голосов. В такой ситуации несколько процентов голосов, которые может им дать какая-нибудь малозначительная партия, принесут им победу; и тогда эти партии объединяются и маленькая партия получает большое политическое значение.
		\item С доминирующей партией. Суть в том, что одна и та же партия в течении длительного времени побеждает всех своих конкурентов и находится у власти. Это может длиться годами, десятилетиями. Например либерально-демократическая партия Японии находилась у власти без перерыва $38$ лет, причем в стране не однопартийная система. Похожая ситуация была в Индии. Причин может быть несколько: люди просто искренне поддерживают партию, неосознанность населения ("лишь бы не стало хуже, поэтому менять ничего не будем").
	\end{itemize}
	\subsection{Разбор политики на примерах.}
	\subsubsection{Великобритания.}
	\begin{note}
		В Великобритании парламентская монархия. Король/королева несет только символическую роль. Основная власть принадлежит премьер-министру. Зачастую он~--- представитель партии, победившей на выборах в нижней палате. Нижняя палата является главной в английском парламенте, так как она общая и избирается на $5$ лет народным голосованием. Также есть верхняя палата Лордов, которая не избирается. До недавнего времени она формировалась по наследственному признаку. Сейчас же наследственным путем формируется $60\%$ палаты (наследственные пэры), а остальные получают место в палате за выдающиеся достижения (пожизненные пэры). Во главе верховной палаты~--- лорд-канцлер, который назначается монархом на 5 лет.
	\end{note}
	\begin{note}
		По форме государственного устройства Великобритания~--- унитарное государство, которое состоит из Англии, Уэльса, Шотландия и Северная Ирландия. Хотя все ее части претендуют на самостоятельность.
	\end{note}
	\begin{note}
		В Великобритании двухпартийная система. Партии~--- лейбористская и консервативная.
	\end{note}
\end{document}
