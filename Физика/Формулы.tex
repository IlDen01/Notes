\documentclass{article}

\input{/home/ilden/Documents/School/TeX памятка/header.tex}

\begin{document}
	\noindent
	\textbf{Плотность.} $\rho = \frac{m}{V}$. $[\rho] = \frac{\text{кг}}{\text{м}^3}$. \\
	\textbf{Вес.} $P = mg$. $[P] =$ Н. \\
	\textbf{Давление.} $p = \frac{F}{S}$. $[p] =$ Па. \\
	\textbf{Давление столба жидкости.} $p = \rho g h$. \\
	\textbf{Сила Архимеда.} $F_{\text{арх}} = \rho g V$. \\
	\textbf{Скорость.} $V = \frac{S}{t}$. $[V] = \frac{\text{м}}{\text{с}}$. \\
	\textbf{Ускорение.} $a = \frac{\varDelta V}{\varDelta t}$. $[a] = \frac{\text{м}}{\text{с}^2}$. \\
	\textbf{Формулы с ускорением:}
	\begin{itemize}
		\item $V_x = V_{0x} + a_xt$.
		\item $S_x = V_{0x}t \pm \frac{a_xt^2}{2}$.
		\item $x = x_0 + V_{0x}t + \frac{a_xt^2}{2}$.
	\end{itemize}
	\textbf{Сила трения.} $F_{\text{тр}} = N \mu$. \\
	\textbf{Закон Гука.} $F_{\text{упр}} = -k \varDelta x$. \\
	\textbf{Параллельное соединение пружин.} $k_{\text{об}} = k_1 + k_2 + \dots$. \\
	\textbf{Последовательное соединение пружин.} $\frac{1}{k_{\text{об}}} = \frac{1}{k_1} + \frac{1}{k_2} + \dots$. \\
	\textbf{Коэффиицент полезного действия.} $\eta = \frac{A_{\text{пол}}}{A_{\text{зат}}}$. \\
	\textbf{Момент.} $Fl$. \\
	\textbf{Кинетическая энергия.} $E_{\text{к}} = \frac{mV^2}{2}$. \\
	\textbf{Потенциальная энергия.} $E_{\text{п}} = mgh$. \\
	\textbf{Потенциальная энергия пружины.} $E_{\text{п}} = -\frac{k \varDelta x^2}{2}$. \\
	\textbf{Внутренняя энергия.} $\sum E_{\text{к. мол.}} + E_{\text{п. взаим.}}$. \\
	\textbf{Количество теплоты через теплоемкость.} $Q = C \varDelta t$. \\
	\textbf{Количество теплоты через удельную теплоемкость.} $Q = c m \varDelta t$. \\
	\textbf{Закон Ньютона-Рихмана.} $P = \alpha (t_{\text{тела}} - t_{\text{окр}})$. \\
	\textbf{Абсолютная влажность воздуха.} $\rho_{\text{абс}} = \frac{m_{H_2O}}{V}$. \\
	\textbf{Относительная влажность воздуха.} $\varphi = \frac{\rho_{\text{абс}}}{\rho_{\text{нп}(t)}} \cdot 100\%$. \\
	\textbf{Закон Фурье.} $P = \frac{\alpha (t_1 - t_2)}{l}$. \\
	\textbf{Закон Кулона.} $F = \frac{k \cdot |q_{1} \cdot q_{2}|}{\varepsilon \cdot R^{2}}$. $k = 9 \cdot 10^{9} \frac{\text{Н} \cdot \text{м}^{2}}{\text{Кл}^{2}}$, $\varepsilon$ - диэлектрическая проницаемость(в вакууме 1). \\
	\textbf{Напряженность.} $E = \frac{F}{q} = \frac{k \cdot q}{r^{2}}$. $[E] = \frac{\text{В}}{\text{м}} = \frac{\text{Н}}{\text{Кл}}$. \\
	\textbf{Потенциальная энергия в электрическом поле, действующий на точку.} $W = q \varphi$. $[\varphi] =$ В. \\
	\textbf{Напряжение.} $U = \varphi_{1} - \varphi_{2} = I \cdot R = \frac{A}{q}$. $[U] = \text{В}$. \\
	\textbf{Сила тока.} $I = \frac{q}{t} = \frac{U}{R}$. $[I] = A = \frac{\text{Кл}}{\text{с}}$. \\
	\textbf{Сопротивление.} $R = \frac{U}{I} = \frac{\rho \cdot l}{S}$. $[R] = \frac{\text{В}}{\text{А}} = \text{Ом}$. \\
	\textbf{Закон Ома.} $I \sim U$; $I = \frac{U}{R}$. \\
	\textbf{Последовательное соединение резисторов.} $I_{\text{об}} = I_{1} = I_{2} = \dots$; $U_{\text{об}} = U_{1} + U_{2} + \dots$; $R_{\text{об}} = R_{1} + R_{2} + \dots$. \\
	\textbf{Параллельное соединение резисторов.} $I_{\text{об}} = I_{1} + I_{2} + \dots$; $U_{\text{об}} = U_{1} = U_{2} = \dots$; $\frac{1}{R_{\text{об}}} = \frac{1}{R_{1}} + \frac{1}{R_{2}} + \dots$. \\
	\textbf{Закон Джоуля-Ленца.} $Q = I^2Rt = \frac{U^2t}{R} = IUt$. \\
	\textbf{Мощность электрического тока.} $P = I^2R = \frac{U^2}{R} = IU$. \\
	\textbf{ЭДС(Электро-движущая сила).} $\varepsilon = \frac{A_{\text{ст}}}{q}$. $[\varepsilon] = \text{В}$. \\
	\textbf{Закон Ома для участка цепи с источником.} $\Phi_{A} - \Phi_{B} + \varepsilon = I \cdot (R + r)$. \\
	\textbf{Законы Кирхгофа:}
	\begin{enumerate}
		\item $\sum \limits_{i} \pm I_{i} = 0$.
		\item $\sum \limits_{i} \pm \varepsilon_{i} = \sum \limits_{i} \pm I_{i} \cdot R_{i} + \sum \limits_{i} \pm I_{i} \cdot r_{i}$.
	\end{enumerate}
	\textbf{Шунты:}
	\begin{itemize}
		\item Амперметр. $R = \frac{R_{A}}{n - 1}$.
		\item Вольтметр. $R = (n - 1) \cdot R_{V}$.
	\end{itemize}
	\textbf{Емкость конденсатора.} $c = \frac{q}{U} = \frac{\varepsilon_{0} \cdot \varepsilon \cdot S}{d}$. $[c] = \frac{\text{Кл}}{\text{В}} = \text{Ф}$; $\varepsilon_{0}$ - электрическая постоянная; $\varepsilon$ - диэлектрическая проницаемость, величина, которая показывает во сколько раз диэлектрик ослабевает электрическое поле. $\varepsilon_{0} = \frac{1}{4 \cdot \pi \cdot k} = 8.85 \cdot 10^{-12} \frac{\text{Ф}}{\text{м}}$. \\
	\textbf{Сила Ампера.} $F_{A} = B \cdot I \cdot l \cdot \sin\alpha$. $\alpha$ - угол между линиями индукции магнитного поля и направлением тока. \\
	\textbf{Сила Лоренца.} $F_{\text{Л}} = B \cdot q \cdot v \cdot \sin\alpha$. $\alpha$ - угол между линиями индукции магнитного поля и направлением скорости заряда. \\
	\textbf{Поток вектора магнитной индукции.} $\text{Ф}_{\text{в}} = BS \cos\alpha$. $[\text{Ф}_{\text{в}}] =$ Вб. \\
	\textbf{Индукция магнитного поля.} $B = \frac{F_{max}}{I \cdot l}$. $[B] = \text{Тл}$. \\
	\textbf{Закон радиоактивного распада.} $N = \frac{N_{0}}{2^{\frac{t}{T}}}$. $T$ - время полураспада, $N_{0}$ - изначальное число атомов, $t$ - прошедшее время. \\
	\textbf{Дефект масс.} $\varDelta m = M_{\text{п}} + M_{\text{н}} - M_{\text{я}}$. \\
	\textbf{Формула фокусного расстояния линз.} $\pm \frac{1}{F} = \pm \frac{1}{d} \pm \frac{1}{f}$; \\
	$F$ --- фокусное расстояние, $d$ --- расстояние от объекта до линзы, $f$ --- расстояние от изображения до линзы. \\
	$\pm$ перед $\frac{1}{F}$ --- собирающая/рассеивающая линза, $\pm$ перед $\frac{1}{d}$ --- действительный/мнимый предмет, $\pm$ перед $\frac{1}{f}$ --- действительное/ мнимое изображение. \\
	\textbf{Диоптрия.} $D = \frac{1}{F}$. $[D] =$ Дптр. $D_{\text{об}} = D_1 + D_2 + \dots$. \\
	\textbf{Нормальное ускорение.} $a_{\text{н}} = \frac{V^2}{R}$. \\
	\textbf{Углова скорость.} $\omega = \lim_{\varDelta t \rightarrow 0} \frac{\varDelta \varphi}{\varDelta t}$. $[\omega] = \frac{\text{рад}}{\text{с}}$. \\
	\textbf{Период.} $T = \frac{2 \pi R}{V} = \frac{2 \pi}{\omega}$. $[T] =$ с. \\
	\textbf{Формула связи линейной скорости с угловой.} $V = \omega R$. \\
	\textbf{Частота.} $\nu = \frac{1}{T}$. $[\nu] =$ Гц. \\
	\textbf{Преобразование Галилея.} $\vec{V_{\text{абс}}} = \vec{V_{\text{относ}}} + \vec{V_{\text{пер}}}$. \\
	\textbf{Закон Снелиуса.} $n_1 \sin\alpha = n_2 \sin\beta$. \\
	\textbf{Второй закон Ньютона.} $\sum\vec{F} = m\vec{a}$. \\
	\textbf{Механическое напряжение.} $\sigma = \frac{F}{S} = \varepsilon \cdot \frac{kl_0}{S} = E \cdot |\varepsilon|$. $[\sigma] = \frac{\text{Н}}{\text{м}^2} = \text{Па}$. \\
	\textbf{Модуль Юнга}. $E = \frac{kl_0}{S}$. $[E] =$ Па. \\
	\textbf{Закон всемирного тяготения.} $F_{\text{грав}} = \frac{GM_1M_2}{R^2}$. \\
	\textbf{Ускорение свободного падения.} $F = G \frac{Mm}{R^2} \rightarrow G \frac{M}{R^2} = g = 9.8$. $G = 6.67 \cdot 10^{-11} \frac{\text{Н} \cdot \text{м}^2}{\text{кг}^2}$. \\
	\textbf{Сила инерции.} $\vec{F}_{\text{и}} = -m \cdot \vec{a}_{\text{пер}}$. \\
	\textbf{Импульс.} $p = m \cdot V; [p] = \frac{\text{кг} \cdot \text{м}}{\text{с}}$. \\
	\textbf{Второй закон Ньютона в импульсной форме.} $\vec{F} \varDelta t = \varDelta \vec{p} \rightarrow \vec{F} = \frac{\varDelta \vec{p}}{\varDelta t}$. \\
	\textbf{Закон изменения импульса системы.} $\varDelta \vec{p}_{\text{сис}} = \vec{F}_{\text{внеш}} \cdot \varDelta t$. \\
	\textbf{Уравнение Мещерского.} $\vec{F}_{p} = -\mu \vec{u}$. \\
	\textbf{Механическая работа.} $A = Fl \cdot \cos \alpha = \vec{F} \cdot \vec{l}$. $\alpha$~--- угол между силой и вектором перемещения. $[A] =$ Дж. \\
	\textbf{Мощность.} $P = \frac{A}{t} = FV \cdot \cos \alpha = \vec{F} \cdot \vec{V}$. $[P] =$ Вт. \\
	\textbf{Работа силы упругости.} $A = -\varDelta E_{\text{п}} = \frac{k (\varDelta x)^2}{2}$. \\
	\textbf{Потенциальная энергия силы тяготения.}
	$E_{\text{п}} = \frac{GM_1M_2}{R}$. \\
	\textbf{Формула координаты центра масс.} $x_c = \frac{\sum\limits_i m_i x_i}{m} = \frac{\sum\limits_i m_i x_i}{\sum\limits_i m_i}$. $y_c = \frac{\sum\limits_i m_i y_i}{m} = \frac{\sum\limits_i m_i y_i}{\sum\limits_i m_i}$. $z_c = \frac{\sum\limits_i m_i z_i}{m} = \frac{\sum\limits_i m_i z_i}{\sum\limits_i m_i}$. $\vec{r_c} = \frac{\sum\limits_i m_i \vec{r_i}}{m} = \frac{\sum\limits_i m_i \vec{r_i}}{\sum\limits_i m_i}$. \\
	\textbf{КПД.} $\eta = \frac{A_{\text{пол}}}{A_{\text{зат}}} \cdot 100\%$. \\
	\textbf{Теорема о движении центра масс.} $m \vec{a}_c = \vec{F}_{\text{внеш}}$. \\
	\textbf{Основное уравнение динамики вращательного движения.} $I (\text{кг} \cdot \text{м}^2)$ $\cdot$ $\beta (\frac{\text{рад}}{\text{с}^2}) = \sum M (\text{Н} \cdot \text{м})$. \\
	\textbf{Энергия вращательного движения тела.}
	$E = \frac{I\omega^2}{2}$.
\end{document}