\documentclass[12pt]{article}

\input{C:/Users/ilden/Documents/School/TeX-Cheat-Sheet/header.tex}

\begin{document}
	\tableofcontents
	\setcounter{tocdepth}{3}
	\newpage
	\section{Правила дорожного движения и ДТП.}
	\begin{definition}
		ДТП~--- это событие, возникшее в процессе движение по дороге транспортного средства и с его участием, при котором погибли или пострадали люди, повреждены транспортные средства, сооружения, грузы или причинен иной материальный ущерб.
	\end{definition}
	\subsection{Участники дорожного движения.}
	\begin{itemize}
		\item Пешеход.
		\item Водитель.
	\end{itemize}
	\subsection{Фактор риска возникновения ДТП.}
	\begin{itemize}
		\item Нарушение ПДД.
		\begin{itemize}
			\item Превышение скорости.
			\item Проезд на запрещающий свет светофора.
			\item Не соблюдение дистанции.
		\end{itemize}
		\item Погодные условия и время суток.
		\begin{itemize}
			\item Снегопад.
			\item Лед.
			\item Дождь.
			\item Темное время суток.
			\item Яркий солнечный свет.
		\end{itemize}
		\item Состояние дорожного покрытия.
		\item Техническая неисправность транспортного средства.
		\begin{itemize}
			\item Отказ тормозной системы.
			\item Износ шин.
			\item Неисправность рулевого управления.
			\item Проблемы с фарами.
		\end{itemize}
	\end{itemize}
	\subsection{Знаки дорожного движения.}
	\subsubsection{Виды знаков дорожного движения.}
	\begin{itemize}
		\item Предупреждающие знаки. Чаще всего имеют треугольную форму с белым фоном и красной рамкой. Исключения~--- знаки, которые показывают о железнодорожном переезде или направлении поворота.
		\begin{figure}[H]
			\includegraphics[height=0.6\textwidth]{extra-materials/Предупреждающие-знаки-ПДД}
			\caption{Предупреждающие знаки ПДД.}
		\end{figure}
		\item Запрещающие знаки. Имеют круглую форму.
		\begin{figure}[H]
			\includegraphics[height=0.55\textwidth]{extra-materials/Запрещающие-знаки-ПДД}
			\caption{Запрещающие знаки ПДД.}
		\end{figure}
	\end{itemize}
	\section{Аварии с ж/д транспортом.}
	\subsection{Виды аварий на ж/д транспорте.}
	\begin{itemize}
		\item Столкновение ж/д состава с:
		\begin{itemize}
			\item другим ж/д составом;
			\item человеком/животным;
			\item другим транспортом.
		\end{itemize}
		\item Выход вагонов из колеи (сход с рельс).
		\item Возгорание ж/д путей/составов.
	\end{itemize}
	\subsection{Причины.}
	\begin{itemize}
		\item Нарушение правил эксплуатации.
		\item Некачественное ТО.
		\item Внешнее воздействие (природа/терроризм).
	\end{itemize}
	
	\section{АХОВ (аварийно химические опасные вещества).}
	
	\subsection{Способы воздействия АХОВ.}
	
	\begin{itemize}
		\item Дыхательные пути (ингаляционное воздействие).
		\item Кожные покровы (кожно-резорбного типа).
		\item Пищевой тракт (перорального действия).
	\end{itemize}
	
	\subsection{Классификация ахов.}
	
	\begin{itemize}
		\item По степени воздействия на организм.
		
		\begin{xltabular}{\textwidth}{|X|X|}
			\hline
			\textbf{Класс опасности} & \textbf{Наименования АХОВ}\\
			\hline
			Чрезвычайно опасные вещества. & Хлористый водород, фтористый водород.\\
			\hline
			Высоко-опасные. & Фосген, хлор.\\
			\hline
			Средне-опасные. & Азотная кислота.\\
			\hline
			Мало-опасные. & Амиак.\\
			\hline
		\end{xltabular}
		
		\item По основным физико-химическим составам и условиям хранения.
		
		\begin{xltabular}{\textwidth}{|X|X|}
			\hline
			\textbf{Характеристика} & \textbf{Наименование АХОВ}\\
			\hline
			Жидкие и летучие, хранимые под давлением. & Аммиак, хлор.\\
			\hline
			Жидкие и летучие, хранимые без давления. & Ди-фосген.\\
			\hline
			Дымящие кислоты. & Азотная, серная и соляная кислоты.\\
			\hline
			Сыпучие, твердые, не летучие, хранимые до $40^{\circ}$C. & Фосфор.\\
			\hline
			Сыпучие, твердые, летучие, хранимые до $40^{\circ}$C. & Соли синильной кислоты.\\
			\hline
		\end{xltabular}
		
		\item По преимущественному синдрому, складывающемуся при острой интоксикации.
		
		\begin{xltabular}{\textwidth}{|X|X|X|}
			\hline
			\textbf{Группа} & \textbf{Характер воздействия} & \textbf{Наименование АХОВ}\\
			\hline
		\end{xltabular}
		
		\item По способности к горению.
		
		\begin{xltabular}{\textwidth}{|X|X|X|}
			\hline
			\textbf{Характеристика состояния} & \textbf{Наименования АХОВ} & \textbf{Примечания}\\
			\hline
			Негорячие вещества. & Фосген. & Не горит в условиях до $900^{\circ}$C. \\
			\hline
			Не горючие, пожароопасные вещества. & Азотная кислота, фтористый водород, хлор. & Не горит в условиях до $900^{\circ}$C. Распадаются с выделением паров.\\
			\hline
			Трудно-горючие вещества. & Цианистый водород, сжиженный аммиак. & Возгораются только при действии источника огня.\\
			\hline
			Горючие вещества & Газообразный аммиак, серо-углерод. & Способны самовозгореться и гореть, даже после удаления источника огня.\\
			\hline
		\end{xltabular}
	\end{itemize}
	
	\begin{definition}[Аммиак]
		Бесцветный газ, с резким раздражающим запахом нашатырного спирта. В газообразном состоянии легче воздуха. Переходит в жидкое состояние при $-33^{\circ}$C, при $-77^{\circ}$C переходит в твердое. При наличии источника огня хорошо горит. Используется для производства: азотной кислоты, соды, удобрений. Применяется в окрашивание тканей и в холодильниках, в качестве охлаждающей ткани.
		
		$10\%$ раствор является нашатырным спиртов. $18--20\%$ раствор называется аммиачной водой и используется в качестве удобрения.
	\end{definition}
	
	\begin{definition}[Хлор]
		В обычных условиях газ, желто-зеленного цвета, с резким запахом, тяжелее воздуха.
	\end{definition}
\end{document}
