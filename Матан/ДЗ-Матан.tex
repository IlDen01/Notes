\documentclass[12pt]{article}

\input{C:/Users/ilden/Documents/School/TeX-Cheat-Sheet/header.tex}

\begin{document}
	\begin{center}
		\large\bf Домашнее задание
	\end{center}
	\begin{enumerate}
		\item Доказать, что определения равносильны.
		\begin{enumerate}[I.]
			\item Функция называется выпуклой на отрезке $[a, b]$ если для каждого отрезка $[x_1, x_2]$, принадлежащего $[a, b]$, график функции f расположен не выше отрезка с концами в $(x_1, f(x_1))$ и $(x_2, f(x_2))$ (строго выпуклой, если ниже). Иными словами, если для любых $x_1, x, x_2 \in [a, b]$, $x_1 < x < x_2$ справедливо неравенство $f(x) \leqslant y(x) = \dfrac{x_2 - x}{x_2 - x_1} f(x_1) + \dfrac{x - x_1}{x_2 - x_1} f(x_2)$, что равносильно $(x_2 - x_1) f(x) \leqslant (x_2 - x) f(x_1) + (x - x_1) f(x_2)$.
			\item Пусть $f: X \rightarrow \mathbb{R}$, $X$~--- выпуклое множество.\\
			$f(x)$~--- выпуклая (выпуклая вниз) на $X$, если
			$$
			\forall x_1, x_2 \in X \qquad f \left( \dfrac{x_1 + x_2}{2} \right) \leqslant \dfrac{f(x_1) + f(x_2)}{2}
			$$
			$f(x)$~--- строго выпуклая на $X$, если
			$$
			\forall x_1, x_2 \in X \qquad f \left( \dfrac{x_1 + x_2}{2} \right) < \dfrac{f(x_1) + f(x_2)}{2}
			$$
			$f(x)$~--- вогнутая (выпуклая вверх) на $X$, если
			$$
			\forall x_1, x_2 \in X \qquad f \left( \dfrac{x_1 + x_2}{2} \right) \geqslant \dfrac{f(x_1) + f(x_2)}{2}
			$$
			$f(x)$~--- строго вогнутая на $X$, если
			$$
			\forall x_1, x_2 \in X \qquad f \left( \dfrac{x_1 + x_2}{2} \right) > \dfrac{f(x_1) + f(x_2)}{2}
			$$
		\end{enumerate}
		
		\item Доказать лемму.
		\begin{lemma}
			Пусть функция $f$ определена на отрезке $[a, b]$. Тогда она выпукла тогда и только тогда, когда для любых $x_1, x, x_2 \in [a, b]$, $x_1 < x < x_2$, справедливо любое из следующих трех неравенств:
			$$
			\frac{f(x) - f(x_1)}{x - x_1} \leqslant \frac{f(x_2) - f(x)}{x_2 - x},
			$$
			$$
			\frac{f(x) - f(x_1)}{x - x_1} \leqslant \frac{f(x_2) - f(x_1)}{x_2 - x_1},
			$$
			$$
			\frac{f(x_2) - f(x_1)}{x_2 - x_1} \leqslant \frac{f(x_2) - f(x)}{x_2 - x},
			$$
			причем в случае строгой выпуклости неравенства также будут строгими.
		\end{lemma}
		
		\item Докажите, что если $f(x)$ выпуклая на $(a,b)$, то для любой точки $x_0$, в которой функция дифференцируема верно 
		\begin{equation*}
			f(x)\geq f(x_0)+f'(x_0)(x-x_0), \quad \forall x\in(a,b)    
		\end{equation*}
		(т.е. график функции лежит выше касательной). Причем для строго выпуклых функций неравенство строгое при всех $x\ne x_0$.
		
		\item Сформулируйте и докажите обратное утверждение.
		
		\item Докажите, что функция $\ln x$ вогнута на $(0,+\infty)$.
		
		\item Приведите пример функции, непрерывной в точке $x_0$, но не имеющей в этой точке ни левой, ни правой производной.
		
		\item При каких $\alpha$ функция $f(x) =
		\begin{cases}
			x^\alpha\sin \left( \dfrac{1}{x} \right),\ x \ne 0,\\
			0,\ x = 0
		\end{cases}$ дифференцируема в нуле.
		
		\item Найти $f'_{\pm}(x_0)$ для функции $f(x) = |x - x_0|\varphi(x)$, где функция $\varphi(x)$ непрерывна в точке $x_0$.
		
		\item Исследуйте на диффиренцируемость функцию $f(x) = |x^3 (x+1)^2 (x+2)|$.
		
		\item Функция $f(x)$ дифференцируема в точке $x_0$. Найдите предел последовательности\\
		$a_n = n \left( f \left( x_0 + \dfrac{1}{n} \right) - f(x_0) \right)$. Верно ли обратное утверждение (т.е. влечет ли сходимость такой последовательности существование производной).
	\end{enumerate}
\end{document}