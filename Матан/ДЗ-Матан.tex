\documentclass[12pt]{article}

\input{C:/Users/ilden/Documents/School/TeX памятка/header.tex}

\begin{document}
	\begin{center}
		\large\bf Домашнее задание \textnumero 2
	\end{center}
	Каждая задача (если специально не указано) оценивается в 0,5 балла.\medskip
	
	\begin{enumerate}
		\item Доказать, что $e^{\frac{1}{x}} = \overline{o}\left(x^n\right)$, $\forall n \in \mathbb{N}$, $x \rightarrow 0 - 0$.
		\item Доказать, что $\forall s > 0$; $\forall a > 1$: $x^s = \overline{o}\left(a^x\right)$, $x \rightarrow + \infty$.
		\item Доказать, что $\forall s > 0$; $\forall p > 0$: $\left(\ln x\right)^5 = \overline{o}\left(x^p\right)$, $x \rightarrow + \infty$.\bigskip
		
		В задачах (4)--(8) нужно найти степенную асимптотику (т.е. получить формулу вида $f(x)\sim ax^s$ для каких-нибудь $a$ и $s$.)
		
		\item $f(x) = \dfrac{\sqrt{x}}{\sqrt{x + 2} - 2\sqrt{x + 1} + \sqrt{x}}$, $x \rightarrow 0$, $x \rightarrow + \infty$.
		\item $f(x) = \dfrac{1 - \cos x \sqrt{\cos 2x}}{x^5}$, $x \rightarrow 0$.
		\item $f(x) = \dfrac{x^2 \arctg x}{x^5 + x^2 + 1}$, $x \rightarrow 0$, $x \rightarrow + \infty$, $x \rightarrow - \infty$.
		\item $f(x) = \sqrt{1 + 2x} - \sqrt[3]{1 + 3x}$, $x \rightarrow 0$, $x \rightarrow + \infty$.
		\item $f(x) = 1 - \cos \left(1 - \cos \left(\dfrac{1}{x}\right)\right)$, $x \rightarrow \infty$.
		\item Найти асимптотику $\ln (1 + e^x)$, $x \rightarrow + \infty$, $x \rightarrow - \infty$.
		
		\item (*) Найти асимптотику $x^x-1\sim a(x-1)^s$, $x \rightarrow 1$.
		
		\item Найти асимптотику $\sqrt{1-\sqrt[3]{x}}\sim a(1-x)^s$, $x \rightarrow 1-0$.
		
		\item Пусть $\forall\delta>0$\quad $\exists\varepsilon=\varepsilon(\delta,x_0)>0$ такое, что если $|x-x_0|<\delta$, то $|f(x)-f(x_0)|<\varepsilon$. Следует ли отсюда, что функция $f$ непрерывна в точке $x_0$. Какое свойство функции описывается таким образом?
		
		\item Пусть $\forall\varepsilon>0$\quad $\exists\delta=\delta(\varepsilon,x_0)>0$ такое, что если $|f(x)-f(x_0)|<\varepsilon$, то $|x-x_0|<\delta$. Следует ли отсюда, что функция $f$ непрерывна в точке $x_0$. Какое свойство функции описывается таким образом?
		
		\item Пусть $\forall\delta>0$\quad $\exists\varepsilon=\varepsilon(\delta,x_0)>0$ такое, что если $|f(x)-f(x_0)|<\varepsilon$, то $|x-x_0|<\delta$. Следует ли отсюда, что функция $f$ непрерывна в точке $x_0$. Какое свойство функции описывается таким образом?
		
		\textit{Указание.} Рассмотрите функцию
		$$
		f(x)=
		\begin{cases}
			\arctg(x),\quad x\in\mathbb{Q}\\
			\pi-\arctg(x),\quad x\in\mathbb{R}\setminus\mathbb{Q}.
		\end{cases}
		$$
	\end{enumerate}
\end{document}