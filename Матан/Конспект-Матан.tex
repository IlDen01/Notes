\documentclass[12pt]{article}

\input{C:/Users/ilden/Documents/School/TeX-Cheat-Sheet/header.tex}

\newcommand{\twoheaduparrow}{\rotatebox[origin=c]{90}{$\twoheadrightarrow$}}
\newcommand{\twoheaddownarrow}{\rotatebox[origin=c]{-90}{$\twoheadrightarrow$}}

\begin{document}
	\tableofcontents
	\setcounter{tocdepth}{3}
	\newpage
	\section{Последовательность.}
	$f: \mathbb{N} \rightarrow \mathbb{R}$ \\
	$f(n) =: f_n$
	\begin{definition}
		Последовательность называется ограниченной сверху, если $\exists M: |f_n| \leqslant M$. Снизу, если $\exists m:  f_n \geqslant m$. $f_n$~--- ограниченная, если ограничена сверху и снизу.
	\end{definition}
	\begin{definition}
		$M_0 = \sup f_n$, если $M_0$~--- верхняя грань и $\forall \varepsilon > 0$ $\exists n_0: f_{n0} > M_0 - \varepsilon$. $m_0 = \inf f_n$, если $m_0$~--- нижняя грань и $\forall \varepsilon > 0$ $\exists n_0: f_{n0} < m_0 + \varepsilon$.
	\end{definition}
	\begin{axiom}[Вещественных чисел]
		Если множество $X$ ограничено сверху, то $\exists sup X$. Если $f_n$ неограничено сверху, то $sup f_n =: + \infty$. Если снизу, то $inf f_n =: - \infty$.
	\end{axiom}
	\begin{definition}
		$f_n$~--- бесконечно большая (бб), если $\forall \varepsilon > 0$ $\exists N = N(\varepsilon): |f_n| > \frac{1}{\varepsilon}$ $\forall n \geqslant N$. $f_n$~--- не бб, $\exists \varepsilon > 0: \forall N \exists n > N: |f_n| \leqslant \frac{1}{\varepsilon}$.
	\end{definition}
	\begin{definition}
		$f_n$~--- бесконечно малая (бм), если $\forall \varepsilon > 0$ $\exists N = N(\varepsilon): |f_n| <\varepsilon$ $\forall n \geqslant N$.
	\end{definition}
	\begin{lemma}
		$f_n$~--- бм $\Rightarrow f_n$~--- ограничена.
	\end{lemma}
	\begin{proof}
		Пусть $\varepsilon = 1$, тогда $\exists N: |f_n| \leqslant 1$ $\forall n \geqslant N$. \\
		$M := max{|f_1|, \dots, |f_{N - 1}|, 1}$, тогда $|f_n| \leqslant M$ $\forall n \in N$.
	\end{proof}
	\begin{lemma}
		\begin{enumerate}[a)]
			\item $f_n$~--- бб $\Rightarrow \frac{1}{f_n}$~--- бм
			\item $f_n$~--- бм ($f_n \not= 0$) $\Rightarrow \frac{1}{f_n}$~--- бб
		\end{enumerate}
	\end{lemma}
	\begin{lemma}
		$f_n$~--- неограниченная последовательность, тогда существует бб подпоследовательность $f_{nk}$.
	\end{lemma}
	\begin{proof}
		$\exists n_1: |f_{n1} > 1|$, \\
		$\exists n_2 > n_1: |f_{n2} > 2|$, \\
		$\exists n_3 > n_2: |f_{n3}| > 3$, \\
		$\vdots$, \\
		$\exists n_1 < n_2 < \dots < n_k < \dots$ \\
		$|f_{nk}| > k \Rightarrow f_{nk}$~--- бб.
	\end{proof}
	\begin{lemma}
		\begin{enumerate}[a)]
			\item бм $+$ бм $=$ бм
			\item бм $\cdot$ C $=$ бм
			\item бм $\cdot$ бм $=$ бм
			\item бб $\cdot$ C $=$ бб, $C \not= 0$
			\item бб $\cdot$ бб $=$ бб
		\end{enumerate}
	\end{lemma}
	\subsection{Предел последовательности.}
	$a_n$~--- последовательность.
	\begin{definition}
		$a = \lim$ $a_n$, если $\forall \varepsilon > 0$ $\exists N:$ $|a_n - a| < \varepsilon$ $\forall n \geqslant N$.
	\end{definition}
	\begin{definition}
		Эпсилон окрестность: $U_{\varepsilon} (a) := (a - \varepsilon; a + \varepsilon)$. Выколотая эпсилон окрестность: $\mathring{U}_{\varepsilon} (a) := U_{\varepsilon}(a) \backslash \{a\}$.
	\end{definition}
	\noindent
	$\varepsilon_1 < \varepsilon_2 \Rightarrow U_{\varepsilon1}(a) \subset U_{\varepsilon2}(a), a \in \overline{\mathbb{R}}$.
	\begin{definition}
		$\mathbb{R} \cup \{\pm \infty\} = \overline{\mathbb{R}}$~--- расширенная числовая прямая.
	\end{definition}
	\begin{definition}
		$\varepsilon > 0$ $U_{\varepsilon} (+ \infty) = (\frac{1}{\varepsilon}; + \infty)$; $U_{\varepsilon}(- \infty) = (- \infty; -\frac{1}{\varepsilon})$.
	\end{definition}
	\noindent
	$\lim |a_n| = + \infty \Leftrightarrow \forall \varepsilon > 0$ $\exists N(\varepsilon):$ $|a_n| > \frac{1}{\varepsilon}$. \\
	Если $a_n$~--- бб $\Leftrightarrow \lim |a_n| = + \infty$. \\
	Если $a_n$~--- бм $\Leftrightarrow \lim |a_n| = 0$.
	\begin{statement}
		$\lim a_n = a \Leftrightarrow \exists$ бм последовательность $d_n$, такая что $a_n = a + d_n$.
	\end{statement}
	\begin{statement}
		Если предел последовательности существует, то он единственный.
	\end{statement}
	\begin{proof}
		$\exists a < b$ и $a = \lim a_n$, $b = \lim a_n$. Тогда $\varepsilon := \frac{b - a}{42}:$ \\
		$\exists N_1:$ $a_n \in U_{\varepsilon} (a) \forall n \geqslant N_1$ \\
		$\exists N_2:$ $a_n \in U_{\varepsilon} (b) \forall n \geqslant N_2$ \\
		$\Rightarrow a_n \in (U_{\varepsilon} (a) \cap U_{\varepsilon} (b)) = \emptyset$ $\forall n \geqslant \max\{N_1, N_2\} !?!$.
	\end{proof}
	\begin{lemma}[Предельный переход в неравенства]
		$a_n \leqslant b_n$ $\forall n \geqslant N_{0}$. Пусть $\exists$ $\lim a_n = a$; $\lim b_n = b$, $a, b \in \overline{\mathbb{R}}$. Тогда $a \leqslant b$.
	\end{lemma}
	\begin{proof}
		Пусть $a > b$. Тогда $\varepsilon := \frac{a - b}{42}:$ \\
		$\exists N_1:$ $a_n \in U_{\varepsilon} (a) \forall n \geqslant N_1$ \\
		$\exists N_2:$ $a_n \in U_{\varepsilon} (b) \forall n \geqslant N_2$ \\
		$\Rightarrow a_n > b_n$ $\forall n \geqslant \max\{N_1, N_2\} !?!$.
	\end{proof}
	\begin{lemma}[О сжатой последовательности]
		Пусть $a_n \leqslant b_n \leqslant c_n$ $\forall n \geqslant N_0$ и $\exists$ $\lim a_n = \lim c_n = a \in \overline{\mathbb{R}}$, тогда $\exists$ $\lim b_n = a$.
	\end{lemma}
	\begin{proof}
		$\varepsilon > 0:$ \\
		$\exists$ $a_n \in U_{\varepsilon}$, $n \geqslant N_1$ \\
		$\exists$ $c_n \in U_{\varepsilon}$, $n \geqslant N_2$ \\
		$\Rightarrow b_n \in U_{\varepsilon}:$ $\forall n \geqslant \{N_1, N_2, N_0\} =: N \Rightarrow a = \lim b_n$ по определению.
	\end{proof}
	\begin{lemma}[Об отделимости от нуля]
		Пусть $\exists$ $\lim a_n = a > 0$. Тогда $\exists$ $N:$ $a_n > \frac{a}{2} > 0$, $\forall n \geqslant N$. \\
		Следствие. Если $\lim a_n \not= 0 \Rightarrow \frac{1}{a_n}$ ограничена ($a_n \not= 0$).
	\end{lemma}
	\begin{proof}
		$\lim a_n = a > 0$ \\
		$\exists N_1:$ $a_n > \frac{a}{2} \Rightarrow 0 < \frac{1}{a_n} < \frac{2}{a}$ $\forall n \geqslant N_1$ \\
		$\min\{a_1, \dots, a_{N - 1}, \frac{a}{2}\} \leqslant \frac{1}{a_n} \leqslant \max\{a_1, \dots, a_{N - 1}, \frac{2}{a}\}$
	\end{proof}
	\begin{theorem}[Арифметические свойства предела]
		Пусть $\lim a_n = a$, $\lim b_n = b$; $a, b \in \overline{\mathbb{R}}$. Тогда:
		\begin{enumerate}
			\item $\lim (a_n + b_n) = a + b$, кроме случаев $+ \infty + (- \infty)$, $- \infty + (+ \infty)$
			\item $\lim (ka_n) = ka$, кроме случая $0 \cdot (\pm \infty)$
			\item $\lim (a_n \cdot b_n) = ab$, кроме случая $0 (\pm \infty)$
			\item $\lim \frac{a_n}{b_n} = \frac{a}{b}$, кроме случаев $\frac{0}{0}$, $\frac{\infty}{\infty}$
		\end{enumerate}
	\end{theorem}
	\begin{proof}
		$a, b \in \mathbb{R}$. $a_n = a + \alpha_n$, $b_n = b + \beta_n$; $\alpha_n, \beta_n$~--- бм. 
		\begin{enumerate}
			\item
			$a_n + b_n = (a + b) + (\alpha_n + \beta_n) \Leftrightarrow \lim (a_n + b_n) = a + b$.
			\item Аналогично.
			\item $a_nb_n = (a + \alpha_n)(b + \beta_n) = ab + \alpha_nb + \beta_na + \alpha_n + \beta_n$
			\item Если $b \not= 0$ $\frac{1}{b_n}$~--- ограниченна \\
			$\frac{a_n}{b_n} - \frac{a}{b} = \frac{a + \alpha_n}{b + \beta_n} - \frac{a}{b} = \frac{\alpha_nb - \beta_na}{b_nb} = \frac{1}{b} \cdot \frac{1}{b_n} \cdot (\alpha_nb - \beta_na)$ \\
			Если $b = 0 \Rightarrow b_n$ бм $\Rightarrow \frac{1}{b_n}$~--- бб $\Rightarrow a_n \cdot \frac{1}{b_n} =$ ограниченная бб
		\end{enumerate}
	\end{proof}
	\begin{definition}
		Линейное пространство~--- множество, сумма двух элементов которого лежит в этом множестве и элемент с коэффициентом лежит в этом множестве.
	\end{definition}
	\begin{definition}
		Последовательность называется возвратной, если $a_n = \beta_{n - 1} a_{n - 1} + \beta_{n - 2} a_{n - 2} + \dots + \beta_{n - k} a_{n - k}$; $\beta{i}$~--- фиксированные коэффициенты.
	\end{definition}
	\noindent
	$a_n^{(1)}, a_n^{(2)} \Rightarrow \forall \lambda, \mu \in \mathbb{R}$ $\lambda a_n^{(1)} + \mu a_n^{(2)}$ тоже удовлетворяет $(x)$. \\
	$a_n := t^n$ \\
	$t^k = \beta_{n - 1}t^{k - 1} + \dots + \beta_{n - k}$ \\
	$t_0$~--- простой корень, то $t_0^n$ \\
	$t_0$~--- корень $(m) \Rightarrow t_0^n; nt_0^n; n^2t_0^n; \dots; n^{m - 1}t_0^n$ \\
	\begin{theorem}
		\begin{enumerate}
			\item Пусть $a_n$ возрастает и ограничена сверху. Тогда $\exists \lim a_n = \sup a_n$
			\item Пусть $a_n$ убывает и ограничена снизу. Тогда $\exists \lim a_n = \inf a_n$
		\end{enumerate}
	\end{theorem}
	\begin{proof}
		fix $\varepsilon > 0$. Так как $a_n$~--- ограничена, то $\exists M  \sup a_n \in \mathbb{R}$; И $\exists N: a_N > M - \varepsilon$. \\
		Тогда $\begin{cases}
			a_n \geqslant M - \varepsilon & \forall n \geqslant N \text{, так как } a_n \uparrow \\
			a_n \leqslant M < M + \varepsilon
		\end{cases} \Rightarrow \exists N: |a_n - M| < \varepsilon \forall n \geqslant N \Rightarrow \lim\limits_{n \rightarrow \infty} a_n = M$ по определению.
	\end{proof}
	\begin{definition}
		Найти предел последовательности $a_n = (1 + \frac{1}{n})^n$.
		\begin{quote}
			$b_n = (1 + \frac{1}{n})^{n + 1}; b_1 = 4, b_2 = 3,...$ $b_n \downarrow$ \\
			$b_n \geqslant 1$ \\
			Докажем, что $b_n$ убывает. \\
			$\frac{b_n}{b_{n + 1}} = \frac{(\frac{n + 1}{n})^{n + 1}}{(\frac{n + 2}{n + 1})^{n + 2}} = \frac{n + 1}{n})^{n + 1} \cdot \frac{n + 1}{n + 2})^{n + 2} = \frac{n + 1}{n + 2} \cdot (\frac{n^2 + 2n + 1}{n^2 + 2n})^{n + 1} = \frac{n + 1}{n + 2} \cdot (1 + \frac{1}{n^2 + 2n})^{n + 1} \overbrace{>}^{\text{Неравенство Бернулли}} \frac{n + 1}{n + 2} \cdot (1 + \frac{n + 1}{n^2 + 2n}) = \frac{(n + 1)(n^2 + 3n + 1)}{(n + 2)(n^2 + 2n)} = \frac{n^3 + 4n^2 + 4n + 1}{n^3 + 4n^2 + 4n} > 1$. \\
			$a_n = \frac{b_n}{(1 + \frac{1}{n})}$ \\
			$\lim\limits_{n \rightarrow \infty} = \lim \frac{b_n}{1 + \frac{1}{n}} = \frac{\lim b_n}{\lim (1 + \frac{1}{n})} = \lim b_n$~--- существует. \\
			$e := \lim\limits_{n \rightarrow \infty} (1 + \frac{1}{n})^n \approx 2.718281828459045...$
		\end{quote}
	\end{definition}
	\begin{theorem}[Больцано-Вейерштрасса]
		Пусть последовательность $a_n$ ограничена. Тогда существует сходящаяся подпоследовательность.
	\end{theorem}
	\begin{proof}
		$|a_n| \leqslant M$ \\
		$[-M = \alpha_1; M = \beta_1]$. $\alpha_2$~--- середина. $a_1 = x_1 \in [\alpha_1; \alpha_2]$. \\
		$[\alpha_2; \beta_2]$. $\beta_3$~--- середина. $x_2 = a_{\min n} \in [\alpha_2; \beta_3]$. \\
		И тд. \\
		${\alpha_k}$ неубывающая и ограниченная сверху. $\exists \lim \alpha_k = \alpha$. ${\beta_k}$ неубывающая и ограниченная сверху. $\exists \lim \beta_k = \beta$. \\
		$\beta - \alpha = \lim \limits_{k \rightarrow \infty} (\beta_k - \alpha_k) = \lim \limits_{k \rightarrow \infty} \frac{2M}{2^{k - 1}} = 0$. \\
		По построению $x_k$~--- подпоследовательность и $\alpha_k \leqslant x_k \leqslant \beta_k \Rightarrow \exists \lim x_k$.
	\end{proof}
	\begin{definition}
		Последовательность называется фундаментальной (или последовательностью Коши), если $\forall \varepsilon > 0$ $\exists N(\varepsilon)$: $|a_n - a_k| < \varepsilon$ $\forall n, k \geqslant N$.
	\end{definition}
	\begin{statement}
		Пусть $\exists \lim a_n = a \in \mathbb{R}$. Тогда $\{a_n\}$ фундаментальная.
	\end{statement}
	\begin{proof}
		fix $\varepsilon > 0 \exists N: |a_n - a| < \frac{\varepsilon}{2} \forall n \geqslant N$. Тогда $\forall n, k \geqslant N$ $|a_n - a_k| = |(a_n - a) + (a - a_k)| \leqslant |a_n - a| + |a_k - a| < \frac{\varepsilon}{2} + \frac{\varepsilon}{2} = \varepsilon$.
	\end{proof}
	\begin{theorem}[Коши]
		Пусть $\{a_n\}$ фундаментальная последовательность. Тогда $\exists \lim a_n$.
	\end{theorem}
	\begin{proof}
		\begin{enumerate}[1)]
			\item $(!) \{a_n\}$ ограничена. \\
			$\varepsilon = 1$: $\exists N$: $|a_n - a_k| \leqslant \varepsilon$ $\forall n, k \geqslant N \Rightarrow a_k \in [a_{N - 1}; a_{N + 1}] \forall k \geqslant N$ \\
			$M = max\{|a_1|, |a_2|, \dots, |a_N + 1|\} \Rightarrow |a_n| \leqslant M \forall n$.
			\item Тогда по теореме Вейерштрасса $\exists a_{n_k}$~--- подпоследовательность; $\lim \limits_{n \rightarrow \infty} a_{n_k} = a$.
			\item fix $\varepsilon > 0$. $\exists N_1: |a_{n_k} - a| < \frac{\varepsilon}{2}$ $\forall n_k \geqslant N_1$ \\
			$\exists N_2: |a_m - a_n| < \frac{\varepsilon}{2}$ $\forall m, n \geqslant N_2$ \\
			Пусть $n \geqslant max\{N_1, N_2\}$ $\exists n_k \geqslant m$. \\
			$|a_m - a| = |(a_m - a_{n_k}) + (a_{n_k} - a)| \leqslant |a_n - a_{n_k}| + |a_{n_k} - a| < \frac{\varepsilon}{2} + \frac{\varepsilon}{2} = \varepsilon \Rightarrow a = \lim \limits_{m \rightarrow \infty} a_m$.
		\end{enumerate}
	\end{proof}
	\section{Возведение в вещественную степень.}
	$n \in \mathbb{N}$; $x^n = x \cdot x \cdot ... \cdot x$, $n$ раз. \\
	$x^{-n} = \frac{1}{x^n}$, $x \not= 0$ \\
	$x^0 := 1$ \\
	$\sqrt[n]{x} = x^{\frac{1}{n}}$ \\
	$x^p$, $p \in \mathbb{Q}$, $x \geqslant 0$ $(p > 0)$ или $x > 0$ $(p \geqslant 0)$ \\
	\underline{fix $a > 0$, $x \in \mathbb{R}$} \\
	$a^x := \lim \limits_{n \rightarrow \infty} a^{x_n}$, где $\{x_n\}$ последовательность, такая что $x_n \in \mathbb{Q}$, $\lim \limits_{n \rightarrow \infty} x_n := x$. \\
	Корректность определения.
	\begin{enumerate}
		\item $x \in \mathbb{Q}$. Докажем, что $a^x$ совпадает со старым определением. \\
		$x \in \mathbb{Q}$, берем $x_n = x \Rightarrow a^x = a^x$
		\item Берем произвольную последовательность $\{x_n\}$, $x_n \rightarrow x \Rightarrow x_n$~--- фундаментальная последовательность. \\
		fix $\varepsilon > 0$ $|a^{x_n} - a^{x_k}| = a^{x_n}|1 - a^{x_k - x_n}|$. Сходится. Значит ограничена. Тогда $a^{x_n}$ ограничена. Тогда $a^{x_n}|1 - a^{x_k - x_n}| \leqslant M \cdot |1 - a^{x_k - x_n}|$. \\
		$\exists N: |a^{\frac{1}{m}} - 1| < \frac{M}{\varepsilon}$ $\forall m \geqslant N \Rightarrow |a^{\frac{1}{n}} - 1| < \frac{M}{\varepsilon} \Rightarrow \exists N_0: |x_n - x_k| < \frac{1}{N}$ $\forall n, k \geqslant N$ \\
		$M \cdot |1 - a^{x_k - x_n}| < M \cdot \frac{2}{M}$ $\forall n, k \geqslant N_0 \Rightarrow a^{x_n}$ образует фундаментальную последовательность $\Rightarrow$ (по теореме Коши) $\exists \lim \limits_{n \rightarrow \infty} a^{x_n}$
		\item $x_n \rightarrow x$, $y_m \rightarrow x$, $x_n, y_m \in \mathbb{Q}$ \\
		$\exists a = \lim a^{x_n}$; $\alpha = \lim a^{y_m}$ \\
		$(a - \alpha) = \lim \limits_{n \rightarrow \infty} (a^{x_n} - a^{y_m}) = \lim \limits_{n \rightarrow \infty} a^{x_n} (1 - a^{x_n - y_m}) = 0$ \\
		$\lim(y_n - x_n) = x - x = 0$
	\end{enumerate}
	Свойства:
	\begin{enumerate}
		\item $a^x \cdot a^y = a^{x + y}$ \\
		Пусть $x_n \rightarrow x$, $y_n \rightarrow y$; $x_n, y_n \in \mathbb{Q}$ \\
		$a^{x_n} \cdot a^{y_n} = a^{x_n + y_n}$ \\
		$a^x \cdot a^y = \lim a^{x_n} \cdot \lim a^{y_n} = \lim a^{x_n} \cdot a^{y_n} = \lim a^{x_n + y_n} = a^{x + y}$
		\item $(a^x)^y = a^{xy}$ \\
		Пусть $x_n \rightarrow x$, $y_m \rightarrow y$; $x_n, y_m \in \mathbb{Q}$ \\
		$x_ny_n \rightarrow xy$ \\
		$\lim \limits_{m \rightarrow \infty} (\lim \limits_{n \rightarrow \infty} a^{x_n})^{y_m}$ ? $\lim \limits_{n \rightarrow \infty} a^{x_ny_n}$ \\
		$\lim \limits_{n \rightarrow \infty} (a^x)^{y_n} = \lim \limits_{n \rightarrow \infty} a^{x_ny_n}$ \\
		$\lim \limits_{n \rightarrow \infty} |b^{y_n} - a^{x_ny_n}| = 0$, где $b = a^x$ \\
		$\exists N:$ $|a^{x_n} - b| < \varepsilon$ $\forall n \geqslant N$ \\
		$b - \varepsilon < a^{x_n} < b + \varepsilon$ \\
		$1 - \frac{\varepsilon}{b} < \frac{a^{x_n}}{b} < 1 + \frac{\varepsilon}{b}$ $\uparrow ^{y_n}, y_n > 0$, для $< 0$ аналогично \\
		$(1 - \frac{\varepsilon}{b})^{y_n} < (\frac{a^{x_n}}{b})^{y_n} < (1 + \frac{\varepsilon}{b})^{y_n}$ \\
		$1 - \frac{y_n \varepsilon}{b} < \frac{a^{x_ny_n}}{b^{y_n}} < 1 + \frac{y_n \varepsilon}{b}$ \\
		$\Rightarrow |\frac{a^{x_ny_n}}{b^{y_n}} - 1| \leqslant \frac{|y_n|}{b} \varepsilon$ \\
		$|a^{x_ny_n} - b^{y_n}| < \frac{|y_n|}{b} \cdot b^{y_n} \varepsilon \leqslant M\varepsilon$ $\forall n \geqslant N$ \\
		$\Rightarrow \lim |b^{y_n} - a^{x_ny_n}| = 0$
	\end{enumerate}
	fix $a > 0$, $a \not= 1$ \\
	$y = a^x$, $x \in \mathbb{R}$
	\begin{definition}
		$f(x)$~--- возрастающая на $X$, если $\forall x_1, x_2 \in X$ $x_1 < x_2 \Rightarrow f(x_1) < f(x_2)$. $f(x)$ неубывающая, если $\leqslant$.
	\end{definition}
	\begin{statement}
		При $a > 1$ $f(x) = a^x$~--- возрастает на $\mathbb{R}$; При $0 < a,< 1$ $f(x) = a^x$~--- убывает на $\mathbb{R}$.
	\end{statement}
	\begin{quote}
		Пусть $x < \xi$, $x_n \rightarrow x$, $\xi_n \rightarrow \xi$ \\
		$x < x_0 < \xi_0 < \xi$ и $x_0$ не в окрестности $x$ и аналогично $\xi$. \\
		$\forall n \geqslant N$ \\
		$x_n < x_0 < \xi_0 < \xi_0$ \\
		Тогда $a^{x_n} < a^{x_0} < a^{\xi_0} < a^{\xi_n}$ \\
		$a^{x} \leqslant a^{x_0} < a^{\xi_0} \leqslant a^{\xi} \Rightarrow a^x < a^{\xi}$ \\
		$0 < a < 1$ \\
		$a^x = (\frac{1}{a})^{-x}$; $a^{\xi} = (\frac{1}{a})^{-\xi}$ \\
		$x < \xi \Rightarrow -x > -\xi$ $(\frac{1}{a})^{-x} > (\frac{1}{a})^{-\xi} \Rightarrow a^x > a^{\xi}$
	\end{quote}
	\section{Предел и непрерывность.}
	\begin{definition}
		Передельная точка $x_0$ области $D$~--- такая точка, что $\forall \varepsilon > 0$ $\mathring{U}_{\varepsilon}(x_0) \cap D \not= \emptyset$. Множество предельных точек множества $D$~--- называется замыкание $D$, обозначается $\overline{D}$.
	\end{definition}
	\noindent
	$f: D \rightarrow \mathbb{R}$; $D \subset \mathbb{R}$. Пусть $x_0$~--- предельная точка $D$. \\
	\begin{definition}[По Гейне]
		Если $\forall$ последовательности $x_n \rightarrow x_0$; $x_n \not= x_0$ $\exists \lim \limits_{n \rightarrow \infty} f(x_n) = a$, то говорят, что $\exists \lim \limits_{x \rightarrow x_0} f(x) = a$.
	\end{definition}
	\begin{definition}
		$f$ непрерывна в точке $x_0 \in D$, если $\exists \lim \limits_{x \rightarrow x_0} f(x) = f(x_0)$.
	\end{definition}
	\begin{note}
		Функция не непрерывна:
		\begin{enumerate}
			\item $\lim\limits_{x \rightarrow x_0} \exists$, но $\not= f(x_0)$~--- устранимый разрыв.
			\item $\lim\limits_{x \rightarrow x_0} \nexists$~--- неустранимы разрыв.
			\begin{enumerate}
				\item $\lim\limits_{x \rightarrow x_0 - 0} f(x) = a$, $\lim\limits_{x \rightarrow x_0 + 0} f(x) = b$, $a, b \in \mathbb{R}$~--- разрыв I рода или скачок.
				\item Разрыв второго рода или существенная особенность.
			\end{enumerate}
		\end{enumerate}
	\end{note}
	\begin{definition}[По Коши]
		$\lim \limits_{x \rightarrow x_0} f(x) = a$, если $\forall \varepsilon > 0$ $\exists \delta > 0$: $f(x) \subset U_{\varepsilon}(a)$ $\forall x \in \mathring{U}_{\delta}(x_0) \cap D(f)$.
	\end{definition}
	\begin{statement}
		Определения по Коши и по Гейне равносильны.
	\end{statement}
	\begin{proof}
		Доказательство Коши $\Rightarrow$ Гейне.
		\begin{quote}
			Известно, что $\forall \varepsilon > 0$ $\exists \delta > 0$: $f(x) \in U_{\varepsilon}(a) \forall x \in \mathring{U}_{\delta}(x_0)$. \\
			Берем произвольную $x_n \rightarrow x_0$: fix $\varepsilon > 0 \Rightarrow \exists \delta$, такое что выполнено $f(x) \in U_{\varepsilon}(a) \forall x \in \mathring{U}_{\delta}(x_0)$ и $\exists N$: $x_n \in U_{\delta}(x_0) \forall n \geqslant N \Rightarrow \exists N$: $\forall n \geqslant N$ $x_n \in \mathring{U}_{\delta}(x_0) \Rightarrow f(x_n) \in U_{\varepsilon}(a)$. \\
			Те $\forall \varepsilon > 0$ $\exists N$: $\forall n \geqslant N$ $f(x_n) \in U_{\varepsilon}(a)$, те $a = \lim \limits_{n \rightarrow \infty}f(x_n)$
		\end{quote}
		Доказательство не Коши $\Rightarrow$ не Гейне.
		\begin{quote}
			Нет предела по Коши. fix $a$: $\exists \varepsilon$: $\forall \delta > 0$. $\exists \tilde{x} \in \mathring{U}_{\delta}(x_0)$: $f(\tilde{x}) \not\in U_{\varepsilon}(a) \Rightarrow a$ не является пределом по Гейне? \\
			От противного. $a = \lim \limits_{x \rightarrow x_0} f(x)$ по Гейне. \\
			$\delta_n = \frac{1}{n}$ $\exists \tilde{x_n} \in \mathring{U}_{\frac{1}{n}}(x_0)$ и $f(\tilde{x_n}) \not\in U_{\varepsilon}(a) \Rightarrow \tilde{x_n} \rightarrow x_0, \tilde{x_n} \not= x_0$, но $\lim \limits_{n \rightarrow \infty} f(\tilde{x_n}) \not= a$
		\end{quote}
	\end{proof}
	\noindent
	\begin{statement}
		$f(x)$ бесконечно малая в точке $x_0$, если $\lim \limits_{x \rightarrow x_0} f(x) = 0$.
	\end{statement}
	\begin{statement}
		$f(x)$ бесконечно большая в точке $x_0$, если $\lim \limits_{x \rightarrow x_0} |f(x)| = +\infty$.
	\end{statement}
	\begin{lemma}[О двух милиционерах]
		Пусть $f(x) \leqslant g(x) \leqslant h(x)$ в некоторой окрестности точки $x_0$ (передельная точка $D(f), D(g), D(h)$). Если $\exists \lim \limits_{x \rightarrow x_0} f(x) = \lim \limits_{x \rightarrow x_0} h(x) = a$, то $\exists \lim \limits_{x \rightarrow x_0} g(x) = a$.
	\end{lemma}
	\begin{lemma}[Предельный переход в неравенства]
		Пусть $f(x) \leqslant g(x)$ в некоторой окрестности точки $x_0$ (передельная точка $D(f), D(g)$). Если $\exists \lim \limits_{x \rightarrow x_0} f(x) = a$; $\exists \lim \limits_{x \rightarrow x_0} g(x) = b$, то $a \leqslant b$.
	\end{lemma}
	\begin{lemma}[Об ограниченности]
		Если $\exists \lim \limits_{x \rightarrow x_0} f(x) = a \in \mathbb{R}$. Тогда $f$ ограничена в некотром $\mathring{U}(x_0)$.
	\end{lemma}
	\begin{proof}
		$\varepsilon = 1$: $\exists \delta > 0$: $|f(x) - a| < 1$ $\forall x \in \mathring{U}_{\delta}(x_0) \Rightarrow |f(x)| \leqslant |a| + 1$ $\forall x \in \mathring{U}_{\delta}(x_0)$
	\end{proof}
	\begin{lemma}[Об отделимости от нуля]
		Если $\exists \lim \limits_{x \rightarrow x_0} f(x) = a > 0$. Тогда $\inf f(x) > 0$ в некоторой $\mathring{U}(x_0)$ ($f$ отделима от нуля).
	\end{lemma}
	\begin{proof}
		$\varepsilon = \frac{a}{42}$: $\exists \delta < \frac{a}{42}$ $\forall x \in \mathring{U}_{\delta}(x_0) \Rightarrow f(x) \geqslant \frac{41}{42}a > 0$ $\forall x \in \mathring{U}_{\delta}(x_0) \Rightarrow \inf f(x) \geqslant \frac{41}{42}a > 0$.
	\end{proof}
	\begin{corollary}
		Если $\lim \limits_{x \rightarrow x_0} f(x) \not= 0$, то $\frac{1}{f(x)}$ ограничена в некоторой $\mathring{U}(x_0)$.
	\end{corollary}
	\begin{definition}
		$x^{\alpha}, \alpha \in \mathbb{R}; a^x; \log_ax; \sin, \cos, \tg, \arcsin, \arccos, \arctg$~--- основные элементарные функции.
	\end{definition}
	\begin{definition}
		Функция называется элементарной, если она получается арифметическими операциями или композицией конечного числа основных элементарных функций.
	\end{definition}
	\begin{statement}
		Любая элементарная функция непрерывна.
	\end{statement}
	\begin{theorem}[Теорема о пределе композиций]
		$f: X \rightarrow Y$; $g: Y \rightarrow Z$, $x_0$~--- предельная точка множества $X$, $y_0$~--- предельная точка множества $Y$. $\exists \lim \limits_{x \rightarrow x_0} f(x) = y_0$, $\exists \lim \limits_{y \rightarrow y_0} g(y) = g(y_0) = a$. Тогда $\exists \lim \limits_{x \rightarrow x_0} g(f(x)) = a$.
	\end{theorem}
	\begin{proof}
		$\forall x \in \mathring{U}_{\delta}(x_0) \Rightarrow g(f(x)) \in U_{\varepsilon} (a)$ \\
		$a = \lim \limits_{y \rightarrow y_0} g(y) \Rightarrow \exists \varepsilon_1 > 0$: $y \in U_{\varepsilon_1} (y_0) \Rightarrow g(y) \in U_{\varepsilon} (g(y_0))$ \\
		$\exists \delta > 0$: $x \in \mathring{U}_{\delta} (x_0) \Rightarrow f(x) \in U_{\varepsilon_1} (y_0) \Rightarrow g(f(x)) \subset U_{\varepsilon} (g(y_0))$
	\end{proof}
	\begin{theorem}[Непрерывности обратной функции]
		Пусть $f$ непрерывная биекция на $\langle a; b \rangle$. Тогда $f^{-1}$ тоже непрерывная биекция $Y \rightarrow \langle a; b \rangle$.
	\end{theorem}
	\begin{proof}
		НУО $f$ строго возрастает на $\langle a; b \rangle$ \\
		fix $\varepsilon > 0$, тогда $y_1 := f(x_0 - \varepsilon)$; $y_2 := f(x_0 + \varepsilon)$ \\
		$\delta = \min \{ y_2 - y_0; y_0 - y_1 \}$ \\
		Тогда $\forall y \in U_{\delta} (y) \Rightarrow f^{-1} (y) \in U_{\epsilon} (f^{-1} (y_0))$ тк $y_1 < y < y_2 \Rightarrow f^{-1} (y_1) < f^{-1} (y) < f^{-1} (y_2)$
	\end{proof}
	\subsection{Замечательные пределы.}
	\begin{enumerate}
		\item $\lim\limits_{x \rightarrow +\infty} \left(1 + \dfrac{1}{x}\right)^x = e$
		\item $\lim\limits_{x \rightarrow -\infty} \left(1 + \dfrac{1}{x}\right)^x = e$
		\item $\lim\limits_{x \rightarrow 0} (1 + x)^{\frac{1}{x}} = e$
		\item $\lim\limits_{x \rightarrow 0} \dfrac{\ln (1 + x)}{x} = 1$
		\item $\lim\limits_{x \rightarrow 0} \dfrac{e^x - 1}{x} = 1$; $\lim\limits_{x \rightarrow 0} \dfrac{a^x - 1}{x} = \ln a$
		\item $\lim\limits_{x \rightarrow 0} \dfrac{(1 + x)^m - 1}{x} = m$; $m \in \mathbb{R}$
		\item $\lim\limits_{x \rightarrow 0} \dfrac{1 - \cos x}{x^2} = \dfrac{1}{2}$
		\item $\lim\limits_{x \rightarrow 0} \dfrac{\sin x}{x} = 1$
		\item $\lim\limits_{x \rightarrow +\infty} \dfrac{n}{2^n} = 0$
	\end{enumerate}
	\begin{statement}
		$(1 + x)^p \simeq 1 + px$, $x \rightarrow 0$.
	\end{statement}
	\section{$\overline{o}$ и $\underline{O}$.}
	\begin{definition}
		Символ Ландау~--- $\overline{o}$.
		\begin{enumerate}
			\item $\lim\limits_{x \rightarrow x_0} \frac{f(x)}{g(x)} = 0$, тогда $f(x) := \overline{o}(g(x))$, $x \rightarrow x_0$.
			\item $f(x) = \overline{o}(g(x))$, если $\forall \varepsilon > 0$ $\exists \delta > 0$: $|f(x)| \leqslant \varepsilon |g(x)|$, $\forall x \in \mathring{U}_\delta (x_0)$.
		\end{enumerate}
	\end{definition}
	\begin{property}
		\begin{enumerate}
			\item $f(x)$ называется бесконечно малой в точке $x_0$, если $f(x) = \overline{o}(1)$, $x \rightarrow x_0$.
			\item $\overline{o}(f(x)) + \overline{o}(f(x)) = \overline{o}(f(x))$, $x \rightarrow x_0$.
			\item $C \cdot \overline{o}(f(x)) = \overline{o}(f(x))$, $x \rightarrow x_0$.
			\item $\overline{o}(\overline{o}(f(x))) = \overline{o}(f(x))$, $x \rightarrow x_0$.
		\end{enumerate}
	\end{property}
	\begin{definition}
		\begin{enumerate}
			\item Пусть $f$, $g$ обе бесконечно большие или бесконечно малые. $\lim\limits_{x \rightarrow x_0} \frac{f(x)}{g(x)} = 1$, тогда $f(x) \sim g(x)$, $x \rightarrow x_0$.
			\item $f(x) \sim g(x) \Leftrightarrow \forall \varepsilon > 0$ $\exists \delta > 0$ $(1 - \varepsilon)|g(x)| \leqslant |f(x)| \leqslant (1 + \varepsilon)|g(x)|$, $\forall x \in \mathring{U}_\delta (x_0)$.
		\end{enumerate}
	\end{definition}
	\begin{statement}
		$f(x) \sim g(x)$, $x \rightarrow x_0$ $\Leftrightarrow$ $f(x) = g(x) + \overline{o}(g(x))$, $x \rightarrow x_0$.
	\end{statement}
	\begin{definition}
		$f(x) = \underline{O}(g(x))$, $x \rightarrow x_0$, если $\exists$ $C > 0$: $|f(x)| \leqslant C|g(x)|$ в некоторой выколотой окрестности точки $x_0$.
	\end{definition}
	\begin{definition}
		$f(x) \asymp g(x)$, $x \rightarrow x_0$, если $\exists$ $c_1, c_2 > 0$: $c_1|g(x)| \leqslant |f(x)| \leqslant c_2|g(x)|$ в некоторой выколотой окрестности точки $x_0$.
	\end{definition}
	\begin{theorem}
		Пусть $f(x) \sim f_1(x)$, $x \rightarrow x_0$ и функции не обращаются в $0$ в некоторой $\mathring{U}(x_0)$. Тогда $\lim\limits_{x \rightarrow x_0} f(x)g(x)$ и $\lim\limits_{x \rightarrow x_0} f_1(x)g(x)$ $\exists$ или $\nexists$ одновременно, и если $\exists$, то они равны. Словами: в произведение пределов сомножители можно заменять на эквивалентные; слагаемые нельзя!
	\end{theorem}
	\begin{proof}
		$\lim\limits_{x \rightarrow x_0} f(x)g(x) = \lim\limits_{x \rightarrow x_0} \dfrac{f(x)}{f_1(x)} \cdot f_1(x)g(x)$.
	\end{proof}
	\begin{statement}
		$(1 + t)^\alpha = 1 + \alpha t + \overline{o}(t)$, $t \rightarrow 0$.
	\end{statement}
	\begin{statement}
		$e^t = 1 + t + \overline{o}(t)$, $t \rightarrow 0$.
	\end{statement}
	\begin{statement}
		Если $\exists$ $\lim\limits_{x \rightarrow x_0} \dfrac{f(x)}{g(x)} = k \in \mathbb{R} \setminus \{ 0 \} \Rightarrow f(x) \asymp g(x), x \rightarrow x_0 \Rightarrow f(x) = \underline{O}(g(x))$ и $g(x) = \underline{O}(f(x)), x \rightarrow x_0$.
	\end{statement}
	\subsection{Асимптотические равенства.}
	\begin{enumerate}
		\item $\lim\limits_{x \rightarrow 0} \dfrac{\sin x}{x} = 1$; $\sin x \sim x$; $\sin x = x + \overline{o}(x)$
		\item $\lim\limits_{x \rightarrow 0} \dfrac{1 - \cos x}{x^2} = \dfrac{1}{2}$; $1 - \cos x \sim \dfrac{x^2}{2}$; $\cos x = 1 - \dfrac{x^2}{2} + \overline{o}(x^2)$
		\item $\lim\limits_{x \rightarrow 0} \dfrac{\ln (1 + x)}{x} = 1$; $\ln (1 + x) \sim x$; $\ln (1 + x) = x + \overline{o}$ \\
		$\log_a(1 + x) \sim \dfrac{x}{\ln a}$; $\log_a(1 + x) = \dfrac{x}{\ln a} + \overline{o}(x)$
		\item $\lim\limits_{x \rightarrow 0} \dfrac{e^x - 1}{x} = 1$; $e^x - 1 \sim x$; $e^x = 1 + x + \overline{o}(x)$ \\
		$e^x - 1 = x + \overline{o}(x)$; $a^x = 1 + x \ln a + \overline{o}(x)$
		\item $\lim\limits_{x \rightarrow 0} \dfrac{(1 + x)^s - 1}{x} = s$; $(1 + x)^s - 1 \sim sx$; $(1 + x)^s = 1 + sx + \overline{o}(x)$
	\end{enumerate}
	\section{Функции, непрерывные на отрезке.}
	\begin{definition}
		$f$ непрерывна на отрезке $[a, b]$, $a, b \in \mathbb{R}$; $a < b \Leftrightarrow \forall x_0 \in [a, b]$ $\lim\limits_{x \rightarrow x_0} f(x) = f(x_0)$.
	\end{definition}
	\begin{note}
		$C(X) = \{ \text{множество функций непрерывных на множестве } X \}$. $f \in C(x)$: $\forall x_0 \in X$ $\forall \varepsilon > 0$ $\exists \delta > 0$: $|f(x) - f(x_0)| < \varepsilon$ $\forall |x - x_0| < \delta$.
	\end{note}
	\subsection{Первая теорема Вейерштрасса.}
	\begin{theorem}[Первая теорема Вейерштрасса]
		$f \in C([a, b])$. Тогда $f$ ограничена на $[a, b]$.
	\end{theorem}
	\begin{proof}
		НУО докажем, что $f$ ограничена сверху. \\
		$\#$ $f$ неограничена сверху.	$\exists x_n \in [a, b]$: $f(x_n) > n$.	$\{ x_n \} \in [a, b]$. $\exists$ $x_{n_k} \rightarrow x_0 \in [a, b]$. $f(x_0) = \lim\limits_{x_{n_k} \rightarrow x_0} f(x_{n_k}) = \infty$ $\#$.
	\end{proof}
	\subsection{Вторая Теорема Вейерштрасса.}
	\begin{theorem}[Вторая Теорема Вейерштрасса]
		Пусть $f \in C([a, b])$. Тогда $\exists \overline{x}, \underline{x} \in [a, b]$: $f(\overline{x}) = \min\limits_{[a, b]} f(x), f(\underline{x}) = \max\limits_{[a, b]} f(x)$.
	\end{theorem}
	\begin{proof}
		$\# \inf\limits_{[a, b]} f(x) = m$ и не достигается. $f(x) > m$ $\forall x \in [a, b] \Rightarrow g(x) := \dfrac{1}{f(x) - m}$ неограничена на $[a, b]$. Но $\forall \varepsilon > 0$ $\exists$ $x_{\varepsilon} \in [a, b]$: $f(x_{\varepsilon}) < m + \varepsilon \Rightarrow g(x_{\varepsilon}) > \dfrac{1}{\varepsilon} \Rightarrow g$ неограничена сверху на $[a, b]$ $\#$.
	\end{proof}
	\subsection{Теорема Больцано-Коши.}
	\begin{theorem}[Больцано-Коши]
		$f \in C([a, b])$, $f(a)f(b) < 0$. Тогда $\exists$ $x_0 \in [a, b]$: $f(x_0) = 0$.
	\end{theorem}
	\begin{proof}[Доказательство №1]
		НУО $f(a) < 0; f(b) > 0$. $a_0 := a, b_0 := b$. $[a_k, b_k] \rightarrow [a_{k + 1}, b_{k + 1}]$. $\lim a_k = \alpha$, $\lim b_k = \beta$. $\lim (a_k - b_k) = \alpha - \beta$, $\lim \dfrac{b - a}{2^k} = 0 \Rightarrow \alpha = \beta =: x_0$. $f(a_k) < 0, f(b_k) > 0$. $ \begin{cases}
			f(x_0) = \lim\limits_{k \rightarrow \infty} f(a_k) \leqslant 0 \\
			f(x_0) = \lim\limits_{k \rightarrow \infty} f(b_k) \geqslant 0
		\end{cases} \Rightarrow f(x_0) = 0$.
	\end{proof}
	\begin{proof}[Доказательство №2]
		$X = \{ x \in [a, b]: f(x) < 0 \} \not= \emptyset$. $\exists x_0 = \sup X < b$.
	\end{proof}
	\subsection{Третья Теорема Вейерштрасса.}
	\begin{theorem}[Третья Теорема Вейерштрасса]
		$f \in C([a, b])$, $M = \max\limits_{[a, b]} f(x), m = \min\limits_{[a, b]} f(x)$. Тогда $E(f) = [m; M]$.
	\end{theorem}
	\subsection{Равномерная непрерывность.}
	\begin{definition}[Равномерная непрерывность]
		$f$ равномерно непрерывна на $X$, если $\forall \varepsilon > 0$ $\exists \delta > 0$: $\forall x_0 \in X$ $|f(x) - f(x_0)| < \varepsilon$ $\forall |x - x_0| < \delta$.
	\end{definition}
	\begin{theorem}[?Коши-Вейерштрасса?]
		Пусть $f \in C([a, b])$. Тогда $f$ равномерно непрерывна на $[a, b]$.
	\end{theorem}
	\begin{proof}
		$\#$ $\exists \varepsilon > 0$: $\forall \delta > 0$: $\exists x_1 \xi$ $|f(x) - f(\xi)| \geqslant \varepsilon$, $|x - \xi| < \delta$. Возьмем $\delta_1 = 1$ $\exists x_1, \xi_1$: $|x_1 - \xi_1| < 1$ и $|f(x_1) - f(\xi_1)| \geqslant \varepsilon$, $\dots$, $\delta_n = \dfrac{1}{n}$ $\exists x_n, \xi_n$: $|x_n - \xi_n| < \dfrac{1}{n}$ и $|f(x_n) - f(\xi_n)| \geqslant \varepsilon$. $\{ x_n \}, \{ \xi_n \} \subset [a, b]$. $\{ x_{n_m} \rightarrow x_0 \}, \{ \xi_{n_m} \rightarrow \xi_0 \} \Rightarrow x_0 = \xi_0$. $|f(x_{n_m}) \rightarrow f(x_0) - f(\xi_{n_m}) \rightarrow f(x_0)| \geqslant \varepsilon$ $\#$.
	\end{proof}
	\section{Дифференциальное исчисление.}
	\begin{definition}
		$f$ дифференцируема в точке $x_0$, если $\exists$ $k \in \mathbb{R}$: $f(x) = f(x_0) + k(x - x_0) + \overline{o}(x - x_0)$.
	\end{definition}
	\begin{statement}
		Если $f$ дифференцируема в точке $x_0$ $\Leftrightarrow$ $\exists$ $\lim\limits_{x \rightarrow x_0} \dfrac{f(x) - f(x_0)}{x - x_0} = k$.
	\end{statement}
	\begin{proof}
		$\overline{o}(x - x_0) + k(x - x_0) = f(x) - f(x_0) \Leftrightarrow k + \dfrac{\overline{o}(x - x_0)}{x - x_0} = \dfrac{f(x) - f(x_0)}{x - x_0} \Leftrightarrow k = k + \lim \dfrac{\overline{o}(x - x_0)}{x - x_0} = \lim \dfrac{f(x) - f(x_0)}{x - x_0}$.
	\end{proof}
	\begin{definition}
		Такое $k$ называется производной в точке $x_0$. Обозначается $f'(x_0) = \lim\limits_{x \rightarrow x_0} \dfrac{f(x) - f(x_0)}{x - x_0}$.
	\end{definition}
	\begin{statement}
		Если $f$ дифференцируема, то она непрерывна.
	\end{statement}
	\begin{definition}
		Производная (функция) функции $f \rightarrow g(x) = f'(x)$: $x_0 \mapsto \lim\limits_{x \rightarrow x_0} \dfrac{f(x) - f(x_0)}{x - x_0}$; $\varDelta x = x - x_0$; $x_0 \mapsto \lim\limits_{x \rightarrow \varDelta x} \dfrac{f(x + \varDelta x) - f(x)}{\varDelta x}$; $f'(x) = \lim\limits_{\varDelta x \rightarrow 0} \dfrac{f(x + \varDelta x) - f(x)}{\varDelta x}$.
	\end{definition}
	\begin{note}
		$f(\tilde{x}) = f(x) \cdot f'(x)(\tilde{x} - x) + \overline{o}(\tilde{x} - x)$. $f(x + \varDelta x) = f(x) + f'(x) \cdot \varDelta x + \overline{o}(\varDelta x)$; $\varDelta f(x) := f(x + \varDelta x) - f(x)$. $\varDelta f(x) = f'(x) \cdot \varDelta x + \overline{o}(\varDelta x)$, $\varDelta x \rightarrow 0$.
	\end{note}
	\begin{definition}[Дифференциал функции]
		$df(x) := f'(x) \varDelta x$.
	\end{definition}
	\begin{note}
		Пусть $f(x) = x$. $\varDelta x = 1 \cdot \varDelta x + 0$, $\varDelta f(x) = f'(x) \varDelta x + \overline{o}(\varDelta x)$ $\Rightarrow$ $f'(x) = 1$ $\Rightarrow$ $df(x) = \varDelta x$. Тогда $df(x) = f'(x)d(x)$, $f'(x) = \dfrac{df(x)}{dx}$.
	\end{note}
	\begin{note}
		Дифференциал~--- линейная часть малого приращения функции.
	\end{note}
	\subsection{Правила дифференцирования.}
	(рассматриваем дифференцируемые функции):
	\begin{enumerate}
		\item $(c)' = 0$; $(cf(x))' = cf'(x)$
		\item $(f(x) \pm g(x))' = f'(x) \pm g'(x)$
		\item $(f(x) \cdot g(x))' = f'(x) \cdot g(x) + f(x) \cdot g'(x) \rightarrow d(fg) = df \cdot g + dg \cdot f$
		\item $\left( \dfrac{1}{f(x)} \right)' = - \dfrac{f'(x)}{f^2(x)}$, $f(x) \not= 0$
		\item $\left( \dfrac{f(x)}{g(x)} \right)' = \dfrac{f'(x)g(x) - f(x)g'(x)}{g^2(x)}$
		\item Пусть $f$ дифференцируема в точке $x$, $g$ дифференцируема в точке $f(x)$. Тогда $g(f(x))$ дифференцируема в точке $x$ и $(g(f(x)))' = g'(f(x)) \cdot f'(x)$ или $dg(f(x)) = g'(f(x)) \cdot df(x)$
		\item $f$ дифференцируема в точке $x_0$ и $f'(x_0) \not= 0$. Если $f$ обратима в окрестности точки $x_0$, то $f^{-1}$ дифференцируема в точке $y_0 = f(x_0)$ и $(f^{-1})'(y_0) = \dfrac{1}{f'(x_0)} = \dfrac{1}{f'(f^{-1}(y_0))}$
		\item $(\sin x)' = \cos x$
		\item $(\cos x)' = - \sin x$
		\item $(\tg x)' = \dfrac{1}{\cos^2 x}$
		\item $(\ctg x)' = - \dfrac{1}{\sin^2 x}$
		\item $(\arcsin x)' = \dfrac{1}{\sqrt{1 - x^2}}$
		\item $(\arccos x)' = - \dfrac{1}{\sqrt{1 - x^2}}$
		\item $(\arctg x)' = \dfrac{1}{1 + x^2}$
		\item $(\arcctg x)' = - \dfrac{1}{1 + x^2}$
		\item $(e^x)' = e^x$; $(a^x)' = a^x \cdot \ln a$
		\item $(\ln x)' = \dfrac{1}{x}$; $(\log_ax)' = \dfrac{1}{x \ln a}$
		\item $(x^p)' = px^{p - 1}$
	\end{enumerate}
	\subsection{Выпуклость.}
	\begin{definition}
		Функция называется выпуклой на отрезке $[a, b]$ если для каждого отрезка $[x_1, x_2]$, принадлежащего $[a, b]$, график функции f расположен не выше отрезка с концами в $(x_1, f(x_1))$ и $(x_2, f(x_2))$ (строго выпуклой, если ниже). Иными словами, если для любых $x_1, x, x_2 \in [a, b]$, $x_1 < x < x_2$ справедливо неравенство $f(x) \leqslant y(x) = \dfrac{x_2 - x}{x_2 - x_1} f(x_1) + \dfrac{x - x_1}{x_2 - x_1} f(x_2)$, что равносильно $(x_2 - x_1) f(x) \leqslant (x_2 - x) f(x_1) + (x - x_1) f(x_2)$.
	\end{definition}
	\begin{definition}[Из конспекта к зачету]
		Пусть $f: X \rightarrow \mathbb{R}$, $X$~--- выпуклое множество.\\
		$f(x)$~--- выпуклая (выпуклая вниз) на $X$, если
		$$
		\forall x_1, x_2 \in X \qquad f \left( \dfrac{x_1 + x_2}{2} \right) \leqslant \dfrac{f(x_1) + f(x_2)}{2}
		$$
		$f(x)$~--- строго выпуклая на $X$, если
		$$
		\forall x_1, x_2 \in X \qquad f \left( \dfrac{x_1 + x_2}{2} \right) < \dfrac{f(x_1) + f(x_2)}{2}
		$$
		$f(x)$~--- вогнутая (выпуклая вверх) на $X$, если
		$$
		\forall x_1, x_2 \in X \qquad f \left( \dfrac{x_1 + x_2}{2} \right) \geqslant \dfrac{f(x_1) + f(x_2)}{2}
		$$
		$f(x)$~--- строго вогнутая на $X$, если
		$$
		\forall x_1, x_2 \in X \qquad f \left( \dfrac{x_1 + x_2}{2} \right) > \dfrac{f(x_1) + f(x_2)}{2}
		$$
	\end{definition}
	\begin{lemma}
		Пусть функция $f$ определена на отрезке $[a, b]$. Тогда она выпукла тогда и только тогда, когда для любых $x_1, x, x_2 \in [a, b]$, $x_1 < x < x_2$, справедливо любое из следующих трех неравенств:
		$$
		\frac{f(x) - f(x_1)}{x - x_1} \leqslant \frac{f(x_2) - f(x)}{x_2 - x},
		$$
		$$
		\frac{f(x) - f(x_1)}{x - x_1} \leqslant \frac{f(x_2) - f(x_1)}{x_2 - x_1},
		$$
		$$
		\frac{f(x_2) - f(x_1)}{x_2 - x_1} \leqslant \frac{f(x_2) - f(x)}{x_2 - x},
		$$
		причем в случае строгой выпуклости неравенства также будут строгими.
	\end{lemma}
	\begin{proof}
		Запишем $x_2 - x_1$ как $(x_2 - x) + (x - x_1)$, тогда неравенство примет вид
		\[
		(x_2 - x)f(x) + (x - x_1)f(x) \leqslant (x_2 - x)f(x_1) + (x - x_1)f(x_2),
		\]
		после преобразования получаем
		\[
		(x_2 - x)(f(x) - f(x_1)) \leqslant (x - x_1)(f(x_2) - f(x)),
		\]
		что равносильно
		$$
		\frac{f(x) - f(x_1)}{x - x_1} \leqslant \frac{f(x_2) - f(x)}{x_2 - x}.
		$$
		\noindent
		Запишем $x_2 - x$ как $(x_2 - x_1) - (x - x_1)$, тогда неравенство примет вид
		\[
		(x_2 - x_1)f(x) \leqslant (x_2 - x_1)f(x_1) - (x - x_1)f(x_1) + (x - x_1)f(x_2),
		\]
		после преобразования получаем
		\[
		(x_2 - x_1)(f(x) - f(x_1)) \leqslant (x - x_1)(f(x_2) - f(x_1)),
		\]
		что равносильно
		$$
		\frac{f(x) - f(x_1)}{x - x_1} \leqslant \frac{f(x_2) - f(x_1)}{x_2 - x_1}.
		$$
		
		\noindent
		Запишем $x - x_1$ как $(x_2 - x_1) - (x_2 - x)$, тогда неравенство примет вид
		\[
		(x_2 - x_1)f(x) \leqslant (x_2 - x)f(x_1) + (x_2 - x_1)f(x_2) - (x_2 - x)f(x_2),
		\]
		после преобразования получаем
		\[
		(x_2 - x)(f(x_2) - f(x_1)) \leqslant (x_2 - x_1)(f(x_2) - f(x)),
		\]
		что равносильно
		$$
		\frac{f(x_2) - f(x_1)}{x_2 - x_1} \leqslant \frac{f(x_2) - f(x)}{x_2 - x}.
		$$
	\end{proof}
	\begin{definition}[Левая и правая производная]
		$$
		f'_{-}(x_0) = f'(x_0 - 0) := \lim\limits_{h \to +0} \dfrac{f(x_0) - f(x_0 - h)}{h},
		$$
		$$
		f'_{+}(x_0) = f'(x_0 + 0) := \lim\limits_{h \to +0} \dfrac{f(x_0 + h) - f(x_0)}{h}.
		$$
	\end{definition}
	\begin{note}
		$\exists f'(x_0)$ $\Leftrightarrow$ существует производная слева и справа и $f'_{-}(x_0) = f'_{+}(x_0)$.
	\end{note}
	\begin{theorem}
		Пусть функция $f$ выпукла на $[a, b]$. Тогда
		\begin{enumerate}[a)]
			\item $\forall x_0 \in (a, b)$ $\exists f'_-(x_0)$ и $f'_+(x_0)$, причем $f'_-(x_0) \leqslant f'_+(x_0)$
			\item $f \in C(a, b)$
			\item $\forall x_1, x_2 \in (a, b)$, $x_1 < x_2$, имеем $f'_+(x_1) \leqslant f'_-(x_2)$ (если $f$ строго выпукла, то $f'_+(x_1) < f'_-(x_2)$)
			\item $f'_-$ и $f'_+$~--- неубывающие на $(a, b)$ функции (возрастающие в случае строгой выпуклости)
			\item $f$ дифференцируема на $(a, b)$ всюду, кроме, быть может, не более чем счетного числа точек
		\end{enumerate}
	\end{theorem}
	\begin{proof}
		\begin{enumerate}[a)]
			\item Пусть $x_0 \in (a, b)$, $h > 0$~--- такое, что $x_0 + h \in (a, b)$ тоже. Тогда, учитывая, что $x_0 - (x_0 - h) = (x_0 + h) - x_0 = h$, по неравенству $\dfrac{f(x) - f(x_1)}{x - x_1} \leqslant \dfrac{f(x_2) - f(x)}{x_2 - x}$ имеем
			\[
			\frac{f(x_0) - f(x_0 - h)}{h} \leqslant \frac{f(x_0 + h) - f(x_0)}{h};
			\]
			при убывании $h$ левая часть неравенства неубывает в силу $\dfrac{f(x_2) - f(x_1)}{x_2 - x_1} \leqslant \dfrac{f(x_2) - f(x)}{x_2 - x}$, правая — невозрастает в силу $\dfrac{f(x) - f(x_1)}{x - x_1} \leqslant \dfrac{f(x_2) - f(x_1)}{x_2 - x_1}$. Следовательно при $h \to +0$ существуют оба предела, равные $f'_-(x_0)$ и $f'_+(x_0)$, а из неравенства получаем $f'_-(x_0) \leqslant f'_+(x_0)$.
			\item Из существования односторонних производных имеем для любого $x_0 \in (a, b)$
			\[
			f(x_0 + h) = 
			\begin{cases}
				f(x_0) + f'_+(x_0)h + \overline{o}(h), & h \to +0 \\
				f(x_0) + f'_-(x_0)h + \overline{o}(h), & h \to -0
			\end{cases}
			\]
			откуда в любом случае получаем $f(x_0 + h) = f(x_0) + \overline{o}(1), h \to 0$, т.е. $f$ непрерывна в $x_0$.
			\item Для всякого $x \in (x_1, x_2)$ имеем
			\[
			\frac{f(x) - f(x_1)}{x - x_1} \leqslant \frac{f(x_2) - f(x_1)}{x_2 - x_1}
			\]
			(строгое неравенство в случае строгой выпуклости).
			
			При $x \to x_1 + 0$ левая часть неравенства невозрастает (строгое неравенство при этом сохраняется), поэтому мы можем перейти к пределу:
			\[
			f'_+(x_1) \leqslant \frac{f(x_2) - f(x_1)}{x_2 - x_1}.
			\]
			
			Аналогично, при $x \to x_2 - 0$ имеем
			\[
			\frac{f(x_2) - f(x_1)}{x_2 - x_1} \leqslant \frac{f(x_2) - f(x)}{x_2 - x}
			\]
			(строгое неравенство в случае строгой выпуклости).
			
			Таким образом, имеем
			\[
			f'_+(x_1) \leqslant \frac{f(x_2) - f(x_1)}{x_2 - x_1} \leqslant f'_-(x_2)
			\]
			(строгое неравенство в случае строгой выпуклости), откуда следует требуемое.
			\item Для $x_1, x_2 \in (a, b), x_1 < x_2$ из a) и c) имеем $f'_-(x_1) \leqslant f'_+(x_1) \leqslant f'_-(x_2) \leqslant f'_+(x_2)$ (в случае строгой выпуклости $f'_-(x_1) \leqslant f'_+(x_1) < f'_-(x_2) \leqslant f'_+(x_2)$), откуда следует требуемая монотонность.
			\item Для доказательства потребуется следующая лемма:
			\begin{lemma}
				Пусть $g(x)$ монотонная на $[a, b]$. Тогда $g(x)$ не может иметь разрывов второго рода и имеет не более чем счетное число разрывов (устранимых или первого рода).
			\end{lemma}
			В силу монотонности множество точек разрыва функции $f'_-$ на $(a, b)$ не более чем счетно. Покажем, что в точках непрерывности функции $f'_-$ существует производная $f'$ (т.е. $f'_- = f'_+$). 
			
			Пусть $x_0$~--- точка непрерывности функции $f'_-$. Для всякого $h > 0$ такого, что $x_0 + h \in (a, b)$ из а) и c) имеем
			\[
			f'_-(x_0) \leqslant f'_+(x_0) \leqslant f'_-(x_0 + h),
			\]
			откуда
			\[
			0 \leqslant f'_+(x_0) - f'_-(x_0) \leqslant f'_-(x_0 + h) - f'_-(x_0).
			\]
			
			В силу непрерывности $f'_-$ имеем $f'_-(x_0 + h) - f'_-(x_0) \rightarrow 0$, откуда по теореме о зажатой функции $f'_+(x_0) - f'_-(x_0) \rightarrow 0$, а так как это выражение на самом деле не зависит от $h$, получаем, что $f'_+(x_0) = f'_-(x_0)$.
		\end{enumerate}
	\end{proof}
	\subsection{Вторая производная.}
	\begin{definition}
		Пусть $f(x)$ дифференцируема $\forall x \in X$, тогда $f'(x): X \rightarrow \mathbb{R}$. $(f')' =: f''$.
	\end{definition}
	\subsubsection{Многочлен Тейлора.}
	\begin{definition}[Многочлен Тейлора]
		$T_{n, x_0} f = f(x_0) + f'(x_0)(x - x_0) + \dfrac{f''(x_0)}{2}(x - x_0)^2 + \dfrac{f'''(x_0)}{3!}(x - x_0)^3 + \dots + \dfrac{f^{(n)}(x_0)}{n!}(x - x_0)^n = \sum\limits_{k = 0}^{n} \dfrac{f^{(k)}(x_0)}{k!}(x - x_0)^k$.
	\end{definition}
	\subsubsection{Формула Тейлора.}
	\begin{theorem}[Формула Тейлора.]
		$f(x) = T_{n, x_0} f + R_{n, x_0} f$. Остатки бывают в разных формах (например в форме Пеано это $\overline{o}((x - x_0)^n)$).
	\end{theorem}
	\begin{proof}
		$f(x) = T_{n, x_0} f + \overline{o} ((x - x_0)^n) \Leftrightarrow f(x) - T_{n, x_0} f = \overline{o}((x - x_0)^n) \Leftrightarrow \lim\limits_{x \rightarrow x_0} \dfrac{f(x) - T_{n, x_0} f}{(x - x_0)^n} = \left[ \dfrac{0}{0} \right] \overset{\text{правило Лопиталя}}{=} \lim\limits_{x \rightarrow x_0} \dfrac{f'(x) - (f'(x_0) + f''(x_0)(x - x_0) + \dots + \frac{f^{(n)}}{(n - 1)!} (x - x_0)^{n - 1})}{n(x - x_0)^{n - 1}} = \dots =$\\
		$= \lim\limits_{x \rightarrow x_0} \dfrac{f^{(n)}(x) - f^{(n)}(x_0)}{n!} = 0$.
	\end{proof}
	\subsubsection{Теорема Ферма.}
	\begin{theorem}[Ферма]
		Пусть $f \in C([a, b]) \cap C^1(a, b)$. Тогда, если $x_0 \in (a, b)$ точка экстремума (точка локального максимума или минимума), то $f'(x_0) = 0$.
	\end{theorem}
	\begin{proof}
		$f'(x_0) = f'_-(x_0) = f'_+(x_0)$
		
		$x_0$~--- локальный максимум:
		\begin{align*}
			f'_+(x_0) = \lim\limits_{\varDelta \to +0} \dfrac{\overbrace{f(x_0 + \varDelta x)}^{\leqslant f(x_0)} - f(x_0)}{\varDelta x} \leqslant 0\\
			f'_-(x_0) = \lim\limits{\varDelta x \to +0} \dfrac{f(x_0) - \overbrace{f(x_0 - \varDelta x)}^{\leqslant f(x_0)}}{\varDelta x} \geqslant 0\\
			\Rightarrow f'(x_0) = 0
		\end{align*}
	\end{proof}
	\begin{corollary}
		В точке экстремума производная нулевая или не существует.
	\end{corollary}
	\subsubsection{Немного определений.}
	\begin{definition}
		$x_0$ стационарная, если $f'(x_0) = 0$.
	\end{definition}
	\begin{definition}
		$x_0$ критическая, если $f'(x_0) = 0$ или $f'(x_0) \nexists$
	\end{definition}
	\begin{definition}
		$x_0$ экстремум, если это локальный минимум или максимум.
	\end{definition}
	\subsubsection{Теорема Ролля.}
	\begin{theorem}[Ролля]
		Пусть $f \in C([a, b]) \cap C^1(a, b)$, $f(a) = f(b)$. Тогда $\exists c \in (a, b)$: $f'(c) = 0$.
	\end{theorem}
	\begin{proof}
		$f \in C([a, b]) \Rightarrow$ достигается максимум и минимум $\Rightarrow$ хотя бы одна из них $c \in (a, b) \Rightarrow$ по теореме Ферма $f'(c) = 0$.
	\end{proof}
	\subsubsection{Теорема Коши.}
	\begin{theorem}[Коши]
		$f, g \in C([a, b]) \cap C^1(a, b)$, $g'(x) \ne 0$ на $(a, b)$. Тогда $\exists c \in (a, b)$: $\dfrac{f(b) - f(a)}{g(b) - g(a)} = \dfrac{f'(c)}{g'(c)}$.
	\end{theorem}
	\begin{proof}
		Рассмотрим функцию $h(x) = f(x) - k g(x)$, $k \in \mathbb{R}$: $\in C^1(a, b) \cap C([a, b])$.
		
		Выберем $k$, такое что $h(a) = h(b)$. $f(a) - k g(a) = f(b) - k g(b) \Rightarrow k(g(b) - g(a)) = f(b) - f(a) \Rightarrow k = \dfrac{f(b) - f(a)}{g(b) - g(a)}$.
		
		По теореме Ролля $\exists c \in (a, b)$: $h'(c) = 0$
		
		$f'(c) - k g'(c) = 0$
	\end{proof}
	\subsubsection{Теорема Лагранжа.}
	\begin{theorem}[Лагранжа о среднем в дифференциальной форме.]
		$f \in C([a, b]) \cap C^1(a, b)$. Тогда $\exists c \in (a, b)$: $f(b) - f(a) = f'(c)(b - a)$.
	\end{theorem}
	\begin{proof}
		Следует из теоремы Коши, если взять $g(x) = x$.
	\end{proof}
	\subsubsection{Правило Лопиталя.}
	\begin{theorem}[Правило Лопиталя.]
		Пусть $f(x)$, $g(x)$ непрерывно дифференцируемы в окрестности точки $x_0$, $x_0 \in \mathbb{R}$. Пусть $\lim\limits_{x \rightarrow x_0} \dfrac{f(x)}{g(x)} = \left[ \dfrac{\infty}{\infty} \right]$ или $\left[ \dfrac{0}{0} \right]$. Тогда, если $\exists \lim\limits_{x \rightarrow x_0} \dfrac{f'(x)}{g'(x)} = a$, то $\exists \lim\limits_{x \rightarrow x_0} \dfrac{f(x)}{g(x)} = a$.
	\end{theorem}
	\begin{proof}
		\begin{enumerate}
			\item $x_0 \in \mathbb{R}$, $\left[ \dfrac{0}{0} \right] \Rightarrow f(x_0) = g(x_0) = 0$
			
			$\lim\limits_{x \to x_0} \dfrac{f(x)}{g(x)} = \lim\limits_{x \to x_0} \dfrac{f(x) - f(x_0)}{g(x) - x(x_0)} \overset{\text{по теореме Коши}}{=} \lim\limits_{x \to x_0} \dfrac{f'(c(x))}{g'(c(x))} = \lim\limits_{c \to x_0} \dfrac{f)}{g'(c)} = a$.
			
			\item $x_0 \in \mathbb{R}$, $\left[ \dfrac{\infty}{\infty} \right]$
			
			$\dfrac{f(x)}{g(x)} = \dfrac{\frac{1}{g(x)}}{\frac{1}{f(x)}} = \dfrac{g_1(x)}{f_1(x)}$, пусть
			$$
			f_1(x) =
			\begin{cases}
				\dfrac{1}{f(x)}, x \ne x_0\\
				0, x = x_0
			\end{cases}
			$$
			$$
			g_1(x) =
			\begin{cases}
				\dfrac{1}{g(x)}, x \ne x_0\\
				0, x = x_0
			\end{cases}
			$$
			$f_1, g_1$ непрерывно дифференцируемы.
			
			\item $x_0 = \pm \infty$ (НУО рассмотрим только $+ \infty$); $a \in \mathbb{R}$
			
			fix $\varepsilon > 0$: $\exists \delta > 0$: $\left| \dfrac{f'(x)}{g'(x)} - a \right| < \varepsilon \quad \forall x > \dfrac{1}{\delta}$
			
			fix $y \in (\dfrac{1}{\delta}; + \infty) \quad \exists c$ между $x$ и $y$: $\dfrac{f(x) - f(y)}{g(x) - g(y)} = \dfrac{f'(c)}{g'(c)}$
			
			$$
			\dfrac{f(x)(1 - \frac{f(x)}{f(x)})}{g(x)(1 - \frac{g(y)}{g(x)})} = \dfrac{f'(c)}{g'(c)}
			$$
			$$
			\dfrac{f(x)}{g(x)} = \dfrac{f'(c)}{g'(c)} \cdot \dfrac{1 - \frac{g(y)}{g(x)}}{1 - \frac{f(y)}{f(x)}} = \dfrac{f'(c)}{g'(c)} \cdot \left( 1 + \underbrace{\dfrac{\overbrace{\frac{f(y)}{f(x)}}^{\to 0} - \overbrace{\frac{g(y)}{g(x)}}^{\to 0}}{1 - \underbrace{\frac{f(y)}{f(x)}}_{\to 0}}}_{\text{б. м.}} \right) > \dfrac{1}{2} \cdot \dfrac{f'(c)}{g'(c)} > \dfrac{1}{2 \varepsilon} \quad \forall x \geqslant \dfrac{1}{2 \delta}
			$$
			
			\item $x_0 = \pm \infty$ (НУО рассмотрим только $+ \infty$); $\left[ \dfrac{0}{0} \right]$
			
			$$
			\lim\limits_{x \to + \infty} \dfrac{f(x)}{g(x)} \underset{x = \frac{1}{t}}{=} \lim\limits_{t \to +0} \dfrac{f(\frac{1}{t})}{g(\frac{1}{t})} = \left[ \dfrac{0}{0} \right]
			$$
			
			Рассмотрим
			$$
			\exists \lim \dfrac{(f(\frac{1}{t}))'}{(g(\frac{1}{t}))'} = \lim\limits_{t \to +0} \dfrac{f'(\frac{1}{t}) \cdot (- \frac{1}{t^2})}{g'(\frac{1}{t}) \cdot (- \frac{1}{t^2})} = \lim\limits_{x \to + \infty} \dfrac{f'(x)}{g'(x)} = a
			$$
		\end{enumerate}
	\end{proof}
	\paragraph{Применение Лопиталя.}
	\begin{lemma}
		$\forall p > 0 \quad \forall a > 1 \qquad x^p = \overline{o}(a^x)$, $x \to + \infty$.
	\end{lemma}
	\begin{lemma}
		$\forall p > 0 \quad \forall a \in (0; 1) \cup (1; + \infty) \qquad \log_a x = \overline{o}(x^p)$, $x \to + \infty$.
	\end{lemma}
	\begin{lemma}
		$\forall \alpha > 0 \quad \forall p > 0 \quad \forall a \in (0; 1) \cup (1; + \infty) \qquad \log_a^\alpha x = \overline{o}(x^p)$, $x \to + \infty$.
	\end{lemma}
	\subsubsection{Утверждения.}
	\begin{statement}
		$\exists f'(x_0) \Leftrightarrow f(x) = f(x_0) + A \cdot (x - x_0) + \overline{o} (x - x_0)$ и $A = f'(x_0)$.
	\end{statement}
	\begin{statement}
		$f$ непрерывно дифференцируема $n$ раз в некоторой окрестности точки $x_0$ $\Rightarrow$ $f(x) = f(x_0) + f'(x_0)(x - x_0) + \dfrac{f''(x_0)}{2!}(x - x_0)^2 + \dots + \dfrac{f^{(n)}(x_0)}{n!}(x - x_0)^n + \overline{o}((x - x_0)^n)$.
	\end{statement}
	\subsubsection{Применение Тейлора.}
	\begin{statement}
		$e^x = 1 + x + \dfrac{x^2}{2!} + \dfrac{x^3}{3!} + \dots + \dfrac{x^n}{n!} + \overline{o}(x^n)$.
	\end{statement}
	\begin{statement}
		$\sin x = 0 + x + \dfrac{0}{2!}x^2 - \dfrac{1}{3!}x^3 + 0 \cdot x^4 + \dfrac{1}{3!}x^5 + \dots = x - \dfrac{x^3}{3!} + \dfrac{x^5}{5!} - \dots + (-1)^n \dfrac{x^{2n + 1}}{(2n + 1)!} + \overline{o}(x^{2n + 2})$.
	\end{statement}
	%\begin{statement}
	%	$\cos x =$
	%\end{statement}
	%\begin{statement}
	%	$(1 + x)^{\alpha} =$
	%\end{statement}
	
	\section{Построение графика функции и анализ.}
	
	$y = f(x)$
	
	\begin{enumerate}
		\item $D(f)$~--- смотрим область определения, поведение в точках разрыва.
		\item Точки пересечения с осями и промежутки знакопостоянства.
		\item Поведение на $\pm \infty$ (асимптоты).\\
		Прямая $y = kx + b$, асимптота $y = f(x)$ $\Leftrightarrow$ $f(x) - (kx + b) = \overline{o}(1)$
		\begin{statement}
			$y = kx + b$ является асимптотой для $y = f(x)$ при $x \to +\infty$ $\Leftrightarrow$ $k = \lim\limits_{x \to +\infty} \dfrac{f(x)}{x}$; $b = \lim\limits_{x \to +\infty} (f(x) - kx)$.
		\end{statement}
		\item Исследование на четность/нечетность.
		\item Исследование на монотонность и точки экстремума.
		\begin{theorem}
			Пусть $f$ непрерывно дифференцируема на $\langle a; b \rangle$. Если:
			\begin{itemize}
				\item $f' \geqslant 0$ на $\langle a; b \rangle$, то $f \uparrow$ на $\langle a; b \rangle$.
				\item $f' > 0$ на $\langle a; b \rangle$, то $f \twoheaduparrow$ на $\langle a; b \rangle$.
				\item $f' \leqslant 0$ на $\langle a; b \rangle$, то $f \downarrow$ на $\langle a; b \rangle$.
				\item $f' < 0$ на $\langle a; b \rangle$, то $f \twoheaddownarrow$ на $\langle a; b \rangle$.
			\end{itemize}
		\end{theorem}
		\begin{proof}
			$\forall x_1 < x_2$, $x_1, x_2 \in \langle a; b \rangle \overset{?}{\Rightarrow} f(x_1) \geqslant f(x_2)$.\\
			$f$ непрерывно дифференцируема на $[x_1, x_2] \Rightarrow \exists c \in [x_1, x_2]: f(x_2) - f(x_1) = f'(x)(x_2 - x_1)$.
		\end{proof}
		
		\begin{theorem}
			Пусть $f \uparrow$ на $\langle a; b \rangle$ и дифференцируема на $\langle a; b \rangle$ $\Rightarrow$ $f' \geqslant 0$ на $\langle a; b \rangle$.
		\end{theorem}
		\begin{proof}
			$f'(x) = \lim\limits_{\varDelta x \to 0} \dfrac{f(x + \varDelta x) - f(x)}{\varDelta x} \geqslant 0$.
		\end{proof}
		\item Исследование на выпуклость.\\
		Если $f \in C^2(\langle a; b \rangle)$, то если:
		\begin{itemize}
			\item Если $f'' \geqslant 0$, то $f$ выпукла вниз.
			\item Если $f'' > 0$, то $f$ строго выпукла вниз.
			\item Если $f'' \leqslant 0$, то $f$ выпукла вверх.
			\item Если $f'' < 0$, то $f$ строго выпукла вверх.
		\end{itemize}
		\item Строим график.
		\item $E(f)$.
	\end{enumerate}
	
	\subsection{Регулярные и сингулярные точки.}
	
	\begin{definition}[Регулярная точка]
		$\forall t \in \langle t_0 - \varepsilon; t_0 + \varepsilon \rangle: \begin{cases}
			x = x(t)\\
			y = y(t)
		\end{cases}$ является графиком функции.
	\end{definition}
\end{document}
